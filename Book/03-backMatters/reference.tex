%%%%%%%%%%%%%%%%%%%%%%%% referenc.tex %%%%%%%%%%%%%%%%%%%%%%%%%%%%%%
% sample references
% %
% Use this file as a template for your own input.
%
%%%%%%%%%%%%%%%%%%%%%%%% Springer-Verlag %%%%%%%%%%%%%%%%%%%%%%%%%%
%
% BibTeX users please use
% \bibliographystyle{}
% \bibliography{}
%
% \biblstarthook{In view of the parallel print and (chapter-wise) online publication of your book at \url{www.springerlink.com} it has been decided that -- as a genreral rule --  references should be sorted chapter-wise and placed at the end of the individual chapters. However, upon agreement with your contact at Springer you may list your references in a single seperate chapter at the end of your book. Deactivate the class option \texttt{sectrefs} and the \texttt{thebibliography} environment will be put out as a chapter of its own.\\\indent
% References may be \textit{cited} in the text either by number (preferred) or by author/year.\footnote{Make sure that all references from the list are cited in the text. Those not cited should be moved to a separate \textit{Further Reading} section or chapter.} The reference list should ideally be \textit{sorted} in alphabetical order -- even if reference numbers are used for the their citation in the text. If there are several works by the same author, the following order should be used: 
% \begin{enumerate}
% \item all works by the author alone, ordered chronologically by year of publication
% \item all works by the author with a coauthor, ordered alphabetically by coauthor
% \item all works by the author with several coauthors, ordered chronologically by year of publication.
% \end{enumerate}
% The \textit{styling} of references\footnote{Always use the standard abbreviation of a journal's name according to the ISSN \textit{List of Title Word Abbreviations}, see \url{http://www.issn.org/en/node/344}} depends on the subject of your book:
% \begin{itemize}
% \item The \textit{two} recommended styles for references in books on \textit{mathematical, physical, statistical and computer sciences} are depicted in ~\cite{science-contrib, science-online, science-mono, science-journal, science-DOI} and ~\cite{phys-online, phys-mono, phys-journal, phys-DOI, phys-contrib}.
% \item Examples of the most commonly used reference style in books on \textit{Psychology, Social Sciences} are~\cite{psysoc-mono, psysoc-online,psysoc-journal, psysoc-contrib, psysoc-DOI}.
% \item Examples for references in books on \textit{Humanities, Linguistics, Philosophy} are~\cite{humlinphil-journal, humlinphil-contrib, humlinphil-mono, humlinphil-online, humlinphil-DOI}.
% \item Examples of the basic Springer style used in publications on a wide range of subjects such as \textit{Computer Science, Economics, Engineering, Geosciences, Life Sciences, Medicine, Biomedicine} are ~\cite{basic-contrib, basic-online, basic-journal, basic-DOI, basic-mono}. 
% \end{itemize}
% }

\begin{thebibliography}{99.}%
% % and use \bibitem to create references.
% %
% % Use the following syntax and markup for your references if 
% % the subject of your book is from the field 
% % "Mathematics, Physics, Statistics, Computer Science"
% %
% % Contribution 
% \bibitem{science-contrib} Broy, M.: Software engineering --- from auxiliary to key technologies. In: Broy, M., Dener, E. (eds.) Software Pioneers, pp. 10-13. Springer, Heidelberg (2002)
% %
% % Online Document
% \bibitem{science-online} Dod, J.: Effective substances. In: The Dictionary of Substances and Their Effects. Royal Society of Chemistry (1999) Available via DIALOG. \\
% \url{http://www.rsc.org/dose/title of subordinate document. Cited 15 Jan 1999}
% %
% % Monograph
% \bibitem{science-mono} Geddes, K.O., Czapor, S.R., Labahn, G.: Algorithms for Computer Algebra. Kluwer, Boston (1992) 
% %
% % Journal article
% \bibitem{science-journal} Hamburger, C.: Quasimonotonicity, regularity and duality for nonlinear systems of partial differential equations. Ann. Mat. Pura. Appl. \textbf{169}, 321--354 (1995)
% %
% % Journal article by DOI
% \bibitem{science-DOI} Slifka, M.K., Whitton, J.L.: Clinical implications of dysregulated cytokine production. J. Mol. Med. (2000) doi: 10.1007/s001090000086 
% %
% \bigskip

% % Use the following (APS) syntax and markup for your references if 
% % the subject of your book is from the field 
% % "Mathematics, Physics, Statistics, Computer Science"
% %
% % Online Document
% \bibitem{phys-online} J. Dod, in \textit{The Dictionary of Substances and Their Effects}, Royal Society of Chemistry. (Available via DIALOG, 1999), 
% \url{http://www.rsc.org/dose/title of subordinate document. Cited 15 Jan 1999}
% %
% % Monograph
% \bibitem{phys-mono} H. Ibach, H. L\"uth, \textit{Solid-State Physics}, 2nd edn. (Springer, New York, 1996), pp. 45-56 
% %
% % Journal article
% \bibitem{phys-journal} S. Preuss, A. Demchuk Jr., M. Stuke, Appl. Phys. A \textbf{61}
% %
% % Journal article by DOI
% \bibitem{phys-DOI} M.K. Slifka, J.L. Whitton, J. Mol. Med., doi: 10.1007/s001090000086
% %
% % Contribution 
% \bibitem{phys-contrib} S.E. Smith, in \textit{Neuromuscular Junction}, ed. by E. Zaimis. Handbook of Experimental Pharmacology, vol 42 (Springer, Heidelberg, 1976), p. 593
% %
% \bigskip
% %
% % Use the following syntax and markup for your references if 
% % the subject of your book is from the field 
% % "Psychology, Social Sciences"
% %
% %
% % Monograph
% \bibitem{psysoc-mono} Calfee, R.~C., \& Valencia, R.~R. (1991). \textit{APA guide to preparing manuscripts for journal publication.} Washington, DC: American Psychological Association.
% %
% % Online Document
% \bibitem{psysoc-online} Dod, J. (1999). Effective substances. In: The dictionary of substances and their effects. Royal Society of Chemistry. Available via DIALOG. \\
% \url{http://www.rsc.org/dose/Effective substances.} Cited 15 Jan 1999.
% %
% % Journal article
% \bibitem{psysoc-journal} Harris, M., Karper, E., Stacks, G., Hoffman, D., DeNiro, R., Cruz, P., et al. (2001). Writing labs and the Hollywood connection. \textit{J Film} Writing, 44(3), 213--245.
% %
% % Contribution 
% \bibitem{psysoc-contrib} O'Neil, J.~M., \& Egan, J. (1992). Men's and women's gender role journeys: Metaphor for healing, transition, and transformation. In B.~R. Wainrig (Ed.), \textit{Gender issues across the life cycle} (pp. 107--123). New York: Springer.
% %
% % Journal article by DOI
% \bibitem{psysoc-DOI}Kreger, M., Brindis, C.D., Manuel, D.M., Sassoubre, L. (2007). Lessons learned in systems change initiatives: benchmarks and indicators. \textit{American Journal of Community Psychology}, doi: 10.1007/s10464-007-9108-14.
% %
% %
% % Use the following syntax and markup for your references if 
% % the subject of your book is from the field 
% % "Humanities, Linguistics, Philosophy"
% %
% \bigskip
% %
% % Journal article
% \bibitem{humlinphil-journal} Alber John, Daniel C. O'Connell, and Sabine Kowal. 2002. Personal perspective in TV interviews. \textit{Pragmatics} 12:257--271
% %
% % Contribution 
% \bibitem{humlinphil-contrib} Cameron, Deborah. 1997. Theoretical debates in feminist linguistics: Questions of sex and gender. In \textit{Gender and discourse}, ed. Ruth Wodak, 99--119. London: Sage Publications.
% %
% % Monograph
% \bibitem{humlinphil-mono} Cameron, Deborah. 1985. \textit{Feminism and linguistic theory.} New York: St. Martin's Press.
% %
% % Online Document
% \bibitem{humlinphil-online} Dod, Jake. 1999. Effective substances. In: The dictionary of substances and their effects. Royal Society of Chemistry. Available via DIALOG. \\
% http://www.rsc.org/dose/title of subordinate document. Cited 15 Jan 1999
% %
% % Journal article by DOI
% \bibitem{humlinphil-DOI} Suleiman, Camelia, Daniel C. O’Connell, and Sabine Kowal. 2002. `If you and I, if we, in this later day, lose that sacred fire...´': Perspective in political interviews. \textit{Journal of Psycholinguistic Research}. doi: 10.1023/A:1015592129296.
% %
% %
% %
% \bigskip
% %
% %
% % Use the following syntax and markup for your references if 
% % the subject of your book is from the field 
% % "Computer Science, Economics, Engineering, Geosciences, Life Sciences"
% %
% %
% % Contribution 
% \bibitem{basic-contrib} Brown B, Aaron M (2001) The politics of nature. In: Smith J (ed) The rise of modern genomics, 3rd edn. Wiley, New York 
% %
% % Online Document
% \bibitem{basic-online} Dod J (1999) Effective Substances. In: The dictionary of substances and their effects. Royal Society of Chemistry. Available via DIALOG. \\
% \url{http://www.rsc.org/dose/title of subordinate document. Cited 15 Jan 1999}
% %
% % Journal article by DOI
% \bibitem{basic-DOI} Slifka MK, Whitton JL (2000) Clinical implications of dysregulated cytokine production. J Mol Med, doi: 10.1007/s001090000086
% %
% % Journal article
% \bibitem{basic-journal} Smith J, Jones M Jr, Houghton L et al (1999) Future of health insurance. N Engl J Med 965:325--329
% %
% % Monograph
% \bibitem{basic-mono} South J, Blass B (2001) The future of modern genomics. Blackwell, London 
% %
\bibitem{CPSTAT-L5} Bisdorff R. (2017) ``Simulating from abitrary empirical random distributions''. MICS \emph{Computational Statistics} course, Lecture 5. FSTC/ILIAS University of Luxembourg, Winter Semester 2017 (see \url{http://hdl.handle.net/10993/37933}).

\bibitem{BIS-2016} Bisdorff R. (2016). ``Computing linear rankings from trillions of pairwise outranking situations''. In Proceedings of DA2PL'2016 \emph{From Multiple Criteria Decision Aid to Preference Learning}, R. Busa-Fekete, E. Hüllermeier, V. Mousseau and K. Pfannschmidt (Eds.), University of Paderborn (Germany), Nov. 7-8 2016: 1-6 (downloadable \href{http://hdl.handle.net/10993/28613}{PDF file 451.4 kB}).
	      
\bibitem{BIS-2015} Bisdorff R. (2015). ``The EURO 2004 Best Poster Award: Choosing the Best Poster in a Scientific Conference''. Chapter 5 in R. Bisdorff, L. Dias, P. Meyer, V. Mousseau, and M. Pirlot (Eds.), \emph{Evaluation and Decision Models with Multiple Criteria: Case Studies.} Springer-Verlag Berlin Heidelberg, International Handbooks on Information Systems, DOI 10.1007/978-3-662-46816-6\_1, pp. 117-166 (downloadable \href{http://hdl.handle.net/10993/23714}{PDF file 754.7 kB}).
	      
\bibitem{ADT-L2} Bisdorff R. (2020)  ``Who wins the election?'' MICS \emph{Algorithmic Decision Theory} course, Lecture 2. FSTC/ILIAS University of Luxembourg, Summer Semester 2020 (see \url{http://hdl.handle.net/10993/37933}).

\bibitem{ADT-L7} Bisdorff R.(2014)  ``Best multiple criteria choice: the Rubis outranking method''. MICS \emph{Algorithmic Decision Theory} course, Lecture 7. FSTC/ILIAS University of Luxembourg, Summer Semester 2014 (see \url{http://hdl.handle.net/10993/37933}).

\bibitem{BIS-2013} Bisdorff R. (2013) ``On Polarizing Outranking Relations with Large Performance Differences'' \emph{Journal of Multi-Criteria Decision Analysis} (Wiley) 20:3-12 (see \url{http://hdl.handle.net/10993/245}).

\bibitem{BIS-2012} Bisdorff R. (2012). ``On measuring and testing the ordinal correlation between bipolar outranking relations''. In Proceedings of DA2PL’2012 \emph{From Multiple Criteria Decision Aid to Preference Learning}, University of Mons 91-100. (see \url{http://hdl.handle.net/10993/23909}).

\bibitem{DIA-2010} Dias L.C. and Lamboray C. (2010). ``Extensions of the prudence principle to exploit a valued outranking relation''. \emph{European Journal of Operational Research} Volume 201 Number 3 pp. 828-837.

\bibitem{LAM-2009} Lamboray C. (2009) ``A prudent characterization of the Ranked Pairs Rule''. \emph{Social Choice and Welfare} 32 pp. 129-155.

\bibitem{BIS-2008} Bisdorff R., Meyer P. and Roubens M.(2008) ``RUBIS: a bipolar-valued outranking method for the choice problem''. 4OR, \emph{A Quarterly Journal of Operations Research} Springer-Verlag, Volume 6,  Number 2 pp. 143-165. (Online) Electronic version: DOI: 10.1007/s10288-007-0045-5 (\href{http://hdl.handle.net/10993/23716}{downloadable preliminary version}).

\bibitem{ISOMIS-08} Bisdorff R. and Marichal J.-L. (2008). ``Counting non-isomorphic maximal independent sets of the n-cycle graph''. \emph{Journal of Integer Sequences}, Vol. 11 Article 08.5.7 (`openly accessible \href{https://cs.uwaterloo.ca/journals/JIS/VOL11/Marichal/marichal.html}{here}).

\bibitem{NR3-2007} Press W.H., Teukolsky S.A., Vetterling W.T. and Flannery B.P. (2007) ``Single-Pass Estimation of Arbitrary Quantiles'' Section 5.8.2 in \emph{Numerical Recipes: The Art of Scientific Computing 3rd Ed.}, Cambridge University Press, pp 435-438.

\bibitem{CHAM-2006} Chambers J.M., James D.A., Lambert D. and Vander Wiel S. (2006) ``Monitoring Networked Applications with Incremental Quantile Estimation''. \emph{Statistical Science}, Vol. 21, No.4, pp.463-475. DOI: 10 12140/088342306000000583.

\bibitem{BIS-2006a} Bisdorff R., Pirlot M. and Roubens M. (2006). ``Choices and kernels from bipolar valued digraphs''. \emph{European Journal of Operational Research}, 175 (2006) 155-170. (Online) Electronic version: DOI:10.1016/j.ejor.2005.05.004 (downloadable preliminary version \href{http://hdl.handle.net/10993/23720}{PDF file 257.3Kb}).

\bibitem{BIS-2006b} Bisdorff R. (2006). ``On enumerating the kernels in a bipolar-valued digraph''. Annales du Lamsade 6, Octobre 2006, pp. 1 - 38. Université Paris-Dauphine. ISSN 1762-455X (downloadable version \href{http://hdl.handle.net/10993/38741}{PDF file 532.2 Kb}).

\bibitem{BIS-2004a} Bisdorff R. (2004) ``On a natural fuzzification of Boolean logic''. In Erich Peter Klement and Endre Pap (editors), Proceedings of the 25th Linz Seminar on \emph{Fuzzy Set Theory, Mathematics of Fuzzy Systems}. Bildungszentrum St. Magdalena, Linz (Austria), February 2004. pp. 20-26 (for downloading \href{http://hdl.handle.net/10993/38740}{PDF file (133.4 Kb)})

\bibitem{BIS-2004b} Bisdorff R. (2004) ``Concordant Outranking with multiple criteria of ordinal significance''. 4OR, \emph{Quarterly Journal of the Belgian, French and Italian Operations Research Societies}, Springer-Verlag, Issue: Volume 2, Number 4, December 2004, Pages: 293 - 308. [ISSN: 1619-4500 (Paper) 1614-2411 (Online)] Electronic version: DOI: 10.1007/s10288-004-0053-7 (for downloading \href{http://hdl.handle.net/10993/23721}{PDF file 137.1Kb})
	       
\bibitem{GOL-2004} Golumbic M.Ch. (2004), \emph{Agorithmic Graph Theory and Perfect Graphs} 2nd Ed., Annals of Discrete Mathematics 57, Elsevier.

\bibitem{FMCAA} Häggström O. (2002) \emph{Finite Markov Chains and Algorithmic Applications}. Cambridge University Press.

\bibitem{BIS-2000} Bisdorff R. (2000), ``Logical foundation of fuzzy preferential systems with application to the ELECTRE decision aid methods'', \emph{Computers and Operations Research}, 27:673-687 (downloadable version \href{http://hdl.handle.net/10993/23724}{PDF file 159.1Kb}).

\bibitem{BIS-1999} Bisdorff R. (1999), ``Bipolar ranking from pairwise fuzzy outrankings'', JORBEL \emph{Belgian Journal of Operations Research, Statistics and Computer Science}, Vol. 37 (4) 97 379-387. (for downloading \href{http://hdl.handle.net/10993/38738}{PDF file (351.7 Kb)}).

\bibitem{WIL-1996} Wilson D.B. (1996), ``Generating random spanning trees more quickly than the cover time'', Proceedings of the Twenty-eighth Annual ACM \emph{Symposium on the Theory of Computing} (Philadelphia, PA, 1996), 296-303, ACM, New York, 1996.

\bibitem{BAR-1991} Barthélemy J.-P. and Guenoche A. (1991), \emph{Trees and Proximities Representations}, Wiley, ISBN: 978-0471922636.

\bibitem{KRU-1956} Kruskal J.B. (1956), \emph{On the shortest spanning subtree of a graph and the traveling salesman problem}, Proceedings of the American Mathematical Society. 7: 48–50.

\end{thebibliography}
