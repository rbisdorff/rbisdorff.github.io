%%%%%%%%%%%%%%%%%%%%% chapter.tex %%%%%%%%%%%%%%%%%%%%%%%%%%%%%%%%%
%
% sample chapter
%
% Use this file as a template for your own input.
%
%%%%%%%%%%%%%%%%%%%%%%%% Springer-Verlag %%%%%%%%%%%%%%%%%%%%%%%%%%
%\motto{Use the template \emph{chapter.tex} to style the various elements of your chapter content.}
\chapter{Working with the \Digraph Python resources}
\label{intro} % chapter1
% use \chaptermark{}
% to alter or adjust the chapter heading in the running head

\abstract*{The basic idea of the \Digraph Python resources is to make easy python interactive sessions or write short Python scripts for computing all kind of results from a bipolar-valued digraph or graph. These include such features as maximal independent, maximal dominant or absorbent choices, rankings, outrankings, linear ordering, etc. Most of the available computing resources are meant to illustrate a Master Course on Algorithmic Decision Theory given at the University of Luxembourg in the context of its Master in Information and Computer Science (MICS). The Python development of these computing resources offers the advantage of an easy to write and maintain OOP source code as expected from a performing scripting language without loosing on efficiency in execution times compared to compiled languages such as C++ or Java.}

\abstract{The basic idea of the \Digraph Python resources is to make easy python interactive sessions or write short Python scripts for computing all kind of results from a bipolar-valued digraph or graph. These include such features as maximal independent, maximal dominant or absorbent choices, rankings, outrankings, linear ordering, etc. Most of the available computing resources are meant to illustrate a Master Course on Algorithmic Decision Theory given at the University of Luxembourg in the context of its Master in Information and Computer Science (MICS). The Python development of these computing resources offers the advantage of an easy to write and maintain OOP source code as expected from a performing scripting language without loosing on efficiency in execution times compared to compiled languages such as C++ or Java.}

\section{Installing the \Digraph resources}
\label{sec:1.1}

Using the \Digraph modules is easy. You only need to have installed on your system the Python programming language of version 3.+ (readily available under Linux and Mac OS). Notice that, from Version 3.3 on, the Python standard decimal module implements very efficiently its decimal.Decimal class in C. Now, Decimal objects are mainly used in the \Digraph characteristic r-valuation functions, which makes the recent python-3.7+ versions much faster (more than twice as fast) when extensive digraph operations are performed.

Several download options (easiest under Linux or Mac OS-X) are given. Either, by using a git client either, from \texttt{github}:
\begin{scriptsize}
\begin{verbatim}
...\$ git clone https://github.com/rbisdorff/Digraph3
\end{verbatim}
\end{scriptsize}
Or, from \texttt{sourceforge.net}:
\begin{scriptsize}
\begin{verbatim}
...\$ git clone https://git.code.sf.net/p/digraph3/code Digraph3}
\end{verbatim}
\end{scriptsize}
Or, with a browser access, download either, from the github link above or, from the sourceforge page the latest distribution zip archive and extract it. On Linux or Mac OS, ..\$ cd to the extracted \texttt{Digraph3} directory.
\begin{scriptsize}
\begin{verbatim}
../Digraph3$ make install
\end{verbatim}
\end{scriptsize}
installs (with \emph{sudo} !!) the \Digraph modules in the current running python environment. Python 3.8 (or later) environment is recommended (see the makefile for adapting to your python environment). Whereas:
\begin{scriptsize}
\begin{verbatim}
../Digraph3$ make installVenv
\end{verbatim}          
\end{scriptsize}
installs the \Digraph modules in an activated virtual python environment. If the \textbf{cython} (\href{https://cython.org}{https://cython.org}) C-compiled modules for Big Data applications are required, it is necessary to previously install the \emph{Cython} package in the running Python environment:
\begin{scriptsize}
\begin{verbatim}
...$ python3 -m pip install cython
\end{verbatim}
\end{scriptsize}
It is recommended to run a test suite:
\begin{scriptsize}
\begin{verbatim}
.../Digraph3$ make tests
\end{verbatim}
\end{scriptsize}
Test results are stored in the \texttt{Digraph3/test} directory. Notice that the python3 \texttt{pytest} package is required:
\begin{scriptsize}
\begin{verbatim}
...$ python3 -m pip install pytest
\end{verbatim}
\end{scriptsize}
A verbose (with stdout not captured) pytest suite may be run as follows:
\begin{scriptsize}
\begin{verbatim}
.../Digraph3$ make verboseTests
\end{verbatim}
\end{scriptsize}
When the GNU \texttt{parallel} shell tool (\href{https://www.gnu.org/software/parallel}{https://www.gnu.org/software/parallel}) is installed and multiple cores are detected, the tests may be executed in multiple processing mode:
\begin{scriptsize}
\begin{verbatim}
../Digraph3$ make pTests 
\end{verbatim}
\end{scriptsize}
Individual module pytest suites are also provided (see the makefile), like the one for the \texttt{outrankingDigraphs} module:
\begin{scriptsize}
\begin{verbatim}
../Digraph3$ make outrankingDigraphsTests
\end{verbatim}
\end{scriptsize}

\paragraph{Dependencies}
\begin{itemize}
\item To be fully functional, the \Digraph resources mainly need the \texttt{graphviz} tools (\href{https://graphviz.org}{https://graphviz.org}) and the R statistics resources (\href{https://www.r-project.org}{https://www.r-project.org}) to be installed.
\item When exploring digraph isomorphisms, the \href{https://www.cs.sunysb.edu/~algorith/implement/nauty/implement.shtml}{\texttt{nauty} isomorphism testing program} is required.
\item Two specific criteria and actions clustering methods of the \texttt{OutrankingDigraph} class furthermore require the \texttt{calmat} matrix computing resource to be installed (see the \texttt{calmat} directory in the \Digraph resources).
\end{itemize}

\section{Starting a Python3 session}
\label{sec:1.2}
You may start an interactive Python3 session in the \texttt{Digraph3} directory.
\begin{footnotesize}
\begin{verbatim}
$HOME/.../Digraph3$ python3
Python 3.9.5 (default, Jun  5 2021, 14:26:42)
[GCC 9.3.0] on linux
Type "help", "copyright", "credits" or
       "license" for more information.
>>>
\end{verbatim}
\end{footnotesize}
For exploring the classes and methods provided by the \Digraph modules (see the \href{https://digraph3.readthedocs.io/en/latest/}{\Digraph documentation}) enter the Python3 commands following the session prompts marked with '\>\>\>' or '...' . The lines without the prompt are output from the Python3 interpreter.

\begin{lstlisting}[caption={Generating a digraph instance},label=list:1.1]
>>> from randomDigraphs import RandomDigraph
>>> dg = RandomDigraph(order=5,arcProbability=0.5,seed=101)
>>> dg
 *------- Digraph instance description ------*
  Instance class   : RandomDigraph
  Instance name    : randomDigraph
  Digraph Order    : 5
  Digraph Size     : 12
  Valuation domain : [-1.00; 1.00]
  Determinateness  : 100.000
  Attributes       : ['actions','valuationdomain',
                      'relation','order','name',
                      'gamma','notGamma']
\end{lstlisting}
                   
In Listing \ref{list:1.1}  we import, for instance, from the \texttt{randomDigraphs} module the \texttt{RandomDigraph} class in order to generate a random digraph object $dg$ of order 5 - number of nodes called (decision) actions - and arc probability of $50\%$. The resulting digraph of order 5 and size -- number of arcs-- 12 is completely determined.

\section{Permanent storage of a digraph object}
\label{sec:1.3}                   
We may save the content of $dg$ in a file called \texttt{tutorialDigraph.py}.

\begin{lstlisting}
>>> dg.save('tutorialDigraph')
 *--- Saving digraph in file: <tutorialDigraph.py> ---*
\end{lstlisting}

with the following content:

\begin{lstlisting}[caption={A stored digraph instance},label=list:1.2]
from decimal import Decimal
from collections import OrderedDict

actions = OrderedDict([
    ('a1', {'shortName': 'a1',
          'name': 'random decision action'}),
    ('a2', {'shortName': 'a2',
          'name': 'random decision action'}),
    ('a3', {'shortName': 'a3',
          'name': 'random decision action'}),
    ('a4', {'shortName': 'a4',
          'name': 'random decision action'}),
    ('a5', {'shortName': 'a5',
          'name': 'random decision action'}),
 ])

 valuationdomain = {
     'min': Decimal('-1.0'),
     'med': Decimal('0.0'),
     'max': Decimal('1.0'),
     'hasIntegerValuation': True, # representation format
}

 relation = {
     'a1': {'a1':Decimal('-1.0'), 'a2':Decimal('-1.0'),
              'a3':Decimal('1.0'), 'a4':Decimal('-1.0'),
              'a5':Decimal('-1.0'),},
     'a2': {'a1':Decimal('1.0'), 'a2':Decimal('-1.0'),
              'a3':Decimal('-1.0'), 'a4':Decimal('1.0'),
              'a5':Decimal('1.0'),},
     'a3': {'a1':Decimal('1.0'), 'a2':Decimal('-1.0'),
              'a3':Decimal('-1.0'), 'a4':Decimal('1.0'),
              'a5':Decimal('-1.0'),},
     'a4': {'a1':Decimal('1.0'), 'a2':Decimal('1.0'),
              'a3':Decimal('1.0'), 'a4':Decimal('-1.0'),
              'a5':Decimal('-1.0'),},
     'a5': {'a1':Decimal('1.0'), 'a2':Decimal('1.0'),
              'a3':Decimal('1.0'), 'a4':Decimal('-1.0'),
              'a5':Decimal('-1.0'),},
 }
\end{lstlisting}

All digraphs instances are of \texttt{Digraph} type and contain at least the following attributes (see Listing \ref{list:1.1}  Lines 11-12):
\begin{enumerate}
\item A \texttt{name} attribute, holding usually the actual name of the stored instance that was used to create the instance;
\item A ordered dictionary of digraph nodes called \texttt{actions} (decision alternatives) with at least a \texttt{name} attribute;
\item An \texttt{order} attribute containing the number of graph nodes (length of the actions dictionary) automatically added by the object constructor;
\item  A logical characteristic \texttt{valuationdomain} dictionary with three decimal entries: the minimum ($-1.0$, means certainly false), the median ($0.0$, means missing information) and the maximum characteristic value ($+1.0$, means certainly true);
\item A double dictionary called \texttt{relation} and indexed by an oriented pair of actions (nodes) and carrying a decimal characteristic value in the range of the previous valuation domain;
\item Its associated \texttt{gamma }attribute, a dictionary containing the direct successors, respectively predecessors of each action, automatically added by the object constructor;
\item Its associated \texttt{notGamma} attribute, a dictionary containing the actions that are not direct successors respectively predecessors of each action, automatically added by the object constructor.
\end{enumerate}

\section{Inspecting a digraph object}
\label{sec:1.4}

Different \texttt{show} methods (see Listing \ref{list:1.3}  Lines 3, 18, 28,31 below) reveal us now that $dg$ is a crisp, irreflexive and connected digraph of order five.

\begin{lstlisting}[caption={Random crisp digraph object},label=list:1.3]
>>> dg.showShort()
 *----- show short -------------*
 Digraph          : tutorialDigraph
 Actions          : OrderedDict([
  ('a1', {'shortName': 'a1', 'name': 'random decision action'}),
  ('a2', {'shortName': 'a2', 'name': 'random decision action'}),
  ('a3', {'shortName': 'a3', 'name': 'random decision action'}),
  ('a4', {'shortName': 'a4', 'name': 'random decision action'}),
  ('a5', {'shortName': 'a5', 'name': 'random decision action'})
  ])
 Valuation domain : {
  'min': Decimal('-1.0'),
  'max': Decimal('1.0'),
  'med': Decimal('0.0'), 'hasIntegerValuation': True
  }

>>> dg.showRelationTable()
 * ---- Relation Table -----
   S   |  'a1'  'a2'  'a3'  'a4'  'a5'
 ------|-------------------------------
  'a1' |   -1    -1     1    -1    -1
  'a2' |    1    -1    -1     1     1
  'a3' |    1    -1    -1     1    -1
  'a4' |    1     1     1    -1    -1
  'a5' |    1     1     1    -1    -1
 Valuation domain: [-1;+1]

>>> dg.showComponents()
 *--- Connected Components ---*
 1: ['a1', 'a2', 'a3', 'a4', 'a5']

>>> dg.showNeighborhoods()
 Neighborhoods:
   Gamma     :
 'a1': in => {'a2', 'a4', 'a3', 'a5'}, out => {'a3'}
 'a2': in => {'a5', 'a4'}, out => {'a1', 'a4', 'a5'}
 'a3': in => {'a1', 'a4', 'a5'}, out => {'a1', 'a4'}
 'a4': in => {'a2', 'a3'}, out => {'a1', 'a3', 'a2'}
 'a5': in => {'a2'}, out => {'a1', 'a3', 'a2'}
   Not Gamma :
 'a1': in => set(), out => {'a2', 'a4', 'a5'}
 'a2': in => {'a1', 'a3'}, out => {'a3'}
 'a3': in => {'a2'}, out => {'a2', 'a5'}
 'a4': in => {'a1', 'a5'}, out => {'a5'}
 'a5': in => {'a1', 'a4', 'a3'}, out => {'a4'}
\end{lstlisting}

The \texttt{exportGraphViz()} method generates in the current working directory a \texttt{tutorialDigraph.dot} file and a \texttt{tutorialdigraph.png} picture of the tutorial digraph $dg$ (see Fig. \ref{fig:1.1}), if the \emph{graphviz} tools are installed on your system.
\begin{lstlisting}
>>> dg.exportGraphViz('tutorialDigraph')
 *---- exporting a dot file do GraphViz tools ---------*
 Exporting to tutorialDigraph.dot
 dot -Grankdir=BT -Tpng tutorialDigraph.dot -o tutorialDigraph.png
\end{lstlisting}
\begin{figure}[h]
\sidecaption
\includegraphics[width=7cm]{Figures/tutorialDigraph.png}
\caption{The tutorial crisp digraph. The \texttt{exportGraphViz()} method is depending on drawing tools from \texttt{https://graphviz.org}. On Linux Ubuntu or Debian you may try '\texttt{sudo apt-get install graphviz}’ to install them. There are ready $dmg$ installers for Mac OSX.}
\label{fig:1.1}       % Give a unique label
\end{figure}

Further methods are provided for inspecting this random \texttt{Digraph} object $dg$, like the following \texttt{showStatistics()} method.

\begin{lstlisting}[caption={Inspecting a \texttt{Digraph} object},label=list:1.5]
>>> dg.showStatistics()
 *----- general statistics -------------*
 for digraph              : <tutorialDigraph.py>
 order                    : 5 nodes
 size                     : 12 arcs
 undetermined           : 0 arcs
 determinateness (%)      : 100.0
 arc density              : 0.60
 double arc density       : 0.40
 single arc density       : 0.40
 absence density          : 0.20
 strict single arc density: 0.40
 strict absence density   : 0.20
 nbr. of components         : 1
 nbr. of strong components  : 1
 transitivity degree (%)  : 60.0
                          : [0, 1, 2, 3, 4, 5]
 outdegrees distribution  : [0, 1, 1, 3, 0, 0]
 indegrees distribution   : [0, 1, 2, 1, 1, 0]
 mean outdegree           : 2.40
 mean indegree            : 2.40
                          : [0, 1, 2, 3, 4, 5, 6, 7, 8, 9, 10]
 symmetric degrees dist.  : [0, 0, 0, 0, 1, 4, 0, 0, 0, 0, 0]
 mean symmetric degree    : 4.80
 outdegrees concentration index   : 0.1667
 indegrees concentration index    : 0.2333
 symdegrees concentration index   : 0.0333
                                  : [0, 1, 2, 3, 4, 'inf']
 neighbourhood depths distribution: [0, 1, 4, 0, 0, 0]
 mean neighbourhood depth         : 1.80
 digraph diameter                 : 2
 agglomeration distribution       :
 a1 : 58.33
 a2 : 33.33
 a3 : 33.33
 a4 : 50.00
 a5 : 50.00
 agglomeration coefficient        : 45.00
\end{lstlisting}

These show methods usually rely upon corresponding compute methods, like the computeSize(), the computeDeterminateness() or the computeTransitivityDegree() method (see Listing \ref{list:1.6} Lines 5,7,16).

\begin{lstlisting}[label=list:1.6,basicstyle=\footnotesize]
>>> dg.computeSize()
 12
>>> dg.computeDeterminateness(InPercents=True)
 Decimal('100.00')
>>> dg.computeTransitivityDegree(InPercents=True)
 Decimal('60.00')
\end{lstlisting}
Mind that \texttt{show} methods output their results in the \emph{Python} console. We provide also some \texttt{showHTML} methods which output their results in a system browser’s window.
\begin{lstlisting}[label=list:1.7,basicstyle=\footnotesize]
>>> dg.showHTMLRelationMap(relationName='r(x,y)',rankingRule=None)
\end{lstlisting}
\begin{figure}[h]
\sidecaption
\includegraphics[width=7cm]{Figures/relationMap1.png}
\caption{Browsing the relation map of the tutorial digraph. $+$ indicates a certainly valid and $-$ indicates a certainly  invalid relation, Here we find confirmed again that our random digraph instance $dg$, is indeed a crisp, i.e. 100\% determined irreflexive digraph instance.}
\label{fig:1.2}       % Give a unique label
\end{figure}
\clearpage

\section{Special digraph instances}
\label{sec:1.5}

Some constructors for universal digraph instances, like the \texttt{CompleteDigraph}, the \texttt{EmptyDigraph} or the \texttt{GridDigraph} constructor, are readily available (see Fig. \ref{fig:1.3}).
\begin{lstlisting}
>>> from digraphs import GridDigraph
>>> grid = GridDigraph(n=5,m=5,hasMedianSplitOrientation=True)
>>> grid.exportGraphViz('tutorialGrid')
 *---- exporting a dot file for GraphViz tools ---------*
 Exporting to tutorialGrid.dot
 dot -Grankdir=BT -Tpng TutorialGrid.dot -o tutorialGrid.png
\end{lstlisting}
\begin{figure}[h]
\sidecaption
\includegraphics[width=5cm]{Figures/tutorialGrid.png}
\caption{The 5x5 grid graph. Notice the median split orientation. This kind of oriented grids show the highest possible number of chordless circuits. }
\label{fig:1.3}       % Give a unique label
\end{figure}

To be written: chaper conclusion and transition to the next chapter.
