\chapter{Rating by sorting into relative performance quantiles}
\label{sec:9}

\abstract*{ In this Chapter, we apply order statistics for sorting a set $X$ of $n$ potential decision alternatives, evaluated on $m$ incommensurable performance criteria, into $q$ quantile equivalence classes. The sorting algorithm is based on pairwise outranking characteristics involving the quantile class limits observed on each criterion. Thus we may implement a weak ordering algorithm of complexity $O(nmq)$.
}

\abstract{ In this Chapter, we apply order statistics for sorting a set $X$ of $n$ potential decision alternatives, evaluated on $m$ incommensurable performance criteria, into $q$ quantile equivalence classes. The sorting algorithm is based on pairwise outranking characteristics involving the quantile class limits observed on each criterion. Thus we may implement a weak ordering algorithm of complexity $O(nmq)$.
}

\section{Quantile sorting on a single performance criterion}
\label{sec:9.1}

A single criterion sorting category $K$ is a (usually) lower-closed interval $[m_k ; M_k[$ on a real-valued performance measurement scale, with $m_k \leq M_k$. If $x$ is a measured performance on this scale, we may distinguish three sorting situations.
\begin{enumerate}[leftmargin=1cm,rightmargin=0.5cm,topsep=1pt]
\item $x < m_k$ and $x < M_k$: The performance $x$ is lower than category $K$.
\item $x \geqslant m_k$ and $x < M_k$: The performance $x$ belongs to category $K$.
\item $x > m_k$ and $x \geqslant M_k$: The performance $x$ is higher than category $K$.
\end{enumerate}

As the relation $<$ is the dual of $\geqslant$ ($\not\geqslant$), it will be sufficient to check that $x \geqslant m_k$ as well as $x \not\geqslant M_k$ are true for $x$ to be considered a member of category $K$.

Upper-closed categories (in a more mathematical integration style) may as well be considered. In this case it is sufficient to check that $m_k \not\geqslant x$ as well as $M_k \geq x$ are true for $x$ to be considered a member of category $K$. It is worthwhile noticing that a category $K$ such that $m_k = M_k$ is hence always empty by definition. In order to be able to properly sort over the complete range of values to be sorted, we will need to use a special, two-sided closed last, respectively first, category.

Let $K = {K_1 , ..., K_q}$ be a non trivial partition of the criterion’s performance measurement scale into $q \geq 2$ ordered categories $K_k$ – i.e. lower-closed intervals $[m_k ; M_k[$ – such that $m_k < M_k$, $M_k = m_{k+1}$ for $k = 0, ..., q - 1$ and $M_q = \infty$. And, let $A=\{a_1 , a_2 , a_3 , ...\}$ be a finite set of not all equal performance measures observed on the scale in question.

\textbf{Property}: For all performance measure $x \in A$ there exists now a unique $k$ such that $x \in K_k$. If we assimilate, like in descriptive statistics, all the measures gathered in a category $K_k$ to the central value of the category – i.e. $(m_k + M_k)/2$ – the sorting result will hence define a weak order (complete preorder) on $A$.

Let $Q=\{Q_0 , Q_1 , ..., Q_q\}$ denote the set of $q + 1$ increasing order-statistical quantiles –like quartiles or deciles– we may compute from the ordered set $A$ of performance measures observed on a performance scale. If $Q_0 = \min(X)$, we may, with the following intervals: $[Q_0 ; Q_1 [$, $[Q_1 ; Q_2 [$, ..., $[Q_{q-1}; \infty[$, hence define a set of $q$ lower-closed sorting categories. And, in the case of upper-closed categories, if $Q_q = \max(X)$, we would obtain the intervals $] -\infty; Q_1]$, $]Q_1 ; Q_2]$, ..., $]Q_{q-1} ; Q_q]$. The corresponding sorting of $A$ will result, in both cases, in a repartition of all measures $x$ into the $q$ quantile categories $K_k$ for $k = 1, ..., q$.

\textbf{Example}: Let $A$ = \{ $a_7 = 7.03$, $a_{15}=9.45$, $a_{11}= 20.35$, $a_{16}= 25.94$, $a_{10}= 31.44$, $a_9= 34.48$, $a_{12}= 34.50$, $a_{13}= 35.61$, $a_{14}= 36.54$, $a_{19}= 42.83$, $a_5= 50.04$, $a_2= 59.85$, $a_{17}= 61.35$, $a_{18}= 61.61$, $a_3= 76.91$, $a_6= 91.39$, $a_1= 91.79$, $a_4= 96.52$, $a_8= 96.56$, $a_{20}= 98.42$ \} be a set of 20 increasing performance measures observed on a given criterion. The lower-closed category limits we obtain with quartiles ($q = 4$) are: $Q_0 = 7.03$ = $a_7$, $Q_1= 34.485$, $Q_2= 54.945$ (median performance), and $Q_3= 91.69$. And the sorting into these four categories defines on $A$ a complete preorder with the following four equivalence classes: $K_1=\{a_7,a_{10},a_{11},a_{10},a_{15},a_{16}\}$, $K_2=\{a_5,a_9,a_{13},a_{14},a_{19}\}$, $K_3=\{a_2,a_3,a_6,a_{17},a_{18}\}$, and $K_4=\{a_1,a_4,a_8,a_{20}\}$.

\section{Quantiles sorting with multiple performance criteria}
\label{sec:9.2}

Let us now suppose that we are given a performance tableau with a set $X$ of $n$ decision alternatives evaluated on a coherent family of $m$ performance criteria associated with the corresponding outranking relation $\succsim$ defined on $X$. We denote $x_j$ the performance of alternative $x$ observed on criterion $j$.

Suppose furthermore that we want to sort the decision alternatives into $q$ upper-closed quantile equivalence classes. We therefore consider a series : $k = k/q$ for $k = 0, ..., q$ of $q+1$ equally spaced quantiles, like quartiles: $0$, $0.25$, $0.5$, $0.75$, $1$; quintiles: $0$, $0.2$, $0.4$, $0.6$, $0.8$, $1$: or deciles: $0$, $0.1$, $0.2$, ..., $0.9$, $1$, for instance.

The upper-closed $\mathbf{q}^k$ class corresponds to the m quantile intervals $]q_j(p_{k-1});q_j(p_k)]$ observed on each criterion $j$,  where $k = 2, ..., q$ , $q_j(p_q) =  \max_X(x_j)$, and the first class gathers all performances below or equal to $Q_j(p_1)$.

The lower-closed $\mathbf{q}_k$ class corresponds to the $m$ quantile intervals $[q_j(p_{k-1});q_j(p_k)[$ observed on each criterion $j$, where $k = 1, ..., q-1$, $q_j(p_0) = \min_X(x_j)$, and the last class gathers all performances above or equal to $Q_j(p_{q-1})$.

We call \textbf{q-tiles} a complete series of $k = 1, ..., q$ upper-closed $\mathbf{q}^k$, respectively lower-closed $\mathbf{q}_k$, multiple criteria quantile classes.

\textbf{Property}: With the help of the bipolar-valued characteristic of the outranking relation $r(\succsim)$ we may compute the bipolar-valued characteristic of the assertion: $x$ belongs to upper-closed $q$-tiles class $\mathbf{q}^k$ class, resp. lower-closed class $\mathbf{q}_k$, as follows \citep{BIS-2016} :
\begin{eqnarray}\label{eq:9.1}
r(x \in \mathbf{q}^k) \; = \; \min \big[ -r\big(\mathbf{q}(p_{q-1}) \succsim x\big), \,r\big(\mathbf{q}(p_{q}) \succsim x)\big]\\
r(x \in \mathbf{q}_k) \; = \; \min \big[ r\big(x \succsim \mathbf{q}(p_{q-1})\big),\, -r\big(x \succsim\mathbf{q}(p_{q})\big)\big]
\end{eqnarray}

The outranking relation $\succsim$ verifying the coduality principle, $-r\big(\mathbf{q}(p_{q-1}) \succsim x\big) \,=\, r\big(\mathbf{q}(p_{q-1}) \prec x\big)$, resp. $-r\big(x \succsim \mathbf{q}(p_{q}) \,=\, r\big(x \prec \mathbf{q}(p_{q}\big)$.

We may compute, for instance, a five-tiling of a given random performance tableau with the help of the \texttt{QuantilesSortingDigraph} class \index{QuantilesSortingDigraph@\texttt{QuantilesSortingDigraph} class}.
\begin{lstlisting}[caption={Computing a quintiles sorting result},label=list:9.1]
>>> from randomPerfTabs import RandomPerformanceTableau
>>> t = RandomPerformanceTableau(numberOfActions=50,seed=5)
>>> from sortingDigraphs import QuantilesSortingDigraph
>>> qs = QuantilesSortingDigraph(t,limitingQuantiles=5)
>>> qs
    *-----  Object instance description -----------*
     Instance class  : QuantilesSortingDigraph
     Instance name   : sorting_with_5-tile_limits
     # Actions       : 50
     # Criteria      : 7
     # Categories    : 5
     Lowerclosed     : False
     Size            : 841
     Valuation domain  : [-1.00;1.00]
     Determinateness (%) : 81.39
     Attributes          : ['actions','actionsOrig',
          'criteria','evaluation','runTimes','name',
	  'limitingQuantiles','LowerClosed',
	  'categories','criteriaCategoryLimits',
	  'profiles','profileLimits','hasNoVeto',
	  'valuationdomain','nbrThreads','relation',
	  'categoryContent','order','gamma','notGamma']
     *------  Constructor run times (in sec.) ------*
     # Threads        : 1
     Total time       : 0.03120
     Data input       : 0.00300
     Compute profiles : 0.00075
     Compute relation : 0.02581
     Weak Ordering    : 0.00052
>>> qs.showCriteriaQuantileLimits()
     Quantile Class Limits (q = 5)
     Upper-closed classes
     crit.	 0.20	 0.40	 0.60	 0.80	 1.00	 
     *------------------------------------------------
      g1	 31.35	 41.09	 58.53	 71.91	 98.08	 
      g2	 27.81	 39.19	 49.87	 61.66	 96.18	 
      g3	 25.10	 34.78	 49.45	 63.97	 92.59	 
      g4	 24.61	 37.91	 53.91	 71.02	 89.84	 
      g5	 26.94	 36.43	 52.16	 72.52	 96.25	 
      g6	 23.94	 44.06	 54.92	 67.34	 95.97	 
      g7	 30.94	 47.40	 55.46	 69.04	 97.10	 
>>> qs.showSorting()
    *--- Sorting results in descending order ---*
     ]0.80 - 1.00]: ['a22']
     ]0.60 - 0.80]: ['a03','a07','a08','a11','a14','a17',
                     'a19','a20','a29','a32','a33','a37',
		     'a39','a41','a42','a49']
     ]0.40 - 0.60]: ['a01','a02','a04','a05','a06','a08',
                     'a09','a16','a17','a18','a19','a21',
		     'a24','a27','a28','a30','a31','a35',
		     'a36','a40','a43','a46','a47','a48',
		     'a49','a50']
     ]0.20 - 0.40]: ['a04','a10','a12','a13','a15','a23',
                     'a25','a26','a34','a38','a43','a44',
		     'a45','a49']
     ]   < - 0.20]: ['a44']
   \end{lstlisting}
With the \texttt{showCriteriaQuantileLimits()}\index{showCriteriaQuantileLimits@\texttt{showCriteriaQuantileLimits()}} and the \texttt{showSorting()}\index{showSorting@\texttt{showSorting()}} methods, we may inspect the quintiles sorting result. Most of the decision alternatives (26) are gathered in the median quintile $]0.40 - 0.60]$ class, whereas the highest quintile $]0.80-1.00]$ and the lowest quintile $]< - 0.20]$ classes gather each one a unique decision alternative (\texttt{a22}, resp. \texttt{a44}) (see Listing~\vref{list:9.1} Lines 43-).

We may inspect the details of the corresponding sorting characteristics with the \texttt{showSortingCharacteristics()} method\index{showSortingCharacteristics@\texttt{showSortingCharacteristics()}}.
\begin{lstlisting}[caption={Bipolar-valued sorting characteristics (extract)},label=list:9.2]
>>> qs.valuationdomain
    {'min': Decimal('-1.0'), 'med': Decimal('0'),
     'max': Decimal('1.0')}
>>> qs.showSortingCharacteristics()
     x  in  q^k          r(q^k-1 < x)  r(q^k >= x)  r(x in q^k)
    a22 in ]< - 0.20]	    1.00	 -0.86	      -0.86
    a22 in ]0.20 - 0.40]    0.86	 -0.71	      -0.71
    a22 in ]0.40 - 0.60]    0.71	 -0.71	      -0.71
    a22 in ]0.60 - 0.80]    0.71	 -0.14	      -0.14
    a22 in ]0.80 - 1.00]    0.14	  1.00	       0.14
    ...
    ...
    a44 in ]< - 0.20]	    1.00	  0.00	       0.00
    a44 in ]0.20 - 0.40]    0.00	  0.57	       0.00
    a44 in ]0.40 - 0.60]   -0.57	  0.86	      -0.57
    a44 in ]0.60 - 0.80]   -0.86	  0.86	      -0.86
    a44 in ]0.80 - 1.00]   -0.86	  0.86	      -0.86
    ...
    ...
    a49 in ]< - 0.20]	    1.00	 -0.43	      -0.43
    a49 in ]0.20 - 0.40]    0.43	  0.00	       0.00
    a49 in ]0.40 - 0.60]    0.00	  0.00	       0.00
    a49 in ]0.60 - 0.80]    0.00	  0.57	       0.00
    a49 in ]0.80 - 1.00]   -0.57	  0.86	      -0.57
\end{lstlisting}
Alternative \texttt{a22} verifies indeed positively both sorting conditions only for the highest quintile $[0.80 - 1.00]$ class (see Listing~\vref{list:9.2} Lines 10). Whereas alternatives \texttt{a44} and \texttt{a49}, for instance, weakly verify both sorting conditions each one for two, resp. three, adjacent quintile classes (see Lines 13-14 and 21-23).  

Quantiles sorting results indeed always verify the following \textbf{Properties}:
\begin{enumerate}[leftmargin=0.5cm,rightmargin=0.5cm,topsep=1pt]
\item \textbf{Coherence}: Each object is sorted into a non-empty subset of \emph{adjacent} $q$-tiles classes. An alternative that would \emph{miss} evaluations on all the criteria will be sorted conjointly in all $q$-tiled classes.
\item \textbf{Uniqueness}: If $r(x \in \mathbf{q}^k) \neq 0$  for $k = 1, ..., q$, then performance $x$ is sorted into \emph{exactly one single} $q$-tiled class.
\item \textbf{Separability}: Computing the sorting result for performance $x$ is independent from the computing of the other performances’ sorting results. This property gives access to efficient parallel processing of class membership characteristics.
\end{enumerate}

The $q$-tiles sorting result leaves us hence with more or less \emph{overlapping} ordered quantile equivalence classes. For constructing now a linearly ranked q-tiles partition of $X$ , we may apply three strategies:
\begin{enumerate}[leftmargin=0.5cm,rightmargin=0.5cm,topsep=1pt]
\item \textbf{Average} (default): In decreasing lexicographic order of the average of the lower and upper quantile limits and the upper quantile class limit;
\item \textbf{Optimistic}: In decreasing lexicographic order of the upper and lower quantile class limits;
\item \textbf{Pessimistic}: In decreasing lexicographic order of the lower and upper quantile class limits;
\end{enumerate}

\begin{lstlisting}[caption={Weakly ranking the quintiles sorting result},label=list:9.3]
>>> qs.showQuantileOrdering(strategy='average')
    ]0.80-1.00] : ['a22']
    ]0.60-0.80] : ['a03','a07','a11','a14','a20','a29',
                   'a32','a33','a37','a39','a41','a42']
    ]0.40-0.80] : ['a08','a17','a19']
    ]0.20-0.80] : ['a49']
    ]0.40-0.60] : ['a01','a02','a05','a06','a09','a16',
                    'a18','a21','a24','a27','a28','a30',
		    'a31','a35','a36','a40','a46','a47',
		    'a48','a50']
    ]0.20-0.60] : ['a04','a43']
    ]0.20-0.40] : ['a10','a12','a13','a15','a23','a25',
                    'a26','a34','a38','a45']
    ]  < -0.40] : ['a44']
\end{lstlisting}
Following, for instance, the \emph{average} ranking strategy, we find confirmed in the weak ranking shown in Listing~\vref{list:9.3},  that alternative \texttt{a49}  is indeed sorted into three adjacent quintiles classes, namely $]0.20-0.80]$ (see Line 6) and precedes the $]0.40-0.60]$ class, of same average of lower and upper limits.

The \texttt{QuantilesSortingDigraph} constructor gives hence a linearly ordered decomposition of the corresponding bipolar-valued outranking digraph. This decomposition leads us to a new \textbf{sparse pre-ranked} outranking digraph model.

\section{The sparse pre-ranked outranking digraph model}
\label{sec:9.3}

We may notice that a given outranking digraph -the association of a set of decision alternatives and an outranking relation- is, following the methodological requirements of the outranking approach, necessarily associated with a corresponding performance tableau. And, we may use this underlying performance tableau for linearly decomposing the set of potential decision alternatives into ordered quantiles equivalence classes by using the quantiles sorting technique seen in the previous Section. 

In the coding example shown in Listing~\vref{list:9.4} below, we generate for instance, first (Lines 2-3), a simple performance tableau of 75 decision alternatives and, secondly (Lines 4-6), we construct the corresponding \texttt{PreRankedOutrankingDigraph} instance called $prg$. Notice by the way the \texttt{BigData} flag (Line 3) used here for generating a parsimoniously commented performance tableau.

\begin{lstlisting}[caption={Computing a \emph{pre-ranked} sparse outranking digraph},label=list:9.4]
>>> from randomPerfTabs import RandomPerformanceTableau
>>> tp = RandomPerformanceTableau(numberOfActions=75,\
...                               BigData=True,seed=100)
>>> from sparseOutrankingDigraphs import\
...          PreRankedOutrankingDigraph
>>> prg = PreRankedOutrankingDigraph(Tp,quantiles=5)
>>> prg
    *----- Object instance description ------*
     Instance class    : PreRankedOutrankingDigraph
     Instance name     : randomperftab_pr
     Actions           : 75
     Criteria          : 7
     Sorting by        : 5-Tiling
     Ordering strategy : average
     Components        : 9
     Minimal order     : 1
     Maximal order     : 25
     Average order     : 8.3
     fill rate         : 20.432%
     Attributes        : ['actions','criteria','evaluation','NA','name',
         'order','runTimes','dimension','sortingParameters',
	 'valuationdomain','profiles','categories','sorting',
	 'decomposition','nbrComponents','components',
	 'fillRate','minimalComponentSize','maximalComponentSize', ... ]
\end{lstlisting}

The ordering of the 5-tiling result is following the \emph{average} lower and upper quintile limits strategy (see Listing~\vref{list:9.4} Line 14). We obtain here 9 ordered components of minimal order 1 and maximal order 25. The corresponding \emph{pre-ranked decomposition} may be visualized as follows.

\begin{lstlisting}[caption={The quantiles decomposition of a pre-ranked outranking digraph},label=list:9.5]
>>> prg.showDecomposition()
    *--- quantiles decomposition in decreasing order---*
     c1. ]0.80-1.00] : [5, 42, 43, 47]
     c2. ]0.60-1.00] : [73]
     c3. ]0.60-0.80] : [1, 4, 13, 14, 22, 32, 34, 35, 40,
                        41, 45, 61, 62, 65, 68, 70, 75]
     c4. ]0.40-0.80] : [2, 54]
     c5. ]0.40-0.60] : [3, 6, 7, 10, 15, 18, 19, 21, 23, 24,
                        27, 30, 36, 37, 48, 51, 52, 56, 58,
			63, 67, 69, 71, 72, 74]
     c6. ]0.20-0.60] : [8, 11, 25, 28, 64, 66]
     c7. ]0.20-0.40] : [12, 16, 17, 20, 26, 31, 33, 38, 39,
                        44, 46, 49, 50, 53, 55]
     c8. ]   <-0.40] : [9, 29, 60]
     c9. ]   <-0.20] : [57, 59]
\end{lstlisting}

The highest quintile class ($]80\%-100\%]$) contains decision alternatives 5, 42, 43 and 47. Lowest quintile class ($]-20\%]$) gathers alternatives 57 and 59 (see Listing~\vref{list:9.5} Lines 3 and 15). We may inspect the resulting sparse outranking relation map as follows in a browser view.
\begin{lstlisting}
   >>> prg.showHTMLRelationMap()
\end{lstlisting}
\begin{figure}[h]
%\sidecaption
\includegraphics[width=11cm]{Figures/9-1-sparse75RelationMap.png}
\caption{The relation map of a sparse outranking digraph.}
\label{fig:9.1}       % Give a unique label
\end{figure}
%\clearpage
In Fig.~\vref{fig:9.1} we easily recognize the 9 linearly ordered quantile equivalence classes. \emph{Green} and \emph{light-green} show positive \textbf{outranking} situations, whereas positive \textbf{outranked} situations are shown in \emph{red} and \emph{light-red}. Indeterminate situations appear in white. In each one of the 9 quantile equivalence classes we recover in fact the corresponding bipolar-valued outranking \emph{sub-relation}, which leads to an actual \textbf{fill-rate} of $20.4\%$ (see Listing~\vref{list:9.4} Line 19).

We may now check how faithful the sparse model represents the complete outranking relation.

\begin{lstlisting}
   >>> g = BipolarOutrankingDigraph(tp)
   >>> corr = prg.computeOrdinalCorrelation(g)
   >>> g.showCorrelation(corr)
    Correlation indexes:
     Crisp ordinal correlation  : +0.863
     Epistemic determination    :  0.315
     Bipolar-valued equivalence : +0.272
\end{lstlisting}   

The ordinal correlation index between the standard and the sparse outranking relations is quite high ($+0.863$) and their bipolar-valued equivalence is supported by a mean criteria significance majority of $(1.0+0.272)/2 = 64\%$.

It is worthwhile noticing in Listing~\vref{list:9.4} (Lines 20- ) that sparse pre-ranked outranking digraphs do not contain a \texttt{relation} attribute. The access to pairwise outranking characteristic values is here provided via a corresponding \texttt{relation()} function.
\begin{lstlisting}[caption={Access to the binary relation characteristics},label=list:9.6]
def relation(self,x,y):
    """
    Dynamic construction of the global
    outranking characteristic function r(x,y).
    """
    Min = self.valuationdomain['min']
    Med = self.valuationdomain['med']
    Max = self.valuationdomain['max']
    if x == y:
        return Med
    cx = self.actions[x]['component']
    cy = self.actions[y]['component']
    if cx == cy:
        return self.components[cx]['subGraph'].relation[x][y]
    elif self.components[cx]['rank'] > self.components[cy]['rank']:
        return Min
    else:
        return Max
\end{lstlisting}
In Listing~\vref{list:9.6} Lines 9-10, all reflexive situations are set to the \emph{indeterminate} value. When two decision alternatives belong to a same component --quantile equivalence class-- we access the relation attribute of the corresponding outranking sub-digraph (Lines 13-14). Otherwise we just check the respective ranks of the components (Lines 15-18).

\section{Ranking pre-ranked sparse outranking digraphs}
\label{sec:9.4}

Each one of these nine ordered components may now be locally ranked by using a suitable ranking rule. Best operational results, both in run times and quality, are more or less equally obtained with the \Copeland and the \NetFlows rules\footnote{See Sections~\vref{sec:8.2} and \vref{sec:8.3}}. The eventually obtained linear ordering (from the worst to best) is stored in a \texttt{prg.boostedOrder} attribute. A reversed linear order (from the best to the worst) is stored in a \texttt{prg.boostedRanking} attribute.
  \begin{lstlisting}
>>> prg.boostedRanking
  [43, 47, 42, 5, 73, 65, 68, 32, 62, 70, 35, 22, 75, 45, 1,
   61, 41, 34, 4, 13, 40, 14, 2, 54, 63, 37, 56, 71, 69, 36,
   19, 72, 15, 48, 6, 30, 74, 3, 21, 58, 52, 18, 7, 24, 27,
   23, 67, 51, 10, 25, 11, 8, 64, 28, 66, 53, 12, 31, 39, 55,
   20, 46, 49, 16, 44, 26, 38, 33, 17, 50, 29, 60, 9, 59, 57]
\end{lstlisting}
Alternative \texttt{43} appears \emph{first ranked}, whereas alternative \texttt{57} is \emph{last ranked} (see Line 2 and 6 above).

The quality of this ranking result may be assessed by computing its ordinal correlation with the standard outranking relation.  
\begin{lstlisting}
>>> corr = g.computeRankingCorrelation(prg.boostedRanking)
>>> g.showCorrelation(corr)
 Correlation indexes:
   Crisp ordinal correlation  : +0.807
   Epistemic determination    :  0.315
   Bipolar-valued equivalence : +0.254
\end{lstlisting}

We may also verify below that the \Copeland ranking obtained from the standard outranking digraph is highly correlated ($+0.822$) with the one obtained from the sparse outranking digraph.
\begin{lstlisting}
   >>> from linearOrders import CopelandOrder
   >>> cop = CopelandOrder(g)
   >>> print(cop.computeRankingCorrelation(prg.boostedRanking))
    {'correlation': 0.822, 'determination': 1.0}
\end{lstlisting}

\vspace{1cm}
Noticing the computational efficiency of the quantiles sorting construction, coupled with the separability property of the quantile class membership characteristics computation, we will make usage of the \texttt{PreRankedOutrankingDigraph} constructor in Chapter~\vref{sec:11} for HPC ranking big and even huge performance tableaux. Before we shall address in the next Chapter~\vref{sec:10} the problem of absolute rating into learned quantile performance norms.

%%%%%%% The chapter bibliography
%\normallatexbib
\clearpage
%\phantomsection
%\addcontentsline{toc}{section}{Chapter Bibliography}
\bibliographystyle{spbasic}
%\typeout{}
\bibliography{03-backMatters/reference}
%\chapter{Ranking with multiple incommensurable criteria}
\label{sec:8}

\abstract*{ The \Digraph python resources provide several algorithms for solving the ranking problem with a bipolar-valued outranking digraph. The \Copeland, \NetFlows, \Kemeny, \Slater, \Kohler, and the \RankedPairs ranking rules are presented and illustrated with a random outranking digraph.}

\abstract{The \Digraph python resources provide several algorithms for solving the ranking problem with a bipolar-valued outranking digraph. The \Copeland, \NetFlows, \Kemeny, \Slater, \Kohler, and the \RankedPairs ranking rules are presented and illustrated with a random outranking digraph.}

\section{The ranking problem}
\label{sec:8.1}

We need to rank without ties a set $X$ of items (usually decision alternatives) that are evaluated on multiple incommensurable performance criteria; yet, for which we may know their pairwise bipolar-valued {\em strict outranking\/} characteristics, i.e. $r(x\, \succnsim \, y)$ for all $x, y \in X$ (see Section \ref{sec:3.5} and \citep{BIS-2013}).

Let us consider a didactic outranking digraph \texttt{g} generated from a random \emph{Cost-Benefit} performance tableau (see Section \ref{sec:6.3}) concerning 9 decision alternatives evaluated on 13 performance criteria. We may compute the corresponding {\em strict outranking digraph\/} with a codual transform (see Section \ref{sec:2.6}).

\begin{lstlisting}[caption={Random bipolar-valued strict outranking relation characteristics},label=list:8.1]
>>> from randomPerfTabs import RandomCBPerformanceTableau   
>>> t = RandomCBPerformanceTableau(numberOfActions=9,\
...         numberOfCriteria=13,seed=200)
>>> from outrankingDigraphs import BipolarOutrankingDigraph
>>> g = BipolarOutrankingDigraph(t,Normalized=True)
>>> gcd = ~(-g) # codual digraph
>>> gcd.showRelationTable(ReflexiveTerms=False)
 * ---- Relation Table -----
  r(>) |  'a1'  'a2'  'a3'  'a4'  'a5'  'a6'  'a7'  'a8'  'a9'   
  -----|------------------------------------------------------
  'a1' |    -   0.00 +0.10 -1.00 -0.13 -0.57 -0.23 +0.10 +0.00  
  'a2' | -1.00   -    0.00 +0.00 -0.37 -0.42 -0.28 -0.32 -0.12  
  'a3' | -0.10  0.00   -   -0.17 -0.35 -0.30 -0.17 -0.17 +0.00  
  'a4' |  0.00  0.00 -0.42   -   -0.40 -0.20 -0.60 -0.27 -0.30  
  'a5' | +0.13 +0.22 +0.10 +0.40   -   +0.03 +0.40 -0.03 -0.07  
  'a6' | -0.07 -0.22 +0.20 +0.20 -0.37   -   +0.10 -0.03 -0.07  
  'a7' | -0.20 +0.28 -0.03 -0.07 -0.40 -0.10   -   +0.27 +1.00  
  'a8' | -0.10 -0.02 -0.23 -0.13 -0.37 +0.03 -0.27   -   +0.03  
  'a9' |  0.00 +0.12 -1.00 -0.13 -0.03 -0.03 -1.00 -0.03   -   
\end{lstlisting}
  
Some ranking rules will work on the associated \Condorcet digraph\index{digraph!Condorcet}, i.e. the corresponding \emph{strict median cut} polarised strict outranking digraph.
 \begin{lstlisting}[caption={Median cut polarised strict outranking relation characteristics},label=list:8.2]
>>> ccd = PolarisedOutrankingDigraph(gcd,\
...                   level=g.valuationdomain['med'],\
...                   KeepValues=False,StrictCut=True)
>>> ccd.showRelationTable(ReflexiveTerms=False,\
...                       IntegerValues=True)
 *---- Relation Table -----
  r(>)_med | 'a1' 'a2' 'a3' 'a4' 'a5' 'a6' 'a7' 'a8' 'a9'   
  ---------|---------------------------------------------
     'a1'  |   -    0   +1   -1   -1   -1   -1   +1    0  
     'a2'  |  -1    -   +0    0   -1   -1   -1   -1   -1  
     'a3'  |  -1    0    -   -1   -1   -1   -1   -1    0  
     'a4'  |   0    0   -1    -   -1   -1   -1   -1   -1  
     'a5'  |  +1   +1   +1   +1    -   +1   +1   -1   -1  
     'a6'  |  -1   -1   +1   +1   -1    -   +1   -1   -1  
     'a7'  |  -1   +1   -1   -1   -1   -1    -   +1   +1  
     'a8'  |  -1   -1   -1   -1   -1   +1   -1    -   +1  
     'a9'  |   0   +1   -1   -1   -1   -1   -1   -1    -   
\end{lstlisting}

Unfortunately, such crisp median-cut \Condorcet digraphs, associated with a given strict outranking digraph, only exceptionally present a linear ordering. Usually, pairwise majority comparisons do not even render a \emph{complete} or, at least, a \emph{transitive} partial order. There may even frequently appear \emph{cyclic} outranking situations (see Section \ref{sec:7.4}).

To discover how \emph{difficult} this ranking problem can get here, we have a look in Fig.~\ref{fig:8.1} at the corresponding strict outranking digraph \emph{graphviz} drawing \footnote{ The \texttt{exportGraphViz()} method is depending on drawing tools from graphviz software (https://graphviz.org/).}.
\begin{lstlisting}
>>> gcd.exportGraphViz(fileName='rankingTutorial')
 *---- exporting a dot file for GraphViz tools ---------*
  Exporting to rankingTutorial.dot
  dot -Grankdir=BT -Tpng rankingTutorial.dot\
                   -o rankingTutorial.png
\end{lstlisting}
\begin{figure}[h]
\sidecaption[t]
\includegraphics[width=5.5cm]{Figures/rankingTutorial.pdf}
\caption{The strict outranking relation $\succnsim$ shown here is, for instance, \emph{not transitive}: alternative \texttt{a8} outranks alternative \texttt{a6} and alternative \texttt{a6} outranks \texttt{a4}, however, \texttt{a8} does not outrank \texttt{a4}. Furthermore, alternatives \texttt{a8}, \texttt{a6} and \texttt{a7} show a cyclic outranking relation. }
\label{fig:8.1}       % Give a unique label
\end{figure}

We may compute the \emph{transitivity degree} of the outranking digraph shown in Fig.~\ref{fig:8.1}, i.e. the ratio of the difference between the number of outranking arcs and the number of transitive arcs over the difference of the number of arcs of the transitive closure minus the transitive arcs of the digraph \texttt{gcd}.
\begin{lstlisting}
>>> gcd.computeTransitivityDegree(Comments=True)
 Transitivity degree of graph <codual_rel_randomCBperftab>
  triples x>y>z: 78, closed: 38, open: 40
  closed/triples = 0.487
\end{lstlisting}    
With only $49\%$ of the required transitive arcs, the strict outranking relation here is hence very far from being transitive; a serious problem when a linear ordering of the decision alternatives is looked for.

Let us furthermore see if there are any cyclic outrankings.
\begin{lstlisting}
>>> gcd.computeChordlessCircuits()
>>> gcd.showChordlessCircuits()
  1 circuit(s).
  *---- Chordless circuits ----*    
  1: ['a6', 'a7', 'a8'] , credibility : 0.033
\end{lstlisting}
There is one chordless circuit detected in the given strict outranking digraph \texttt{gcd}, namely alternative \texttt{a6} outranks alternative \texttt{a7}, the latter outranks \texttt{a8}, and \texttt{a8} outranks again alternative \texttt{a6} (see Fig. \ref{fig:8.1}). Any potential linear ordering of these three alternatives will, in fact, always contradict somehow the given outranking relation.

Now, several heuristic ranking rules have been proposed for constructing a linear ordering which is closest in some specific sense to a given outranking relation. The \Digraph resources provide some of the most common of these ranking rules, like \Copeland 's, \Kemeny 's, \Slater 's, \Kohler 's, and the \RankedPairs ranking rule.

\section{The \Copeland ranking}
\label{sec:8.2}

\begin{definition}\label{def:copeland}\Copeland 'sranking rule computes for each alternative a score resulting from the sum of the differences between the crisp \emph{strict outranking} characteristics $r(x\, \succnsim \,y)_{>0}$ and the crisp \emph{strict outranked} characteristics $r(y\, \succnsim \, x)_{>0}$  for all pairs of alternatives where $y$ is different from $x$. The alternatives are ranked in decreasing order of these \Copeland scores; ties, the case given, being resolved with a lexicographical rule applied to the identifiers of the alternatives \citep{COP-1951}.
\end{definition}

\Copeland 's rule, the most intuitive one as it works well for any strict outranking relation which models a linear order on the \emph{median cut} strict outranking digraph \texttt{ccd}. 
\begin{lstlisting}[caption={Computing a \Copeland Ranking},label=list:8.3]
>>> from linearOrders import CopelandRanking
>>> cop = CopelandRanking(gcd,Comments=True)
 Copeland decreasing scores
     a5 : +12
     a1 :  +2
     a6 :  +2
     a7 :  +2
     a8 :   0
     a4 :  -3
     a9 :  -3
     a3 :  -5
     a2 :  -7
  Copeland Ranking:
  ['a5','a1','a6','a7','a8','a4','a9','a3','a2']
\end{lstlisting}
Alternative \texttt{a5} obtains here the best \Copeland score ($+12$), followed by alternatives \texttt{a1}, \texttt{a6} and \texttt{a7} with same score ($+2$); following the lexicographic rule, \texttt{a1} is hence ranked before \texttt{a6} and \texttt{a6} before \texttt{a7}. Same situation is observed for \texttt{a4} and \texttt{a9} with a score of $-3$ (see Listing \ref{list:8.3} Lines 4-12).

\Copeland 's ranking rule is in fact \emph{invariant} under the codual transform (see Section \ref{sec:2.6}) and renders a same linear order indifferently from digraphs \texttt{g} or \texttt{gcd} . The resulting ranking (see Listing \ref{list:8.3} Line 14) is rather correlated ($+0.463$) with the given pairwise outranking relation in the ordinal \Kendall sense\footnote{See Chapter~\ref{sec:16} and \citet{BIS-2012a}.}.
\begin{lstlisting}[caption={Checking the quality of the \Copeland ranking},label=list:8.4]
>>> corr = g.computeRankingCorrelation(cop.copelandRanking)
>>> g.showCorrelation(corr)
 Correlation indexes:
   Crisp ordinal correlation : +0.463
   Valued equivalalence      : +0.107
   Epistemic determination   :  0.230
\end{lstlisting}
With an epistemic determination level of $0.230$ (see above), the \emph{extended} \Kendall $\tau$ index is in fact computed on $61.5\% (100.0 x (1.0 + 0.23)/2)$ of the pairwise strict outranking comparisons. Furthermore, the bipolar-valued \emph{relational equivalence} characteristics between the strict outranking relation and the \Copeland ranking equals $+0.107$, i.e. a \emph{majority} of $55.35\%$ of the criteria significance supports the relational equivalence between the given strict outranking relation and the corresponding \Copeland ranking.

The \Copeland scores deliver actually only a \emph{weak ranking}, i.e. a ranking with potential ties. This weak ranking may be constructed with the\\
\texttt{WeakCopelandOrder} class \index{WeakCopelandOrder@\texttt{WeakCopelandOrder} class}.
\begin{lstlisting}[caption={Computing a weak \Copeland ranking},label=list:8.4]
>>> from transitiveDigraphs import WeakCopelandOrder
>>> wcop = WeakCopelandOrder(g)
>>> wcop.showRankingByChoosing()
 Ranking by Choosing and Rejecting
   1st ranked ['a5']
     2nd ranked ['a1', 'a6', 'a7']
       3rd ranked ['a8']
       3rd last ranked ['a4', 'a9']
     2nd last ranked ['a3']
   1st last ranked ['a2']
\end{lstlisting}
We recover in Listing \ref{list:8.4} above, the ranking with ties delivered by the \Copeland scores (see Listing \ref{list:8.3}). We may draw its corresponding skeleton.
\begin{lstlisting}
>>> wcop.exportGraphViz(fileName='weakCopelandRanking')
 *---- exporting a dot file for GraphViz tools ---------*
  Exporting to weakCopelandRanking.dot
  dot -Grankdir=TB -Tpng weakCopelandRanking.dot\
                   -o weakCopelandRanking.png
\end{lstlisting}
\begin{figure}[h]
\sidecaption[t]
\includegraphics[width=3cm]{Figures/weakCopelandRanking.png}
\caption{Drawing of the weak \Copeland ranking. The graph show the skeleton of the preorder produced by the corresponding ties of the \Copeland scores.}
\label{fig:8.2}       % Give a unique label
\end{figure}

Let us now consider a similar ranking rule, but working directly on the criteria \emph{significance majority margins}, i.e. the \emph{bipolar-valued} outranking relations.

\section{The \NetFlows ranking}
\label{sec:8.3}

\begin{definition}\label{def:netflows} The bipolar-valued version of the \Copeland ranking rule, we call \NetFlows \footnote{This ranking rule is also known under the name \Promethee ranking rule \citep*{BRA-1985}.}, computes for each alternative $x$ a \emph{net flow} score,  i.e. the sum of the differences between the \emph{strict outranking} characteristics $r(x\, \succnsim \,y)$ and the \emph{strict outranked} characteristics $r(y\, \succnsim \,x)$ for all pairs of alternatives where $y$ is different from $x$ .
\end{definition}
\begin{lstlisting}[caption={Computing a \NetFlows ranking},label=list:8.5]
>>> from linearOrders import NetFlowsRanking
>>> nf = NetFlowsRanking(gcd,Comments=True)
  Net Flows :
    a5 : +3.600
    a7 : +2.800
    a6 : +1.300
    a3 : +0.033
    a1 : -0.400
    a8 : -0.567
    a4 : -1.283
    a9 : -2.600
    a2 : -2.883
  NetFlows Ranking:
   ['a5','a7','a6','a3','a1','a8','a4','a9','a2']
>>> cop.copelandRanking # comparing both
   ['a5','a1','a6','a7','a8','a4','a9','a3','a2']
\end{lstlisting}
In our example here, the \NetFlows scores actually deliver a ranking \emph{without ties} which is rather different from the one delivered by \Copeland 's rule (see Listing~\ref{list:8.5} Line 16). It may happen, however, that we obtain, as with the \Copeland scores above, only a ranking with ties, which may then be resolved similarly by following a lexicographic rule applied to the identifiers of the decision alternatives. In such cases, it is possible to construct again a \emph{weak ranking} with the corresponding \texttt{WeakNetFlowsOrder} class\index{WeakNetFlowsOrder@\texttt{WeakNetFlowsOrder} class}.

It is worthwhile noticing again, that similar to the \Copeland ranking rule seen before, the \NetFlows ranking rule is also \emph{invariant} under the codual transform (see Secion \ref{sec:2.6}) and delivers again the same ranking result indifferently from digraphs \texttt{g} or \texttt{gcd} (see Listing \ref{list:8.5} Line 14). 

The \NetFlows ranking result appears to be slightly better correlated ($+0.638$) with the given outranking relation than its crisp cousin, the \Copeland ranking (see Lines 4-6 below).
\begin{lstlisting}[caption={Checking the quality of the \NetFlows Ranking},label=list:8.6]   
>>> corr = gcd.computeOrdinalCorrelation(nf)
>>> gcd.showCorrelation(corr)
 Correlation indexes:
   Extended Kendall tau       : +0.638
   Epistemic determination    :  0.230
   Bipolar-valued equivalence : +0.147
\end{lstlisting}
Indeed, the extended \Kendall $\tau$ index of $+0.638$ leads to a bipolar-valued \emph{relational equivalence} characteristics of $+0.147$, i.e. a majority of $57.35\%$ of the criteria significance supports the relational equivalence between the given outranking digraphs $g$ or $gcd$  and the corresponding \NetFlows ranking. This lesser ranking performance of the \Copeland rule stems in this example essentially from the \emph{weakness} of the actual ranking result and our subsequent \emph{arbitrary} lexicographic resolution of the many ties given by the \Copeland scores (see Fig. \ref{fig:8.2}).

To appreciate now the more or less correlation of both the \Copeland and the \NetFlows rankings with the underlying pairwise outranking relation, it is useful to consider \Kemeny 's and \Slater 's '\emph{optimal}' ranking rules.

\section{\Kemeny rankings}
\label{sec:8.4}

A \Kemeny ranking is a linear ranking without ties which is \emph{closest}, in the sense of the ordinal \Kendall distance (see Chapter~\ref{sec:16} and \citet{BIS-2012a}), to the given valued outranking digraphs \texttt{g} or \texttt{gcd} \citep{KEM-1959}. This rule is also \emph{invariant} under the codual transform. 
\begin{lstlisting}[caption={Computing a \Kemeny ranking},label=list:8.7]   
>>> from linearOrders import KemenyRanking
>>> ke = KemenyRanking(gcd,orderLimit=9)
>>> # default orderLimit is 7
>>> ke.showRanking()
 ['a5','a6','a7','a3','a9','a4','a1','a8','a2']
>>> corr = gcd.computeOrdinalCorrelation(ke)
>>> gcd.showCorrelation(corr)
 Correlation indexes:
   Extended Kendall tau       : +0.779
   Epistemic determination    :  0.230
   Bipolar-valued equivalence : +0.179
\end{lstlisting}    
So, $+0.779$ represents the \emph{highest possible} ordinal correlation --\emph{fitness}-- any potential linear ranking can achieve with the given pairwise outranking digraph (see Listing \ref{list:8.7} Lines 7-10).

A \Kemeny ranking may not be unique. In our example here, we obtain in fact two such \Kemeny rankings with a same \emph{maximal} \Kemeny index of $12.9$. 
\begin{lstlisting}[caption={Optimal \Kemeny rankings},label=list:8.8] >>> ke.maximalRankings
  [['a5','a6','a7','a3','a8','a9','a4','a1','a2'],
   ['a5','a6','a7','a3','a9','a4','a1','a8','a2']]
>>> ke.maxKemenyIndex
 Decimal('12.9166667')
\end{lstlisting}

We may visualize the partial order defined by the epistemic disjunction (see Section \ref{sec:2.5}) of both optimal \Kemeny rankings by using the \texttt{RankingsFusion} class\index{RankingsFusion@\texttt{RankingsFusion} class}.
\begin{lstlisting}[caption={Computing the epistemic disjunction of all optimal \Kemeny rankings},label=list:8.9]   
>>> from transitiveDigraphs import RankingsFusion
>>> wke = RankingsFusion(ke,ke.maximalRankings)
>>> wke.exportGraphViz(fileName='tutorialKemeny')
 *---- exporting a dot file for GraphViz tools ---------*
  Exporting to tutorialKemeny.dot
  dot -Grankdir=TB -Tpng tutorialKemeny.dot -o tutorialKemeny.png
\end{lstlisting}
\begin{figure}[h]
\sidecaption[t]
\includegraphics[width=3cm]{Figures/tutorialKemeny.png}
\caption{Epistemic disjunction of optimal \Kemeny rankings. It is interesting to notice that both \Kemeny rankings only differ in their respective positioning of alternative \texttt{a8}; either before or after alternatives \texttt{a9}, \texttt{a4} and \texttt{a1}. }
\label{fig:8.3}       % Give a unique label
\end{figure}

To retain now a specific representative among all the potential rankings with maximal \Kemeny index, we will choose, with the help of the \\
\texttt{showRankingConsensusQuality()} method\index{showRankingConsensusQuality@Showrankingconsensusquality()}, the one proposing the best performance criteria consensus.
\begin{lstlisting}[caption={Computing the consensus quality of a ranking},label=list:8.10]   
>>> g.showRankingConsensusQuality(ke.maximalRankings[0])
 Consensus quality of ranking:
  ['a5','a6','a7','a3','a8','a9','a4','a1','a2']
  criterion (weight): correlation
  -------------------------------
      b09 (0.050)  : +0.361
      b04 (0.050)  : +0.333
      b08 (0.050)  : +0.292
      b01 (0.050)  : +0.264
      c01 (0.167)  : +0.250
      b03 (0.050)  : +0.222
      b07 (0.050)  : +0.194
      b05 (0.050)  : +0.167
      c02 (0.167)  : +0.000
      b10 (0.050)  : +0.000
      b02 (0.050)  : -0.042
      b06 (0.050)  : -0.097
      c03 (0.167)  : -0.167
  Summary:
    Weighted mean marginal correlation (a): +0.099
    Standard deviation (b)                : +0.177
    Ranking fairness (a)-(b)              : -0.079
>>> g.showRankingConsensusQuality(ke.maximalRankings[1])
 Consensus quality of ranking:
  ['a5','a6','a7','a3','a9','a4','a1','a8','a2']
  criterion (weight): correlation
  -------------------------------
      b09 (0.050)   : +0.306
      b08 (0.050)   : +0.236
      c01 (0.167)   : +0.194
      b07 (0.050)   : +0.194
      c02 (0.167)   : +0.167
      b04 (0.050)   : +0.167
      b03 (0.050)   : +0.167
      b01 (0.050)   : +0.153
      b05 (0.050)   : +0.056
      b02 (0.050)   : +0.014
      b06 (0.050)   : -0.042
      c03 (0.167)   : -0.111
      b10 (0.050)   : -0.111
  Summary:
    Weighted mean marginal correlation (a): +0.099
    Standard deviation (b)                : +0.132
    Ranking fairness (a)-(b)              : -0.033
\end{lstlisting}
Both \Kemeny rankings show the same \emph{weighted mean marginal correlation} ($+0.099$, see Listing \ref{list:8.10} Lines 19-22, 42-44) with all thirteen performance criteria. However, the second ranking shows a slightly lower \emph{standard deviation} ($+0.132$ versus $+0.177$), resulting in a slightly \emph{fairer} ranking result ($-0.033$ versus $-0.079$).

When several rankings with maximal \Kemeny index are given,\\ the \texttt{KemenyRanking} class constructor instantiates the ranking with \emph{highest} mean marginal correlation and, in case of ties, with \emph{lowest} weighted standard deviation. Here we obtain ranking: [\texttt{a5}, \texttt{a6}, \texttt{a7}, \texttt{a3}, \texttt{a9}, \texttt{a4}, \texttt{a1}, \texttt{a8}, \texttt{a2}] (see Line 4 in Listing~\ref{list:8.10} above).

\section{\Slater rankings}
\label{sec:8.5}

The \Slater ranking rule is identical to \Kemeny 's, but it is working, instead, on the \Condorcet --\emph{median cut polarised}-- digraph \texttt{ccd} \citep{SLA-1961}. \Slater 's rule is also \emph{invariant} Under the codual transform and delivers again indifferently on \texttt{g} or \texttt{gcd} the following results:
\begin{lstlisting}[caption={Computing a \Slater ranking},label=list:8.11]   
>>> from linearOrders import SlaterRanking
>>> sl = SlaterRanking(gcd,orderLimit=9)
>>> sl.slaterRanking
  ['a5','a6','a4','a1','a3','a7','a8','a9','a2']
>>> corr = gcd.computeOrderCorrelation(sl.slaterRanking)
>>> sl.showCorrelation(corr)
  Correlation indexes:
   Extended Kendall tau       : +0.676
   Epistemic determination    :  0.230
   Bipolar-valued equivalence : +0.156
>>> len(sl.maximalRankings)
  7
\end{lstlisting}
We notice in Listing~\ref{list:8.11} Line 8 that the first \Slater ranking is a rather good fit ($+0.676$), slightly better apparently than the \NetFlows ranking result ($+0.638$). However, there are in fact 7 such potentially optimal \Slater rankings (see Line 12). The corresponding epistemic disjunction gives the partial ordering shown in Fig~\ref{fig:8.4}:
\begin{lstlisting}[caption={Computing the epistemic disjunction of optimal \Slater rankings},label=list:8.12]   
>>> slw = RankingsFusion(sl,sl.maximalRankings)
>>> slw.exportGraphViz(fileName='tutorialSlater')
 *---- exporting a dot file for GraphViz tools ----*
  Exporting to tutorialSlater.dot
  dot -Grankdir=TB -Tpng tutorialSlater.dot\
                   -o tutorialSlater.png
\end{lstlisting}
\begin{figure}[h]
\sidecaption[t]
\includegraphics[width=4cm]{Figures/tutorialSlater.png}
\caption{Epistemic disjunction of optimal \Slater rankings. What precise \Slater ranking result should we hence adopt?}
\label{fig:8.4}       % Give a unique label
\end{figure}
       
\Kemeny 's and \Slater 's ranking rules are furthermore computationally \emph{difficult} problems and effective ranking results are only computable for tiny outranking digraphs ($< 20$ objects). 

More efficient ranking heuristics, like the \Copeland and the \NetFlows rules, are therefore needed in practice. Let us finally, after these \emph{ranking-by-scoring} strategies, also present two popular \emph{ranking-by-choosing} strategies.

\section{\Kohler 's ranking-by-choosing rule}
\label{sec:8.6}

\Kohler 's \emph{ranking-by-choosing} rule\footnote{\citep{KOH-1978}} is formulated like this:
\begin{definition}[\Kohler 's \emph{ranking-by-choosing} rule]\label{def:kohler}
  
\noindent At step $i$ ($i$ goes from 1 to $n$) do the following:
\begin{enumerate}[leftmargin=0.5cm,rightmargin=0.5cm]
\item Compute for each row of the bipolar-valued \emph{strict} outranking relation table (see Listing \ref{list:8.1}) the smallest value;
\item Select the row where this minimum is maximal. Ties are resolved in lexicographic order;
\item Put the selected decision alternative at rank $i$;
\item Delete the corresponding row and column from the relation table and restart until the table is empty.
\end{enumerate}
\end{definition}
\begin{lstlisting}[caption={Computing a \Kohler ranking},label=list:8.13]   
>>> from linearOrders import KohlerRanking
>>> kocd = KohlerRanking(gcd)
>>> kocd.showRanking()
  ['a5','a7','a6','a3','a9','a8','a4','a1','a2']
>>> corr = gcd.computeOrdinalCorrelation(kocd)
>>> gcd.showCorrelation(corr)
  Correlation indexes:
    Extended Kendall tau       : +0.747
    Epistemic determination    :  0.230
    Bipolar-valued equivalence : +0.172
\end{lstlisting}

With this \emph{min-max} lexicographic ranking-by-choosing strategy, we find a correlation result ($+0.747$) that is until now clearly the nearest to an optimal \Kemeny ranking (see Listing \ref{list:8.8}). Only two adjacent pairs: (\texttt{a6}, \texttt{a7}) and (\texttt{a8}, \texttt{a9}) are actually inverted here. Notice that \Kohler 's ranking rule, contrary to the previously mentioned rules, is \textbf{not} invariant under the codual transform and requires to work on the \texttt{strict} outranking digraph \texttt{gcd} for a better correlation result.
\begin{lstlisting}
>>> ko = KohlerRanking(g)  
>>> corr = g.computeOrdinalCorrelation(ko)
>>> g.showCorrelation(corr)
  Correlation indexes:
   Crisp ordinal correlation  : +0.483
   Epistemic determination    :  0.230
   Bipolar-valued equivalence : +0.111
\end{lstlisting}

But \Kohler 's ranking has a \emph{dual} version, the prudent \emph{Arrow-Raynaud} ordering-by-choosing rule, where a corresponding \emph{max-min} strategy, when used on the \emph{non-strict} outranking digraph $g$, for ordering from \emph{last} to \emph{first} produces a similar ranking result \citep{ARR-1986}.

Noticing that the \NetFlows score of an alternative $x$ represents in fact a bipolar-valued characteristic of the assertion ``\emph{alternative x is ranked first}'', we may enhance \Kohler 's rule by replacing the simple \emph{min-max} strategy with an \emph{iterated} maximal \NetFlows score selection.

\begin{definition}[The iterated \NetFlows ranking-by-choosing rule]
  
\noindent At step $i$ ($i$ goes from 1 to $n$) do the following:
\begin{enumerate}[leftmargin=0.5cm,rightmargin=0.5cm]
\item Compute for each row of the bipolar-valued outranking relation table (see Listing \ref{list:8.1}) the corresponding \NetFlows score;
\item Select the row where this score is \emph{maximal}, ties being resolved by lexicographic order;
\item Put the corresponding decision alternative at rank $i$;
\item Delete the corresponding row and column from the relation table and restart until the table is empty.
\end{enumerate}
\end{definition}

The \texttt{IteratedNetFlowsRanking}\index{IteratedNetFlowsRanking@\texttt{IteratedNetFlowsRanking} class} class from the \texttt{linearOrders} module computes this ranking result. 
\begin{lstlisting}[caption={Ranking-by-choosing with iterated maximal \NetFlows scores},label=list:8.14]   
>>> from linearOrders import IteratedNetFlowsRanking  
>>> inf = IteratedNetFlowsRanking(g)
>>> inf
 *------- Digraph instance description ------*
   Instance class      : IteratedNetFlowsRanking
   Instance name       : rel_randomCBperftab_ranked
   Digraph Order       : 9
   Digraph Size        : 36
   Valuation domain    : [-1.00;1.00]
   Determinateness (%) : 100.00
   Attributes     : ['valuedRanks', 'valuedOrdering',
                     'iteratedNetFlowsRanking',
                     'iteratedNetFlowsOrdering',
                     'name', 'actions', 'order',
                     'valuationdomain', 'relation',
                     'gamma', 'notGamma']
>>> inf.iteratedNetFlowsRanking
  ['a5','a7','a6','a3','a4','a1','a8','a9','a2']
>>> corr = g.computeRankingCorrelation(\
...             inf.iteratedNetFlowsRanking)
>>> g.showCorrelation(corr)
  Correlation indexes:
    Crisp ordinal correlation  : +0.743
    Epistemic determination    :  0.230
    Bipolar-valued equivalence : +0.171
\end{lstlisting}

Like \Kohler 's rule, the iterated \NetFlows 's rule has also a dual \emph{ordering-by-choosing} version, where instead of choosing at each step $i$ the row with maximal \NetFlows score, we choose the row with the \emph{minimal} \NetFlows score. Both the ranking and ordering result are computed by the \texttt{IteratedNetFlowsRanking} class constructor (see Lines 12 and 13 in Listing~\ref{list:8.14}).
\begin{lstlisting}
>>> inf.iteratedNetFlowsOrdering
  ['a2','a9','a1','a4','a3','a8','a7','a6','a5']
>>> corr = g.computeOrderCorrelation(\
...                inf.iteratedNetFlowsOrdering)
>>> g.showCorrelation(corr)
  Correlation indexes:
    Crisp ordinal correlation  : +0.751
    Epistemic determination    : 0.230
    Bipolar-valued equivalence : +0.173
\end{lstlisting}
The iterated \NetFlows ranking and its dual, the iterated \NetFlows ordering, do not usually deliver both the same result. With our example outranking digraph $g$ for instance, it is the \emph{ordering-by-choosing} result who obtains a slightly better correlation with the given outranking digraph ($+0.751$), a result that is also slightly better than \Kohler 's original result ($+0.747$, see Listing \ref{list:8.13} Line 8).

With different \emph{ranking-by-choosing} and \emph{ordering-by-choosing} results, it may be useful to \emph{fuse} now, similar to what we have done before with \Kemeny 's and \Slater 's optimal rankings, both, the iterated \NetFlows ranking and ordering into a partial ranking. But we are hence back to the practical problem of what linear ranking should we eventually retain? 

Let us finally mention another interesting \emph{ranking-by-choosing} approach.

\section{The \RankedPairs ranking rule}
\label{sec:8.7}

Tideman's \index{Tideman@\emph{Tideman}} ranking-by-choosing heuristic, the \RankedPairs rule, working best this time on the non strict outranking digraph $g$, is based on a \emph{prudent incremental} construction of linear orders that avoids on the fly any cycling outrankings \citep{TID-1987}. The ranking rule may be formulated as follows:
\begin{definition}[The \RankedPairs ranking rule]
\begin{enumerate}[leftmargin=0.5cm,rightmargin=0.5cm]
 \item Rank the ordered pairs $(x,y)$ of alternatives in decreasing order of $r(x\, \succsim \,y) \,+\, r(y\, \not\succsim \,x)$;
 \item Consider the pairs in that order (ties are resolved by a lexicographic rule):
   \begin{itemize}
     \item if the next pair does not create a \emph{circuit} with the pairs already blocked, block this pair;
     \item if the next pair creates a \emph{circuit} with the already blocked pairs, skip it.
    \end{itemize}
\end{enumerate}
\end{definition}  
With our didactic outranking digraph $g$, we get the following result.\index{RankedPairsRanking@\texttt{RankedPairsRanking} class}
\begin{lstlisting}[caption={Computing a \RankedPairs ranking},label=list:8.15]   
>>> from linearOrders import RankedPairsRanking
>>> rp = RankedPairsRanking(g)
>>> rp.showRanking()
  ['a5','a6','a7','a3','a8','a9','a4','a1','a2']
\end{lstlisting}

The \RankedPairs rule renders in our example here luckily one of the two optimal \Kemeny ranking, as we may verify below.
 \begin{lstlisting}
>>> ke.maximalRankings
  [['a5','a6','a7','a3','a8','a9','a4','a1','a2'],
   ['a5','a6','a7','a3','a9','a4','a1','a8','a2']]
>>> corr = g.computeOrdinalCorrelation(rp)
>>> g.showCorrelation(corr)
  Correlation indexes:
    Extended Kendall tau       : +0.779
    Epistemic determination    :  0.230
    Bipolar-valued equivalence : +0.179
\end{lstlisting}

Similar to \Kohler 's rule, the \RankedPairs rule has also a prudent dual version, the \emph{Dias-Lamboray} \emph{ordering-by-choosing} rule, which produces, when working this time on the codual strict outranking digraph gcd, a similar ranking result (see \citet*{DIA-2010}).

Besides of not providing a unique linear ranking, the \emph{ranking-by-choosing} rules, as well as their duals, the \emph{ordering-by-choosing} rules, are unfortunately not scalable to outranking digraphs of larger orders ($> 100$). For such bigger outranking digraphs, with several hundred or thousands of alternatives, only the \Copeland and the \NetFlows \emph{ranking-by-scoring} rules, with a polynomial complexity of $\mathcal{O}(n^2)$, where $n$ is the order of the outranking digraph, remain in fact tractable. Furthermore, as computing the \Copeland and \NetFlows scores may be done separately per alternative, the latter ranking rules can right away be processed in parallel when multiprocessing resources are available.

\vspace{1cm}

%%%%%%% The chapter bibliography
%\normallatexbib
\clearpage
%\phantomsection
%\addcontentsline{toc}{section}{Chapter Bibliography}
\bibliographystyle{spbasic}
%\typeout{}
\bibliography{03-backMatters/reference}
%\chapter{Ranking with multiple incommensurable criteria}
\label{sec:8}

\abstract*{ The \Digraph python resources provide several algorithms for solving the ranking problem with a bipolar-valued outranking digraph. The \Copeland, \NetFlows, \Kemeny, \Slater, \Kohler, and the \RankedPairs ranking rules are presented and illustrated with a random outranking digraph.}

\abstract{The \Digraph python resources provide several algorithms for solving the ranking problem with a bipolar-valued outranking digraph. The \Copeland, \NetFlows, \Kemeny, \Slater, \Kohler, and the \RankedPairs ranking rules are presented and illustrated with a random outranking digraph.}

\section{The ranking problem}
\label{sec:8.1}

We need to rank without ties a set $X$ of items (usually decision alternatives) that are evaluated on multiple incommensurable performance criteria; yet, for which we may know their pairwise bipolar-valued {\em strict outranking\/} characteristics, i.e. $r(x\, \succnsim \, y)$ for all $x, y \in X$ (see Section \ref{sec:3.5} and \citep{BIS-2013}).

Let us consider a didactic outranking digraph \texttt{g} generated from a random \emph{Cost-Benefit} performance tableau (see Section \ref{sec:6.3}) concerning 9 decision alternatives evaluated on 13 performance criteria. We may compute the corresponding {\em strict outranking digraph\/} with a codual transform (see Section \ref{sec:2.6}).

\begin{lstlisting}[caption={Random bipolar-valued strict outranking relation characteristics},label=list:8.1]
>>> from randomPerfTabs import RandomCBPerformanceTableau   
>>> t = RandomCBPerformanceTableau(numberOfActions=9,\
...         numberOfCriteria=13,seed=200)
>>> from outrankingDigraphs import BipolarOutrankingDigraph
>>> g = BipolarOutrankingDigraph(t,Normalized=True)
>>> gcd = ~(-g) # codual digraph
>>> gcd.showRelationTable(ReflexiveTerms=False)
 * ---- Relation Table -----
  r(>) |  'a1'  'a2'  'a3'  'a4'  'a5'  'a6'  'a7'  'a8'  'a9'   
  -----|------------------------------------------------------
  'a1' |    -   0.00 +0.10 -1.00 -0.13 -0.57 -0.23 +0.10 +0.00  
  'a2' | -1.00   -    0.00 +0.00 -0.37 -0.42 -0.28 -0.32 -0.12  
  'a3' | -0.10  0.00   -   -0.17 -0.35 -0.30 -0.17 -0.17 +0.00  
  'a4' |  0.00  0.00 -0.42   -   -0.40 -0.20 -0.60 -0.27 -0.30  
  'a5' | +0.13 +0.22 +0.10 +0.40   -   +0.03 +0.40 -0.03 -0.07  
  'a6' | -0.07 -0.22 +0.20 +0.20 -0.37   -   +0.10 -0.03 -0.07  
  'a7' | -0.20 +0.28 -0.03 -0.07 -0.40 -0.10   -   +0.27 +1.00  
  'a8' | -0.10 -0.02 -0.23 -0.13 -0.37 +0.03 -0.27   -   +0.03  
  'a9' |  0.00 +0.12 -1.00 -0.13 -0.03 -0.03 -1.00 -0.03   -   
\end{lstlisting}
  
Some ranking rules will work on the associated \Condorcet digraph\index{digraph!Condorcet}, i.e. the corresponding \emph{strict median cut} polarised strict outranking digraph.
 \begin{lstlisting}[caption={Median cut polarised strict outranking relation characteristics},label=list:8.2]
>>> ccd = PolarisedOutrankingDigraph(gcd,\
...                   level=g.valuationdomain['med'],\
...                   KeepValues=False,StrictCut=True)
>>> ccd.showRelationTable(ReflexiveTerms=False,\
...                       IntegerValues=True)
 *---- Relation Table -----
  r(>)_med | 'a1' 'a2' 'a3' 'a4' 'a5' 'a6' 'a7' 'a8' 'a9'   
  ---------|---------------------------------------------
     'a1'  |   -    0   +1   -1   -1   -1   -1   +1    0  
     'a2'  |  -1    -   +0    0   -1   -1   -1   -1   -1  
     'a3'  |  -1    0    -   -1   -1   -1   -1   -1    0  
     'a4'  |   0    0   -1    -   -1   -1   -1   -1   -1  
     'a5'  |  +1   +1   +1   +1    -   +1   +1   -1   -1  
     'a6'  |  -1   -1   +1   +1   -1    -   +1   -1   -1  
     'a7'  |  -1   +1   -1   -1   -1   -1    -   +1   +1  
     'a8'  |  -1   -1   -1   -1   -1   +1   -1    -   +1  
     'a9'  |   0   +1   -1   -1   -1   -1   -1   -1    -   
\end{lstlisting}

Unfortunately, such crisp median-cut \Condorcet digraphs, associated with a given strict outranking digraph, only exceptionally present a linear ordering. Usually, pairwise majority comparisons do not even render a \emph{complete} or, at least, a \emph{transitive} partial order. There may even frequently appear \emph{cyclic} outranking situations (see Section \ref{sec:7.4}).

To discover how \emph{difficult} this ranking problem can get here, we have a look in Fig.~\ref{fig:8.1} at the corresponding strict outranking digraph \emph{graphviz} drawing \footnote{ The \texttt{exportGraphViz()} method is depending on drawing tools from graphviz software (https://graphviz.org/).}.
\begin{lstlisting}
>>> gcd.exportGraphViz(fileName='rankingTutorial')
 *---- exporting a dot file for GraphViz tools ---------*
  Exporting to rankingTutorial.dot
  dot -Grankdir=BT -Tpng rankingTutorial.dot\
                   -o rankingTutorial.png
\end{lstlisting}
\begin{figure}[h]
\sidecaption[t]
\includegraphics[width=5.5cm]{Figures/rankingTutorial.pdf}
\caption{The strict outranking relation $\succnsim$ shown here is, for instance, \emph{not transitive}: alternative \texttt{a8} outranks alternative \texttt{a6} and alternative \texttt{a6} outranks \texttt{a4}, however, \texttt{a8} does not outrank \texttt{a4}. Furthermore, alternatives \texttt{a8}, \texttt{a6} and \texttt{a7} show a cyclic outranking relation. }
\label{fig:8.1}       % Give a unique label
\end{figure}

We may compute the \emph{transitivity degree} of the outranking digraph shown in Fig.~\ref{fig:8.1}, i.e. the ratio of the difference between the number of outranking arcs and the number of transitive arcs over the difference of the number of arcs of the transitive closure minus the transitive arcs of the digraph \texttt{gcd}.
\begin{lstlisting}
>>> gcd.computeTransitivityDegree(Comments=True)
 Transitivity degree of graph <codual_rel_randomCBperftab>
  triples x>y>z: 78, closed: 38, open: 40
  closed/triples = 0.487
\end{lstlisting}    
With only $49\%$ of the required transitive arcs, the strict outranking relation here is hence very far from being transitive; a serious problem when a linear ordering of the decision alternatives is looked for.

Let us furthermore see if there are any cyclic outrankings.
\begin{lstlisting}
>>> gcd.computeChordlessCircuits()
>>> gcd.showChordlessCircuits()
  1 circuit(s).
  *---- Chordless circuits ----*    
  1: ['a6', 'a7', 'a8'] , credibility : 0.033
\end{lstlisting}
There is one chordless circuit detected in the given strict outranking digraph \texttt{gcd}, namely alternative \texttt{a6} outranks alternative \texttt{a7}, the latter outranks \texttt{a8}, and \texttt{a8} outranks again alternative \texttt{a6} (see Fig. \ref{fig:8.1}). Any potential linear ordering of these three alternatives will, in fact, always contradict somehow the given outranking relation.

Now, several heuristic ranking rules have been proposed for constructing a linear ordering which is closest in some specific sense to a given outranking relation. The \Digraph resources provide some of the most common of these ranking rules, like \Copeland 's, \Kemeny 's, \Slater 's, \Kohler 's, and the \RankedPairs ranking rule.

\section{The \Copeland ranking}
\label{sec:8.2}

\begin{definition}\label{def:copeland}\Copeland 'sranking rule computes for each alternative a score resulting from the sum of the differences between the crisp \emph{strict outranking} characteristics $r(x\, \succnsim \,y)_{>0}$ and the crisp \emph{strict outranked} characteristics $r(y\, \succnsim \, x)_{>0}$  for all pairs of alternatives where $y$ is different from $x$. The alternatives are ranked in decreasing order of these \Copeland scores; ties, the case given, being resolved with a lexicographical rule applied to the identifiers of the alternatives \citep{COP-1951}.
\end{definition}

\Copeland 's rule, the most intuitive one as it works well for any strict outranking relation which models a linear order on the \emph{median cut} strict outranking digraph \texttt{ccd}. 
\begin{lstlisting}[caption={Computing a \Copeland Ranking},label=list:8.3]
>>> from linearOrders import CopelandRanking
>>> cop = CopelandRanking(gcd,Comments=True)
 Copeland decreasing scores
     a5 : +12
     a1 :  +2
     a6 :  +2
     a7 :  +2
     a8 :   0
     a4 :  -3
     a9 :  -3
     a3 :  -5
     a2 :  -7
  Copeland Ranking:
  ['a5','a1','a6','a7','a8','a4','a9','a3','a2']
\end{lstlisting}
Alternative \texttt{a5} obtains here the best \Copeland score ($+12$), followed by alternatives \texttt{a1}, \texttt{a6} and \texttt{a7} with same score ($+2$); following the lexicographic rule, \texttt{a1} is hence ranked before \texttt{a6} and \texttt{a6} before \texttt{a7}. Same situation is observed for \texttt{a4} and \texttt{a9} with a score of $-3$ (see Listing \ref{list:8.3} Lines 4-12).

\Copeland 's ranking rule is in fact \emph{invariant} under the codual transform (see Section \ref{sec:2.6}) and renders a same linear order indifferently from digraphs \texttt{g} or \texttt{gcd} . The resulting ranking (see Listing \ref{list:8.3} Line 14) is rather correlated ($+0.463$) with the given pairwise outranking relation in the ordinal \Kendall sense\footnote{See Chapter~\ref{sec:16} and \citet{BIS-2012a}.}.
\begin{lstlisting}[caption={Checking the quality of the \Copeland ranking},label=list:8.4]
>>> corr = g.computeRankingCorrelation(cop.copelandRanking)
>>> g.showCorrelation(corr)
 Correlation indexes:
   Crisp ordinal correlation : +0.463
   Valued equivalalence      : +0.107
   Epistemic determination   :  0.230
\end{lstlisting}
With an epistemic determination level of $0.230$ (see above), the \emph{extended} \Kendall $\tau$ index is in fact computed on $61.5\% (100.0 x (1.0 + 0.23)/2)$ of the pairwise strict outranking comparisons. Furthermore, the bipolar-valued \emph{relational equivalence} characteristics between the strict outranking relation and the \Copeland ranking equals $+0.107$, i.e. a \emph{majority} of $55.35\%$ of the criteria significance supports the relational equivalence between the given strict outranking relation and the corresponding \Copeland ranking.

The \Copeland scores deliver actually only a \emph{weak ranking}, i.e. a ranking with potential ties. This weak ranking may be constructed with the\\
\texttt{WeakCopelandOrder} class \index{WeakCopelandOrder@\texttt{WeakCopelandOrder} class}.
\begin{lstlisting}[caption={Computing a weak \Copeland ranking},label=list:8.4]
>>> from transitiveDigraphs import WeakCopelandOrder
>>> wcop = WeakCopelandOrder(g)
>>> wcop.showRankingByChoosing()
 Ranking by Choosing and Rejecting
   1st ranked ['a5']
     2nd ranked ['a1', 'a6', 'a7']
       3rd ranked ['a8']
       3rd last ranked ['a4', 'a9']
     2nd last ranked ['a3']
   1st last ranked ['a2']
\end{lstlisting}
We recover in Listing \ref{list:8.4} above, the ranking with ties delivered by the \Copeland scores (see Listing \ref{list:8.3}). We may draw its corresponding skeleton.
\begin{lstlisting}
>>> wcop.exportGraphViz(fileName='weakCopelandRanking')
 *---- exporting a dot file for GraphViz tools ---------*
  Exporting to weakCopelandRanking.dot
  dot -Grankdir=TB -Tpng weakCopelandRanking.dot\
                   -o weakCopelandRanking.png
\end{lstlisting}
\begin{figure}[h]
\sidecaption[t]
\includegraphics[width=3cm]{Figures/weakCopelandRanking.png}
\caption{Drawing of the weak \Copeland ranking. The graph show the skeleton of the preorder produced by the corresponding ties of the \Copeland scores.}
\label{fig:8.2}       % Give a unique label
\end{figure}

Let us now consider a similar ranking rule, but working directly on the criteria \emph{significance majority margins}, i.e. the \emph{bipolar-valued} outranking relations.

\section{The \NetFlows ranking}
\label{sec:8.3}

\begin{definition}\label{def:netflows} The bipolar-valued version of the \Copeland ranking rule, we call \NetFlows \footnote{This ranking rule is also known under the name \Promethee ranking rule \citep*{BRA-1985}.}, computes for each alternative $x$ a \emph{net flow} score,  i.e. the sum of the differences between the \emph{strict outranking} characteristics $r(x\, \succnsim \,y)$ and the \emph{strict outranked} characteristics $r(y\, \succnsim \,x)$ for all pairs of alternatives where $y$ is different from $x$ .
\end{definition}
\begin{lstlisting}[caption={Computing a \NetFlows ranking},label=list:8.5]
>>> from linearOrders import NetFlowsRanking
>>> nf = NetFlowsRanking(gcd,Comments=True)
  Net Flows :
    a5 : +3.600
    a7 : +2.800
    a6 : +1.300
    a3 : +0.033
    a1 : -0.400
    a8 : -0.567
    a4 : -1.283
    a9 : -2.600
    a2 : -2.883
  NetFlows Ranking:
   ['a5','a7','a6','a3','a1','a8','a4','a9','a2']
>>> cop.copelandRanking # comparing both
   ['a5','a1','a6','a7','a8','a4','a9','a3','a2']
\end{lstlisting}
In our example here, the \NetFlows scores actually deliver a ranking \emph{without ties} which is rather different from the one delivered by \Copeland 's rule (see Listing~\ref{list:8.5} Line 16). It may happen, however, that we obtain, as with the \Copeland scores above, only a ranking with ties, which may then be resolved similarly by following a lexicographic rule applied to the identifiers of the decision alternatives. In such cases, it is possible to construct again a \emph{weak ranking} with the corresponding \texttt{WeakNetFlowsOrder} class\index{WeakNetFlowsOrder@\texttt{WeakNetFlowsOrder} class}.

It is worthwhile noticing again, that similar to the \Copeland ranking rule seen before, the \NetFlows ranking rule is also \emph{invariant} under the codual transform (see Secion \ref{sec:2.6}) and delivers again the same ranking result indifferently from digraphs \texttt{g} or \texttt{gcd} (see Listing \ref{list:8.5} Line 14). 

The \NetFlows ranking result appears to be slightly better correlated ($+0.638$) with the given outranking relation than its crisp cousin, the \Copeland ranking (see Lines 4-6 below).
\begin{lstlisting}[caption={Checking the quality of the \NetFlows Ranking},label=list:8.6]   
>>> corr = gcd.computeOrdinalCorrelation(nf)
>>> gcd.showCorrelation(corr)
 Correlation indexes:
   Extended Kendall tau       : +0.638
   Epistemic determination    :  0.230
   Bipolar-valued equivalence : +0.147
\end{lstlisting}
Indeed, the extended \Kendall $\tau$ index of $+0.638$ leads to a bipolar-valued \emph{relational equivalence} characteristics of $+0.147$, i.e. a majority of $57.35\%$ of the criteria significance supports the relational equivalence between the given outranking digraphs $g$ or $gcd$  and the corresponding \NetFlows ranking. This lesser ranking performance of the \Copeland rule stems in this example essentially from the \emph{weakness} of the actual ranking result and our subsequent \emph{arbitrary} lexicographic resolution of the many ties given by the \Copeland scores (see Fig. \ref{fig:8.2}).

To appreciate now the more or less correlation of both the \Copeland and the \NetFlows rankings with the underlying pairwise outranking relation, it is useful to consider \Kemeny 's and \Slater 's '\emph{optimal}' ranking rules.

\section{\Kemeny rankings}
\label{sec:8.4}

A \Kemeny ranking is a linear ranking without ties which is \emph{closest}, in the sense of the ordinal \Kendall distance (see Chapter~\ref{sec:16} and \citet{BIS-2012a}), to the given valued outranking digraphs \texttt{g} or \texttt{gcd} \citep{KEM-1959}. This rule is also \emph{invariant} under the codual transform. 
\begin{lstlisting}[caption={Computing a \Kemeny ranking},label=list:8.7]   
>>> from linearOrders import KemenyRanking
>>> ke = KemenyRanking(gcd,orderLimit=9)
>>> # default orderLimit is 7
>>> ke.showRanking()
 ['a5','a6','a7','a3','a9','a4','a1','a8','a2']
>>> corr = gcd.computeOrdinalCorrelation(ke)
>>> gcd.showCorrelation(corr)
 Correlation indexes:
   Extended Kendall tau       : +0.779
   Epistemic determination    :  0.230
   Bipolar-valued equivalence : +0.179
\end{lstlisting}    
So, $+0.779$ represents the \emph{highest possible} ordinal correlation --\emph{fitness}-- any potential linear ranking can achieve with the given pairwise outranking digraph (see Listing \ref{list:8.7} Lines 7-10).

A \Kemeny ranking may not be unique. In our example here, we obtain in fact two such \Kemeny rankings with a same \emph{maximal} \Kemeny index of $12.9$. 
\begin{lstlisting}[caption={Optimal \Kemeny rankings},label=list:8.8] >>> ke.maximalRankings
  [['a5','a6','a7','a3','a8','a9','a4','a1','a2'],
   ['a5','a6','a7','a3','a9','a4','a1','a8','a2']]
>>> ke.maxKemenyIndex
 Decimal('12.9166667')
\end{lstlisting}

We may visualize the partial order defined by the epistemic disjunction (see Section \ref{sec:2.5}) of both optimal \Kemeny rankings by using the \texttt{RankingsFusion} class\index{RankingsFusion@\texttt{RankingsFusion} class}.
\begin{lstlisting}[caption={Computing the epistemic disjunction of all optimal \Kemeny rankings},label=list:8.9]   
>>> from transitiveDigraphs import RankingsFusion
>>> wke = RankingsFusion(ke,ke.maximalRankings)
>>> wke.exportGraphViz(fileName='tutorialKemeny')
 *---- exporting a dot file for GraphViz tools ---------*
  Exporting to tutorialKemeny.dot
  dot -Grankdir=TB -Tpng tutorialKemeny.dot -o tutorialKemeny.png
\end{lstlisting}
\begin{figure}[h]
\sidecaption[t]
\includegraphics[width=3cm]{Figures/tutorialKemeny.png}
\caption{Epistemic disjunction of optimal \Kemeny rankings. It is interesting to notice that both \Kemeny rankings only differ in their respective positioning of alternative \texttt{a8}; either before or after alternatives \texttt{a9}, \texttt{a4} and \texttt{a1}. }
\label{fig:8.3}       % Give a unique label
\end{figure}

To retain now a specific representative among all the potential rankings with maximal \Kemeny index, we will choose, with the help of the \\
\texttt{showRankingConsensusQuality()} method\index{showRankingConsensusQuality@Showrankingconsensusquality()}, the one proposing the best performance criteria consensus.
\begin{lstlisting}[caption={Computing the consensus quality of a ranking},label=list:8.10]   
>>> g.showRankingConsensusQuality(ke.maximalRankings[0])
 Consensus quality of ranking:
  ['a5','a6','a7','a3','a8','a9','a4','a1','a2']
  criterion (weight): correlation
  -------------------------------
      b09 (0.050)  : +0.361
      b04 (0.050)  : +0.333
      b08 (0.050)  : +0.292
      b01 (0.050)  : +0.264
      c01 (0.167)  : +0.250
      b03 (0.050)  : +0.222
      b07 (0.050)  : +0.194
      b05 (0.050)  : +0.167
      c02 (0.167)  : +0.000
      b10 (0.050)  : +0.000
      b02 (0.050)  : -0.042
      b06 (0.050)  : -0.097
      c03 (0.167)  : -0.167
  Summary:
    Weighted mean marginal correlation (a): +0.099
    Standard deviation (b)                : +0.177
    Ranking fairness (a)-(b)              : -0.079
>>> g.showRankingConsensusQuality(ke.maximalRankings[1])
 Consensus quality of ranking:
  ['a5','a6','a7','a3','a9','a4','a1','a8','a2']
  criterion (weight): correlation
  -------------------------------
      b09 (0.050)   : +0.306
      b08 (0.050)   : +0.236
      c01 (0.167)   : +0.194
      b07 (0.050)   : +0.194
      c02 (0.167)   : +0.167
      b04 (0.050)   : +0.167
      b03 (0.050)   : +0.167
      b01 (0.050)   : +0.153
      b05 (0.050)   : +0.056
      b02 (0.050)   : +0.014
      b06 (0.050)   : -0.042
      c03 (0.167)   : -0.111
      b10 (0.050)   : -0.111
  Summary:
    Weighted mean marginal correlation (a): +0.099
    Standard deviation (b)                : +0.132
    Ranking fairness (a)-(b)              : -0.033
\end{lstlisting}
Both \Kemeny rankings show the same \emph{weighted mean marginal correlation} ($+0.099$, see Listing \ref{list:8.10} Lines 19-22, 42-44) with all thirteen performance criteria. However, the second ranking shows a slightly lower \emph{standard deviation} ($+0.132$ versus $+0.177$), resulting in a slightly \emph{fairer} ranking result ($-0.033$ versus $-0.079$).

When several rankings with maximal \Kemeny index are given,\\ the \texttt{KemenyRanking} class constructor instantiates the ranking with \emph{highest} mean marginal correlation and, in case of ties, with \emph{lowest} weighted standard deviation. Here we obtain ranking: [\texttt{a5}, \texttt{a6}, \texttt{a7}, \texttt{a3}, \texttt{a9}, \texttt{a4}, \texttt{a1}, \texttt{a8}, \texttt{a2}] (see Line 4 in Listing~\ref{list:8.10} above).

\section{\Slater rankings}
\label{sec:8.5}

The \Slater ranking rule is identical to \Kemeny 's, but it is working, instead, on the \Condorcet --\emph{median cut polarised}-- digraph \texttt{ccd} \citep{SLA-1961}. \Slater 's rule is also \emph{invariant} Under the codual transform and delivers again indifferently on \texttt{g} or \texttt{gcd} the following results:
\begin{lstlisting}[caption={Computing a \Slater ranking},label=list:8.11]   
>>> from linearOrders import SlaterRanking
>>> sl = SlaterRanking(gcd,orderLimit=9)
>>> sl.slaterRanking
  ['a5','a6','a4','a1','a3','a7','a8','a9','a2']
>>> corr = gcd.computeOrderCorrelation(sl.slaterRanking)
>>> sl.showCorrelation(corr)
  Correlation indexes:
   Extended Kendall tau       : +0.676
   Epistemic determination    :  0.230
   Bipolar-valued equivalence : +0.156
>>> len(sl.maximalRankings)
  7
\end{lstlisting}
We notice in Listing~\ref{list:8.11} Line 8 that the first \Slater ranking is a rather good fit ($+0.676$), slightly better apparently than the \NetFlows ranking result ($+0.638$). However, there are in fact 7 such potentially optimal \Slater rankings (see Line 12). The corresponding epistemic disjunction gives the partial ordering shown in Fig~\ref{fig:8.4}:
\begin{lstlisting}[caption={Computing the epistemic disjunction of optimal \Slater rankings},label=list:8.12]   
>>> slw = RankingsFusion(sl,sl.maximalRankings)
>>> slw.exportGraphViz(fileName='tutorialSlater')
 *---- exporting a dot file for GraphViz tools ----*
  Exporting to tutorialSlater.dot
  dot -Grankdir=TB -Tpng tutorialSlater.dot\
                   -o tutorialSlater.png
\end{lstlisting}
\begin{figure}[h]
\sidecaption[t]
\includegraphics[width=4cm]{Figures/tutorialSlater.png}
\caption{Epistemic disjunction of optimal \Slater rankings. What precise \Slater ranking result should we hence adopt?}
\label{fig:8.4}       % Give a unique label
\end{figure}
       
\Kemeny 's and \Slater 's ranking rules are furthermore computationally \emph{difficult} problems and effective ranking results are only computable for tiny outranking digraphs ($< 20$ objects). 

More efficient ranking heuristics, like the \Copeland and the \NetFlows rules, are therefore needed in practice. Let us finally, after these \emph{ranking-by-scoring} strategies, also present two popular \emph{ranking-by-choosing} strategies.

\section{\Kohler 's ranking-by-choosing rule}
\label{sec:8.6}

\Kohler 's \emph{ranking-by-choosing} rule\footnote{\citep{KOH-1978}} is formulated like this:
\begin{definition}[\Kohler 's \emph{ranking-by-choosing} rule]\label{def:kohler}
  
\noindent At step $i$ ($i$ goes from 1 to $n$) do the following:
\begin{enumerate}[leftmargin=0.5cm,rightmargin=0.5cm]
\item Compute for each row of the bipolar-valued \emph{strict} outranking relation table (see Listing \ref{list:8.1}) the smallest value;
\item Select the row where this minimum is maximal. Ties are resolved in lexicographic order;
\item Put the selected decision alternative at rank $i$;
\item Delete the corresponding row and column from the relation table and restart until the table is empty.
\end{enumerate}
\end{definition}
\begin{lstlisting}[caption={Computing a \Kohler ranking},label=list:8.13]   
>>> from linearOrders import KohlerRanking
>>> kocd = KohlerRanking(gcd)
>>> kocd.showRanking()
  ['a5','a7','a6','a3','a9','a8','a4','a1','a2']
>>> corr = gcd.computeOrdinalCorrelation(kocd)
>>> gcd.showCorrelation(corr)
  Correlation indexes:
    Extended Kendall tau       : +0.747
    Epistemic determination    :  0.230
    Bipolar-valued equivalence : +0.172
\end{lstlisting}

With this \emph{min-max} lexicographic ranking-by-choosing strategy, we find a correlation result ($+0.747$) that is until now clearly the nearest to an optimal \Kemeny ranking (see Listing \ref{list:8.8}). Only two adjacent pairs: (\texttt{a6}, \texttt{a7}) and (\texttt{a8}, \texttt{a9}) are actually inverted here. Notice that \Kohler 's ranking rule, contrary to the previously mentioned rules, is \textbf{not} invariant under the codual transform and requires to work on the \texttt{strict} outranking digraph \texttt{gcd} for a better correlation result.
\begin{lstlisting}
>>> ko = KohlerRanking(g)  
>>> corr = g.computeOrdinalCorrelation(ko)
>>> g.showCorrelation(corr)
  Correlation indexes:
   Crisp ordinal correlation  : +0.483
   Epistemic determination    :  0.230
   Bipolar-valued equivalence : +0.111
\end{lstlisting}

But \Kohler 's ranking has a \emph{dual} version, the prudent \emph{Arrow-Raynaud} ordering-by-choosing rule, where a corresponding \emph{max-min} strategy, when used on the \emph{non-strict} outranking digraph $g$, for ordering from \emph{last} to \emph{first} produces a similar ranking result \citep{ARR-1986}.

Noticing that the \NetFlows score of an alternative $x$ represents in fact a bipolar-valued characteristic of the assertion ``\emph{alternative x is ranked first}'', we may enhance \Kohler 's rule by replacing the simple \emph{min-max} strategy with an \emph{iterated} maximal \NetFlows score selection.

\begin{definition}[The iterated \NetFlows ranking-by-choosing rule]
  
\noindent At step $i$ ($i$ goes from 1 to $n$) do the following:
\begin{enumerate}[leftmargin=0.5cm,rightmargin=0.5cm]
\item Compute for each row of the bipolar-valued outranking relation table (see Listing \ref{list:8.1}) the corresponding \NetFlows score;
\item Select the row where this score is \emph{maximal}, ties being resolved by lexicographic order;
\item Put the corresponding decision alternative at rank $i$;
\item Delete the corresponding row and column from the relation table and restart until the table is empty.
\end{enumerate}
\end{definition}

The \texttt{IteratedNetFlowsRanking}\index{IteratedNetFlowsRanking@\texttt{IteratedNetFlowsRanking} class} class from the \texttt{linearOrders} module computes this ranking result. 
\begin{lstlisting}[caption={Ranking-by-choosing with iterated maximal \NetFlows scores},label=list:8.14]   
>>> from linearOrders import IteratedNetFlowsRanking  
>>> inf = IteratedNetFlowsRanking(g)
>>> inf
 *------- Digraph instance description ------*
   Instance class      : IteratedNetFlowsRanking
   Instance name       : rel_randomCBperftab_ranked
   Digraph Order       : 9
   Digraph Size        : 36
   Valuation domain    : [-1.00;1.00]
   Determinateness (%) : 100.00
   Attributes     : ['valuedRanks', 'valuedOrdering',
                     'iteratedNetFlowsRanking',
                     'iteratedNetFlowsOrdering',
                     'name', 'actions', 'order',
                     'valuationdomain', 'relation',
                     'gamma', 'notGamma']
>>> inf.iteratedNetFlowsRanking
  ['a5','a7','a6','a3','a4','a1','a8','a9','a2']
>>> corr = g.computeRankingCorrelation(\
...             inf.iteratedNetFlowsRanking)
>>> g.showCorrelation(corr)
  Correlation indexes:
    Crisp ordinal correlation  : +0.743
    Epistemic determination    :  0.230
    Bipolar-valued equivalence : +0.171
\end{lstlisting}

Like \Kohler 's rule, the iterated \NetFlows 's rule has also a dual \emph{ordering-by-choosing} version, where instead of choosing at each step $i$ the row with maximal \NetFlows score, we choose the row with the \emph{minimal} \NetFlows score. Both the ranking and ordering result are computed by the \texttt{IteratedNetFlowsRanking} class constructor (see Lines 12 and 13 in Listing~\ref{list:8.14}).
\begin{lstlisting}
>>> inf.iteratedNetFlowsOrdering
  ['a2','a9','a1','a4','a3','a8','a7','a6','a5']
>>> corr = g.computeOrderCorrelation(\
...                inf.iteratedNetFlowsOrdering)
>>> g.showCorrelation(corr)
  Correlation indexes:
    Crisp ordinal correlation  : +0.751
    Epistemic determination    : 0.230
    Bipolar-valued equivalence : +0.173
\end{lstlisting}
The iterated \NetFlows ranking and its dual, the iterated \NetFlows ordering, do not usually deliver both the same result. With our example outranking digraph $g$ for instance, it is the \emph{ordering-by-choosing} result who obtains a slightly better correlation with the given outranking digraph ($+0.751$), a result that is also slightly better than \Kohler 's original result ($+0.747$, see Listing \ref{list:8.13} Line 8).

With different \emph{ranking-by-choosing} and \emph{ordering-by-choosing} results, it may be useful to \emph{fuse} now, similar to what we have done before with \Kemeny 's and \Slater 's optimal rankings, both, the iterated \NetFlows ranking and ordering into a partial ranking. But we are hence back to the practical problem of what linear ranking should we eventually retain? 

Let us finally mention another interesting \emph{ranking-by-choosing} approach.

\section{The \RankedPairs ranking rule}
\label{sec:8.7}

Tideman's \index{Tideman@\emph{Tideman}} ranking-by-choosing heuristic, the \RankedPairs rule, working best this time on the non strict outranking digraph $g$, is based on a \emph{prudent incremental} construction of linear orders that avoids on the fly any cycling outrankings \citep{TID-1987}. The ranking rule may be formulated as follows:
\begin{definition}[The \RankedPairs ranking rule]
\begin{enumerate}[leftmargin=0.5cm,rightmargin=0.5cm]
 \item Rank the ordered pairs $(x,y)$ of alternatives in decreasing order of $r(x\, \succsim \,y) \,+\, r(y\, \not\succsim \,x)$;
 \item Consider the pairs in that order (ties are resolved by a lexicographic rule):
   \begin{itemize}
     \item if the next pair does not create a \emph{circuit} with the pairs already blocked, block this pair;
     \item if the next pair creates a \emph{circuit} with the already blocked pairs, skip it.
    \end{itemize}
\end{enumerate}
\end{definition}  
With our didactic outranking digraph $g$, we get the following result.\index{RankedPairsRanking@\texttt{RankedPairsRanking} class}
\begin{lstlisting}[caption={Computing a \RankedPairs ranking},label=list:8.15]   
>>> from linearOrders import RankedPairsRanking
>>> rp = RankedPairsRanking(g)
>>> rp.showRanking()
  ['a5','a6','a7','a3','a8','a9','a4','a1','a2']
\end{lstlisting}

The \RankedPairs rule renders in our example here luckily one of the two optimal \Kemeny ranking, as we may verify below.
 \begin{lstlisting}
>>> ke.maximalRankings
  [['a5','a6','a7','a3','a8','a9','a4','a1','a2'],
   ['a5','a6','a7','a3','a9','a4','a1','a8','a2']]
>>> corr = g.computeOrdinalCorrelation(rp)
>>> g.showCorrelation(corr)
  Correlation indexes:
    Extended Kendall tau       : +0.779
    Epistemic determination    :  0.230
    Bipolar-valued equivalence : +0.179
\end{lstlisting}

Similar to \Kohler 's rule, the \RankedPairs rule has also a prudent dual version, the \emph{Dias-Lamboray} \emph{ordering-by-choosing} rule, which produces, when working this time on the codual strict outranking digraph gcd, a similar ranking result (see \citet*{DIA-2010}).

Besides of not providing a unique linear ranking, the \emph{ranking-by-choosing} rules, as well as their duals, the \emph{ordering-by-choosing} rules, are unfortunately not scalable to outranking digraphs of larger orders ($> 100$). For such bigger outranking digraphs, with several hundred or thousands of alternatives, only the \Copeland and the \NetFlows \emph{ranking-by-scoring} rules, with a polynomial complexity of $\mathcal{O}(n^2)$, where $n$ is the order of the outranking digraph, remain in fact tractable. Furthermore, as computing the \Copeland and \NetFlows scores may be done separately per alternative, the latter ranking rules can right away be processed in parallel when multiprocessing resources are available.

\vspace{1cm}

%%%%%%% The chapter bibliography
%\normallatexbib
\clearpage
%\phantomsection
%\addcontentsline{toc}{section}{Chapter Bibliography}
\bibliographystyle{spbasic}
%\typeout{}
\bibliography{03-backMatters/reference}
%\chapter{Ranking with multiple incommensurable criteria}
\label{sec:8}

\abstract*{ The \Digraph python resources provide several algorithms for solving the ranking problem with a bipolar-valued outranking digraph. The \Copeland, \NetFlows, \Kemeny, \Slater, \Kohler, and the \RankedPairs ranking rules are presented and illustrated with a random outranking digraph.}

\abstract{The \Digraph python resources provide several algorithms for solving the ranking problem with a bipolar-valued outranking digraph. The \Copeland, \NetFlows, \Kemeny, \Slater, \Kohler, and the \RankedPairs ranking rules are presented and illustrated with a random outranking digraph.}

\section{The ranking problem}
\label{sec:8.1}

We need to rank without ties a set $X$ of items (usually decision alternatives) that are evaluated on multiple incommensurable performance criteria; yet, for which we may know their pairwise bipolar-valued {\em strict outranking\/} characteristics, i.e. $r(x\, \succnsim \, y)$ for all $x, y \in X$ (see Section \ref{sec:3.5} and \citep{BIS-2013}).

Let us consider a didactic outranking digraph \texttt{g} generated from a random \emph{Cost-Benefit} performance tableau (see Section \ref{sec:6.3}) concerning 9 decision alternatives evaluated on 13 performance criteria. We may compute the corresponding {\em strict outranking digraph\/} with a codual transform (see Section \ref{sec:2.6}).

\begin{lstlisting}[caption={Random bipolar-valued strict outranking relation characteristics},label=list:8.1]
>>> from randomPerfTabs import RandomCBPerformanceTableau   
>>> t = RandomCBPerformanceTableau(numberOfActions=9,\
...         numberOfCriteria=13,seed=200)
>>> from outrankingDigraphs import BipolarOutrankingDigraph
>>> g = BipolarOutrankingDigraph(t,Normalized=True)
>>> gcd = ~(-g) # codual digraph
>>> gcd.showRelationTable(ReflexiveTerms=False)
 * ---- Relation Table -----
  r(>) |  'a1'  'a2'  'a3'  'a4'  'a5'  'a6'  'a7'  'a8'  'a9'   
  -----|------------------------------------------------------
  'a1' |    -   0.00 +0.10 -1.00 -0.13 -0.57 -0.23 +0.10 +0.00  
  'a2' | -1.00   -    0.00 +0.00 -0.37 -0.42 -0.28 -0.32 -0.12  
  'a3' | -0.10  0.00   -   -0.17 -0.35 -0.30 -0.17 -0.17 +0.00  
  'a4' |  0.00  0.00 -0.42   -   -0.40 -0.20 -0.60 -0.27 -0.30  
  'a5' | +0.13 +0.22 +0.10 +0.40   -   +0.03 +0.40 -0.03 -0.07  
  'a6' | -0.07 -0.22 +0.20 +0.20 -0.37   -   +0.10 -0.03 -0.07  
  'a7' | -0.20 +0.28 -0.03 -0.07 -0.40 -0.10   -   +0.27 +1.00  
  'a8' | -0.10 -0.02 -0.23 -0.13 -0.37 +0.03 -0.27   -   +0.03  
  'a9' |  0.00 +0.12 -1.00 -0.13 -0.03 -0.03 -1.00 -0.03   -   
\end{lstlisting}
  
Some ranking rules will work on the associated \Condorcet digraph\index{digraph!Condorcet}, i.e. the corresponding \emph{strict median cut} polarised strict outranking digraph.
 \begin{lstlisting}[caption={Median cut polarised strict outranking relation characteristics},label=list:8.2]
>>> ccd = PolarisedOutrankingDigraph(gcd,\
...                   level=g.valuationdomain['med'],\
...                   KeepValues=False,StrictCut=True)
>>> ccd.showRelationTable(ReflexiveTerms=False,\
...                       IntegerValues=True)
 *---- Relation Table -----
  r(>)_med | 'a1' 'a2' 'a3' 'a4' 'a5' 'a6' 'a7' 'a8' 'a9'   
  ---------|---------------------------------------------
     'a1'  |   -    0   +1   -1   -1   -1   -1   +1    0  
     'a2'  |  -1    -   +0    0   -1   -1   -1   -1   -1  
     'a3'  |  -1    0    -   -1   -1   -1   -1   -1    0  
     'a4'  |   0    0   -1    -   -1   -1   -1   -1   -1  
     'a5'  |  +1   +1   +1   +1    -   +1   +1   -1   -1  
     'a6'  |  -1   -1   +1   +1   -1    -   +1   -1   -1  
     'a7'  |  -1   +1   -1   -1   -1   -1    -   +1   +1  
     'a8'  |  -1   -1   -1   -1   -1   +1   -1    -   +1  
     'a9'  |   0   +1   -1   -1   -1   -1   -1   -1    -   
\end{lstlisting}

Unfortunately, such crisp median-cut \Condorcet digraphs, associated with a given strict outranking digraph, only exceptionally present a linear ordering. Usually, pairwise majority comparisons do not even render a \emph{complete} or, at least, a \emph{transitive} partial order. There may even frequently appear \emph{cyclic} outranking situations (see Section \ref{sec:7.4}).

To discover how \emph{difficult} this ranking problem can get here, we have a look in Fig.~\ref{fig:8.1} at the corresponding strict outranking digraph \emph{graphviz} drawing \footnote{ The \texttt{exportGraphViz()} method is depending on drawing tools from graphviz software (https://graphviz.org/).}.
\begin{lstlisting}
>>> gcd.exportGraphViz(fileName='rankingTutorial')
 *---- exporting a dot file for GraphViz tools ---------*
  Exporting to rankingTutorial.dot
  dot -Grankdir=BT -Tpng rankingTutorial.dot\
                   -o rankingTutorial.png
\end{lstlisting}
\begin{figure}[h]
\sidecaption[t]
\includegraphics[width=5.5cm]{Figures/rankingTutorial.pdf}
\caption{The strict outranking relation $\succnsim$ shown here is, for instance, \emph{not transitive}: alternative \texttt{a8} outranks alternative \texttt{a6} and alternative \texttt{a6} outranks \texttt{a4}, however, \texttt{a8} does not outrank \texttt{a4}. Furthermore, alternatives \texttt{a8}, \texttt{a6} and \texttt{a7} show a cyclic outranking relation. }
\label{fig:8.1}       % Give a unique label
\end{figure}

We may compute the \emph{transitivity degree} of the outranking digraph shown in Fig.~\ref{fig:8.1}, i.e. the ratio of the difference between the number of outranking arcs and the number of transitive arcs over the difference of the number of arcs of the transitive closure minus the transitive arcs of the digraph \texttt{gcd}.
\begin{lstlisting}
>>> gcd.computeTransitivityDegree(Comments=True)
 Transitivity degree of graph <codual_rel_randomCBperftab>
  triples x>y>z: 78, closed: 38, open: 40
  closed/triples = 0.487
\end{lstlisting}    
With only $49\%$ of the required transitive arcs, the strict outranking relation here is hence very far from being transitive; a serious problem when a linear ordering of the decision alternatives is looked for.

Let us furthermore see if there are any cyclic outrankings.
\begin{lstlisting}
>>> gcd.computeChordlessCircuits()
>>> gcd.showChordlessCircuits()
  1 circuit(s).
  *---- Chordless circuits ----*    
  1: ['a6', 'a7', 'a8'] , credibility : 0.033
\end{lstlisting}
There is one chordless circuit detected in the given strict outranking digraph \texttt{gcd}, namely alternative \texttt{a6} outranks alternative \texttt{a7}, the latter outranks \texttt{a8}, and \texttt{a8} outranks again alternative \texttt{a6} (see Fig. \ref{fig:8.1}). Any potential linear ordering of these three alternatives will, in fact, always contradict somehow the given outranking relation.

Now, several heuristic ranking rules have been proposed for constructing a linear ordering which is closest in some specific sense to a given outranking relation. The \Digraph resources provide some of the most common of these ranking rules, like \Copeland 's, \Kemeny 's, \Slater 's, \Kohler 's, and the \RankedPairs ranking rule.

\section{The \Copeland ranking}
\label{sec:8.2}

\begin{definition}\label{def:copeland}\Copeland 'sranking rule computes for each alternative a score resulting from the sum of the differences between the crisp \emph{strict outranking} characteristics $r(x\, \succnsim \,y)_{>0}$ and the crisp \emph{strict outranked} characteristics $r(y\, \succnsim \, x)_{>0}$  for all pairs of alternatives where $y$ is different from $x$. The alternatives are ranked in decreasing order of these \Copeland scores; ties, the case given, being resolved with a lexicographical rule applied to the identifiers of the alternatives \citep{COP-1951}.
\end{definition}

\Copeland 's rule, the most intuitive one as it works well for any strict outranking relation which models a linear order on the \emph{median cut} strict outranking digraph \texttt{ccd}. 
\begin{lstlisting}[caption={Computing a \Copeland Ranking},label=list:8.3]
>>> from linearOrders import CopelandRanking
>>> cop = CopelandRanking(gcd,Comments=True)
 Copeland decreasing scores
     a5 : +12
     a1 :  +2
     a6 :  +2
     a7 :  +2
     a8 :   0
     a4 :  -3
     a9 :  -3
     a3 :  -5
     a2 :  -7
  Copeland Ranking:
  ['a5','a1','a6','a7','a8','a4','a9','a3','a2']
\end{lstlisting}
Alternative \texttt{a5} obtains here the best \Copeland score ($+12$), followed by alternatives \texttt{a1}, \texttt{a6} and \texttt{a7} with same score ($+2$); following the lexicographic rule, \texttt{a1} is hence ranked before \texttt{a6} and \texttt{a6} before \texttt{a7}. Same situation is observed for \texttt{a4} and \texttt{a9} with a score of $-3$ (see Listing \ref{list:8.3} Lines 4-12).

\Copeland 's ranking rule is in fact \emph{invariant} under the codual transform (see Section \ref{sec:2.6}) and renders a same linear order indifferently from digraphs \texttt{g} or \texttt{gcd} . The resulting ranking (see Listing \ref{list:8.3} Line 14) is rather correlated ($+0.463$) with the given pairwise outranking relation in the ordinal \Kendall sense\footnote{See Chapter~\ref{sec:16} and \citet{BIS-2012a}.}.
\begin{lstlisting}[caption={Checking the quality of the \Copeland ranking},label=list:8.4]
>>> corr = g.computeRankingCorrelation(cop.copelandRanking)
>>> g.showCorrelation(corr)
 Correlation indexes:
   Crisp ordinal correlation : +0.463
   Valued equivalalence      : +0.107
   Epistemic determination   :  0.230
\end{lstlisting}
With an epistemic determination level of $0.230$ (see above), the \emph{extended} \Kendall $\tau$ index is in fact computed on $61.5\% (100.0 x (1.0 + 0.23)/2)$ of the pairwise strict outranking comparisons. Furthermore, the bipolar-valued \emph{relational equivalence} characteristics between the strict outranking relation and the \Copeland ranking equals $+0.107$, i.e. a \emph{majority} of $55.35\%$ of the criteria significance supports the relational equivalence between the given strict outranking relation and the corresponding \Copeland ranking.

The \Copeland scores deliver actually only a \emph{weak ranking}, i.e. a ranking with potential ties. This weak ranking may be constructed with the\\
\texttt{WeakCopelandOrder} class \index{WeakCopelandOrder@\texttt{WeakCopelandOrder} class}.
\begin{lstlisting}[caption={Computing a weak \Copeland ranking},label=list:8.4]
>>> from transitiveDigraphs import WeakCopelandOrder
>>> wcop = WeakCopelandOrder(g)
>>> wcop.showRankingByChoosing()
 Ranking by Choosing and Rejecting
   1st ranked ['a5']
     2nd ranked ['a1', 'a6', 'a7']
       3rd ranked ['a8']
       3rd last ranked ['a4', 'a9']
     2nd last ranked ['a3']
   1st last ranked ['a2']
\end{lstlisting}
We recover in Listing \ref{list:8.4} above, the ranking with ties delivered by the \Copeland scores (see Listing \ref{list:8.3}). We may draw its corresponding skeleton.
\begin{lstlisting}
>>> wcop.exportGraphViz(fileName='weakCopelandRanking')
 *---- exporting a dot file for GraphViz tools ---------*
  Exporting to weakCopelandRanking.dot
  dot -Grankdir=TB -Tpng weakCopelandRanking.dot\
                   -o weakCopelandRanking.png
\end{lstlisting}
\begin{figure}[h]
\sidecaption[t]
\includegraphics[width=3cm]{Figures/weakCopelandRanking.png}
\caption{Drawing of the weak \Copeland ranking. The graph show the skeleton of the preorder produced by the corresponding ties of the \Copeland scores.}
\label{fig:8.2}       % Give a unique label
\end{figure}

Let us now consider a similar ranking rule, but working directly on the criteria \emph{significance majority margins}, i.e. the \emph{bipolar-valued} outranking relations.

\section{The \NetFlows ranking}
\label{sec:8.3}

\begin{definition}\label{def:netflows} The bipolar-valued version of the \Copeland ranking rule, we call \NetFlows \footnote{This ranking rule is also known under the name \Promethee ranking rule \citep*{BRA-1985}.}, computes for each alternative $x$ a \emph{net flow} score,  i.e. the sum of the differences between the \emph{strict outranking} characteristics $r(x\, \succnsim \,y)$ and the \emph{strict outranked} characteristics $r(y\, \succnsim \,x)$ for all pairs of alternatives where $y$ is different from $x$ .
\end{definition}
\begin{lstlisting}[caption={Computing a \NetFlows ranking},label=list:8.5]
>>> from linearOrders import NetFlowsRanking
>>> nf = NetFlowsRanking(gcd,Comments=True)
  Net Flows :
    a5 : +3.600
    a7 : +2.800
    a6 : +1.300
    a3 : +0.033
    a1 : -0.400
    a8 : -0.567
    a4 : -1.283
    a9 : -2.600
    a2 : -2.883
  NetFlows Ranking:
   ['a5','a7','a6','a3','a1','a8','a4','a9','a2']
>>> cop.copelandRanking # comparing both
   ['a5','a1','a6','a7','a8','a4','a9','a3','a2']
\end{lstlisting}
In our example here, the \NetFlows scores actually deliver a ranking \emph{without ties} which is rather different from the one delivered by \Copeland 's rule (see Listing~\ref{list:8.5} Line 16). It may happen, however, that we obtain, as with the \Copeland scores above, only a ranking with ties, which may then be resolved similarly by following a lexicographic rule applied to the identifiers of the decision alternatives. In such cases, it is possible to construct again a \emph{weak ranking} with the corresponding \texttt{WeakNetFlowsOrder} class\index{WeakNetFlowsOrder@\texttt{WeakNetFlowsOrder} class}.

It is worthwhile noticing again, that similar to the \Copeland ranking rule seen before, the \NetFlows ranking rule is also \emph{invariant} under the codual transform (see Secion \ref{sec:2.6}) and delivers again the same ranking result indifferently from digraphs \texttt{g} or \texttt{gcd} (see Listing \ref{list:8.5} Line 14). 

The \NetFlows ranking result appears to be slightly better correlated ($+0.638$) with the given outranking relation than its crisp cousin, the \Copeland ranking (see Lines 4-6 below).
\begin{lstlisting}[caption={Checking the quality of the \NetFlows Ranking},label=list:8.6]   
>>> corr = gcd.computeOrdinalCorrelation(nf)
>>> gcd.showCorrelation(corr)
 Correlation indexes:
   Extended Kendall tau       : +0.638
   Epistemic determination    :  0.230
   Bipolar-valued equivalence : +0.147
\end{lstlisting}
Indeed, the extended \Kendall $\tau$ index of $+0.638$ leads to a bipolar-valued \emph{relational equivalence} characteristics of $+0.147$, i.e. a majority of $57.35\%$ of the criteria significance supports the relational equivalence between the given outranking digraphs $g$ or $gcd$  and the corresponding \NetFlows ranking. This lesser ranking performance of the \Copeland rule stems in this example essentially from the \emph{weakness} of the actual ranking result and our subsequent \emph{arbitrary} lexicographic resolution of the many ties given by the \Copeland scores (see Fig. \ref{fig:8.2}).

To appreciate now the more or less correlation of both the \Copeland and the \NetFlows rankings with the underlying pairwise outranking relation, it is useful to consider \Kemeny 's and \Slater 's '\emph{optimal}' ranking rules.

\section{\Kemeny rankings}
\label{sec:8.4}

A \Kemeny ranking is a linear ranking without ties which is \emph{closest}, in the sense of the ordinal \Kendall distance (see Chapter~\ref{sec:16} and \citet{BIS-2012a}), to the given valued outranking digraphs \texttt{g} or \texttt{gcd} \citep{KEM-1959}. This rule is also \emph{invariant} under the codual transform. 
\begin{lstlisting}[caption={Computing a \Kemeny ranking},label=list:8.7]   
>>> from linearOrders import KemenyRanking
>>> ke = KemenyRanking(gcd,orderLimit=9)
>>> # default orderLimit is 7
>>> ke.showRanking()
 ['a5','a6','a7','a3','a9','a4','a1','a8','a2']
>>> corr = gcd.computeOrdinalCorrelation(ke)
>>> gcd.showCorrelation(corr)
 Correlation indexes:
   Extended Kendall tau       : +0.779
   Epistemic determination    :  0.230
   Bipolar-valued equivalence : +0.179
\end{lstlisting}    
So, $+0.779$ represents the \emph{highest possible} ordinal correlation --\emph{fitness}-- any potential linear ranking can achieve with the given pairwise outranking digraph (see Listing \ref{list:8.7} Lines 7-10).

A \Kemeny ranking may not be unique. In our example here, we obtain in fact two such \Kemeny rankings with a same \emph{maximal} \Kemeny index of $12.9$. 
\begin{lstlisting}[caption={Optimal \Kemeny rankings},label=list:8.8] >>> ke.maximalRankings
  [['a5','a6','a7','a3','a8','a9','a4','a1','a2'],
   ['a5','a6','a7','a3','a9','a4','a1','a8','a2']]
>>> ke.maxKemenyIndex
 Decimal('12.9166667')
\end{lstlisting}

We may visualize the partial order defined by the epistemic disjunction (see Section \ref{sec:2.5}) of both optimal \Kemeny rankings by using the \texttt{RankingsFusion} class\index{RankingsFusion@\texttt{RankingsFusion} class}.
\begin{lstlisting}[caption={Computing the epistemic disjunction of all optimal \Kemeny rankings},label=list:8.9]   
>>> from transitiveDigraphs import RankingsFusion
>>> wke = RankingsFusion(ke,ke.maximalRankings)
>>> wke.exportGraphViz(fileName='tutorialKemeny')
 *---- exporting a dot file for GraphViz tools ---------*
  Exporting to tutorialKemeny.dot
  dot -Grankdir=TB -Tpng tutorialKemeny.dot -o tutorialKemeny.png
\end{lstlisting}
\begin{figure}[h]
\sidecaption[t]
\includegraphics[width=3cm]{Figures/tutorialKemeny.png}
\caption{Epistemic disjunction of optimal \Kemeny rankings. It is interesting to notice that both \Kemeny rankings only differ in their respective positioning of alternative \texttt{a8}; either before or after alternatives \texttt{a9}, \texttt{a4} and \texttt{a1}. }
\label{fig:8.3}       % Give a unique label
\end{figure}

To retain now a specific representative among all the potential rankings with maximal \Kemeny index, we will choose, with the help of the \\
\texttt{showRankingConsensusQuality()} method\index{showRankingConsensusQuality@Showrankingconsensusquality()}, the one proposing the best performance criteria consensus.
\begin{lstlisting}[caption={Computing the consensus quality of a ranking},label=list:8.10]   
>>> g.showRankingConsensusQuality(ke.maximalRankings[0])
 Consensus quality of ranking:
  ['a5','a6','a7','a3','a8','a9','a4','a1','a2']
  criterion (weight): correlation
  -------------------------------
      b09 (0.050)  : +0.361
      b04 (0.050)  : +0.333
      b08 (0.050)  : +0.292
      b01 (0.050)  : +0.264
      c01 (0.167)  : +0.250
      b03 (0.050)  : +0.222
      b07 (0.050)  : +0.194
      b05 (0.050)  : +0.167
      c02 (0.167)  : +0.000
      b10 (0.050)  : +0.000
      b02 (0.050)  : -0.042
      b06 (0.050)  : -0.097
      c03 (0.167)  : -0.167
  Summary:
    Weighted mean marginal correlation (a): +0.099
    Standard deviation (b)                : +0.177
    Ranking fairness (a)-(b)              : -0.079
>>> g.showRankingConsensusQuality(ke.maximalRankings[1])
 Consensus quality of ranking:
  ['a5','a6','a7','a3','a9','a4','a1','a8','a2']
  criterion (weight): correlation
  -------------------------------
      b09 (0.050)   : +0.306
      b08 (0.050)   : +0.236
      c01 (0.167)   : +0.194
      b07 (0.050)   : +0.194
      c02 (0.167)   : +0.167
      b04 (0.050)   : +0.167
      b03 (0.050)   : +0.167
      b01 (0.050)   : +0.153
      b05 (0.050)   : +0.056
      b02 (0.050)   : +0.014
      b06 (0.050)   : -0.042
      c03 (0.167)   : -0.111
      b10 (0.050)   : -0.111
  Summary:
    Weighted mean marginal correlation (a): +0.099
    Standard deviation (b)                : +0.132
    Ranking fairness (a)-(b)              : -0.033
\end{lstlisting}
Both \Kemeny rankings show the same \emph{weighted mean marginal correlation} ($+0.099$, see Listing \ref{list:8.10} Lines 19-22, 42-44) with all thirteen performance criteria. However, the second ranking shows a slightly lower \emph{standard deviation} ($+0.132$ versus $+0.177$), resulting in a slightly \emph{fairer} ranking result ($-0.033$ versus $-0.079$).

When several rankings with maximal \Kemeny index are given,\\ the \texttt{KemenyRanking} class constructor instantiates the ranking with \emph{highest} mean marginal correlation and, in case of ties, with \emph{lowest} weighted standard deviation. Here we obtain ranking: [\texttt{a5}, \texttt{a6}, \texttt{a7}, \texttt{a3}, \texttt{a9}, \texttt{a4}, \texttt{a1}, \texttt{a8}, \texttt{a2}] (see Line 4 in Listing~\ref{list:8.10} above).

\section{\Slater rankings}
\label{sec:8.5}

The \Slater ranking rule is identical to \Kemeny 's, but it is working, instead, on the \Condorcet --\emph{median cut polarised}-- digraph \texttt{ccd} \citep{SLA-1961}. \Slater 's rule is also \emph{invariant} Under the codual transform and delivers again indifferently on \texttt{g} or \texttt{gcd} the following results:
\begin{lstlisting}[caption={Computing a \Slater ranking},label=list:8.11]   
>>> from linearOrders import SlaterRanking
>>> sl = SlaterRanking(gcd,orderLimit=9)
>>> sl.slaterRanking
  ['a5','a6','a4','a1','a3','a7','a8','a9','a2']
>>> corr = gcd.computeOrderCorrelation(sl.slaterRanking)
>>> sl.showCorrelation(corr)
  Correlation indexes:
   Extended Kendall tau       : +0.676
   Epistemic determination    :  0.230
   Bipolar-valued equivalence : +0.156
>>> len(sl.maximalRankings)
  7
\end{lstlisting}
We notice in Listing~\ref{list:8.11} Line 8 that the first \Slater ranking is a rather good fit ($+0.676$), slightly better apparently than the \NetFlows ranking result ($+0.638$). However, there are in fact 7 such potentially optimal \Slater rankings (see Line 12). The corresponding epistemic disjunction gives the partial ordering shown in Fig~\ref{fig:8.4}:
\begin{lstlisting}[caption={Computing the epistemic disjunction of optimal \Slater rankings},label=list:8.12]   
>>> slw = RankingsFusion(sl,sl.maximalRankings)
>>> slw.exportGraphViz(fileName='tutorialSlater')
 *---- exporting a dot file for GraphViz tools ----*
  Exporting to tutorialSlater.dot
  dot -Grankdir=TB -Tpng tutorialSlater.dot\
                   -o tutorialSlater.png
\end{lstlisting}
\begin{figure}[h]
\sidecaption[t]
\includegraphics[width=4cm]{Figures/tutorialSlater.png}
\caption{Epistemic disjunction of optimal \Slater rankings. What precise \Slater ranking result should we hence adopt?}
\label{fig:8.4}       % Give a unique label
\end{figure}
       
\Kemeny 's and \Slater 's ranking rules are furthermore computationally \emph{difficult} problems and effective ranking results are only computable for tiny outranking digraphs ($< 20$ objects). 

More efficient ranking heuristics, like the \Copeland and the \NetFlows rules, are therefore needed in practice. Let us finally, after these \emph{ranking-by-scoring} strategies, also present two popular \emph{ranking-by-choosing} strategies.

\section{\Kohler 's ranking-by-choosing rule}
\label{sec:8.6}

\Kohler 's \emph{ranking-by-choosing} rule\footnote{\citep{KOH-1978}} is formulated like this:
\begin{definition}[\Kohler 's \emph{ranking-by-choosing} rule]\label{def:kohler}
  
\noindent At step $i$ ($i$ goes from 1 to $n$) do the following:
\begin{enumerate}[leftmargin=0.5cm,rightmargin=0.5cm]
\item Compute for each row of the bipolar-valued \emph{strict} outranking relation table (see Listing \ref{list:8.1}) the smallest value;
\item Select the row where this minimum is maximal. Ties are resolved in lexicographic order;
\item Put the selected decision alternative at rank $i$;
\item Delete the corresponding row and column from the relation table and restart until the table is empty.
\end{enumerate}
\end{definition}
\begin{lstlisting}[caption={Computing a \Kohler ranking},label=list:8.13]   
>>> from linearOrders import KohlerRanking
>>> kocd = KohlerRanking(gcd)
>>> kocd.showRanking()
  ['a5','a7','a6','a3','a9','a8','a4','a1','a2']
>>> corr = gcd.computeOrdinalCorrelation(kocd)
>>> gcd.showCorrelation(corr)
  Correlation indexes:
    Extended Kendall tau       : +0.747
    Epistemic determination    :  0.230
    Bipolar-valued equivalence : +0.172
\end{lstlisting}

With this \emph{min-max} lexicographic ranking-by-choosing strategy, we find a correlation result ($+0.747$) that is until now clearly the nearest to an optimal \Kemeny ranking (see Listing \ref{list:8.8}). Only two adjacent pairs: (\texttt{a6}, \texttt{a7}) and (\texttt{a8}, \texttt{a9}) are actually inverted here. Notice that \Kohler 's ranking rule, contrary to the previously mentioned rules, is \textbf{not} invariant under the codual transform and requires to work on the \texttt{strict} outranking digraph \texttt{gcd} for a better correlation result.
\begin{lstlisting}
>>> ko = KohlerRanking(g)  
>>> corr = g.computeOrdinalCorrelation(ko)
>>> g.showCorrelation(corr)
  Correlation indexes:
   Crisp ordinal correlation  : +0.483
   Epistemic determination    :  0.230
   Bipolar-valued equivalence : +0.111
\end{lstlisting}

But \Kohler 's ranking has a \emph{dual} version, the prudent \emph{Arrow-Raynaud} ordering-by-choosing rule, where a corresponding \emph{max-min} strategy, when used on the \emph{non-strict} outranking digraph $g$, for ordering from \emph{last} to \emph{first} produces a similar ranking result \citep{ARR-1986}.

Noticing that the \NetFlows score of an alternative $x$ represents in fact a bipolar-valued characteristic of the assertion ``\emph{alternative x is ranked first}'', we may enhance \Kohler 's rule by replacing the simple \emph{min-max} strategy with an \emph{iterated} maximal \NetFlows score selection.

\begin{definition}[The iterated \NetFlows ranking-by-choosing rule]
  
\noindent At step $i$ ($i$ goes from 1 to $n$) do the following:
\begin{enumerate}[leftmargin=0.5cm,rightmargin=0.5cm]
\item Compute for each row of the bipolar-valued outranking relation table (see Listing \ref{list:8.1}) the corresponding \NetFlows score;
\item Select the row where this score is \emph{maximal}, ties being resolved by lexicographic order;
\item Put the corresponding decision alternative at rank $i$;
\item Delete the corresponding row and column from the relation table and restart until the table is empty.
\end{enumerate}
\end{definition}

The \texttt{IteratedNetFlowsRanking}\index{IteratedNetFlowsRanking@\texttt{IteratedNetFlowsRanking} class} class from the \texttt{linearOrders} module computes this ranking result. 
\begin{lstlisting}[caption={Ranking-by-choosing with iterated maximal \NetFlows scores},label=list:8.14]   
>>> from linearOrders import IteratedNetFlowsRanking  
>>> inf = IteratedNetFlowsRanking(g)
>>> inf
 *------- Digraph instance description ------*
   Instance class      : IteratedNetFlowsRanking
   Instance name       : rel_randomCBperftab_ranked
   Digraph Order       : 9
   Digraph Size        : 36
   Valuation domain    : [-1.00;1.00]
   Determinateness (%) : 100.00
   Attributes     : ['valuedRanks', 'valuedOrdering',
                     'iteratedNetFlowsRanking',
                     'iteratedNetFlowsOrdering',
                     'name', 'actions', 'order',
                     'valuationdomain', 'relation',
                     'gamma', 'notGamma']
>>> inf.iteratedNetFlowsRanking
  ['a5','a7','a6','a3','a4','a1','a8','a9','a2']
>>> corr = g.computeRankingCorrelation(\
...             inf.iteratedNetFlowsRanking)
>>> g.showCorrelation(corr)
  Correlation indexes:
    Crisp ordinal correlation  : +0.743
    Epistemic determination    :  0.230
    Bipolar-valued equivalence : +0.171
\end{lstlisting}

Like \Kohler 's rule, the iterated \NetFlows 's rule has also a dual \emph{ordering-by-choosing} version, where instead of choosing at each step $i$ the row with maximal \NetFlows score, we choose the row with the \emph{minimal} \NetFlows score. Both the ranking and ordering result are computed by the \texttt{IteratedNetFlowsRanking} class constructor (see Lines 12 and 13 in Listing~\ref{list:8.14}).
\begin{lstlisting}
>>> inf.iteratedNetFlowsOrdering
  ['a2','a9','a1','a4','a3','a8','a7','a6','a5']
>>> corr = g.computeOrderCorrelation(\
...                inf.iteratedNetFlowsOrdering)
>>> g.showCorrelation(corr)
  Correlation indexes:
    Crisp ordinal correlation  : +0.751
    Epistemic determination    : 0.230
    Bipolar-valued equivalence : +0.173
\end{lstlisting}
The iterated \NetFlows ranking and its dual, the iterated \NetFlows ordering, do not usually deliver both the same result. With our example outranking digraph $g$ for instance, it is the \emph{ordering-by-choosing} result who obtains a slightly better correlation with the given outranking digraph ($+0.751$), a result that is also slightly better than \Kohler 's original result ($+0.747$, see Listing \ref{list:8.13} Line 8).

With different \emph{ranking-by-choosing} and \emph{ordering-by-choosing} results, it may be useful to \emph{fuse} now, similar to what we have done before with \Kemeny 's and \Slater 's optimal rankings, both, the iterated \NetFlows ranking and ordering into a partial ranking. But we are hence back to the practical problem of what linear ranking should we eventually retain? 

Let us finally mention another interesting \emph{ranking-by-choosing} approach.

\section{The \RankedPairs ranking rule}
\label{sec:8.7}

Tideman's \index{Tideman@\emph{Tideman}} ranking-by-choosing heuristic, the \RankedPairs rule, working best this time on the non strict outranking digraph $g$, is based on a \emph{prudent incremental} construction of linear orders that avoids on the fly any cycling outrankings \citep{TID-1987}. The ranking rule may be formulated as follows:
\begin{definition}[The \RankedPairs ranking rule]
\begin{enumerate}[leftmargin=0.5cm,rightmargin=0.5cm]
 \item Rank the ordered pairs $(x,y)$ of alternatives in decreasing order of $r(x\, \succsim \,y) \,+\, r(y\, \not\succsim \,x)$;
 \item Consider the pairs in that order (ties are resolved by a lexicographic rule):
   \begin{itemize}
     \item if the next pair does not create a \emph{circuit} with the pairs already blocked, block this pair;
     \item if the next pair creates a \emph{circuit} with the already blocked pairs, skip it.
    \end{itemize}
\end{enumerate}
\end{definition}  
With our didactic outranking digraph $g$, we get the following result.\index{RankedPairsRanking@\texttt{RankedPairsRanking} class}
\begin{lstlisting}[caption={Computing a \RankedPairs ranking},label=list:8.15]   
>>> from linearOrders import RankedPairsRanking
>>> rp = RankedPairsRanking(g)
>>> rp.showRanking()
  ['a5','a6','a7','a3','a8','a9','a4','a1','a2']
\end{lstlisting}

The \RankedPairs rule renders in our example here luckily one of the two optimal \Kemeny ranking, as we may verify below.
 \begin{lstlisting}
>>> ke.maximalRankings
  [['a5','a6','a7','a3','a8','a9','a4','a1','a2'],
   ['a5','a6','a7','a3','a9','a4','a1','a8','a2']]
>>> corr = g.computeOrdinalCorrelation(rp)
>>> g.showCorrelation(corr)
  Correlation indexes:
    Extended Kendall tau       : +0.779
    Epistemic determination    :  0.230
    Bipolar-valued equivalence : +0.179
\end{lstlisting}

Similar to \Kohler 's rule, the \RankedPairs rule has also a prudent dual version, the \emph{Dias-Lamboray} \emph{ordering-by-choosing} rule, which produces, when working this time on the codual strict outranking digraph gcd, a similar ranking result (see \citet*{DIA-2010}).

Besides of not providing a unique linear ranking, the \emph{ranking-by-choosing} rules, as well as their duals, the \emph{ordering-by-choosing} rules, are unfortunately not scalable to outranking digraphs of larger orders ($> 100$). For such bigger outranking digraphs, with several hundred or thousands of alternatives, only the \Copeland and the \NetFlows \emph{ranking-by-scoring} rules, with a polynomial complexity of $\mathcal{O}(n^2)$, where $n$ is the order of the outranking digraph, remain in fact tractable. Furthermore, as computing the \Copeland and \NetFlows scores may be done separately per alternative, the latter ranking rules can right away be processed in parallel when multiprocessing resources are available.

\vspace{1cm}

%%%%%%% The chapter bibliography
%\normallatexbib
\clearpage
%\phantomsection
%\addcontentsline{toc}{section}{Chapter Bibliography}
\bibliographystyle{spbasic}
%\typeout{}
\bibliography{03-backMatters/reference}
%\input{02-mainMatters/08-chapterRankings.bbl}
 
