%%%%%%%%%%%%%%%%%%%%%part.tex%%%%%%%%%%%%%%%%%%%%%%%%%%%%%%%%%%
% 
% sample part title
%
% Use this file as a template for your own input.
%
%%%%%%%%%%%%%%%%%%%%%%%% Springer %%%%%%%%%%%%%%%%%%%%%%%%%%

\begin{partbacktext}
\part{Introduction to the \Digraph software resources}
\noindent The first part contains a set of tutorials introducing the \Digraph software collection of Python3 modules.

The basic idea of the \Digraph Python resources is to make easy python interactive sessions or write short Python scripts for computing all kind of results from a bipolar-valued digraph or graph. These include such features as maximal independent, maximal dominant or absorbent choices, rankings, outrankings, linear ordering, etc. Most of the available computing resources are meant to illustrate a Master Course on Algorithmic Decision Theory given at the University of Luxembourg in the context of its Master in Information and Computer Science (MICS).

The Python development of these computing resources offers the advantage of an easy to write and maintain OOP source code as expected from a performing scripting language without loosing on efficiency in execution times compared to compiled languages such as C++ or Java.
\end{partbacktext}