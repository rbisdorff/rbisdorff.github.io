%%%%%%%%%%%%%%%%%%%%%part.tex%%%%%%%%%%%%%%%%%%%%%%%%%%%%%%%%%%
% 
% sample part title
%
% Use this file as a template for your own input.
%
%%%%%%%%%%%%%%%%%%%%%%%% Springer %%%%%%%%%%%%%%%%%%%%%%%%%%

\begin{partbacktext}
\part{Advanced topics}
\noindent 

\begin{quotation}
``\emph{The rule for the combination of independent concurrent arguments takes a very simple form when expressed in terms of the intensity of belief ... It is this: Take the sum of all the feelings of belief which would be produced separately by all the arguments pro, subtract from that the similar sum for arguments con, and the remainder is the feeling of belief which ought to have the whole. This is a proceeding which men often resort to, under the name of balancing reasons.}'' -- C.S. Peirce, The probability of induction (1878)
\end{quotation}
\end{partbacktext}