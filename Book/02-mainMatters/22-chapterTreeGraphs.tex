\chapter{On tree graphs and graph forests}
\label{sec:22}

\abstract*{ The chapter specifically addresses working with tree graphs and graph forests. We illustrate how to generate and recognise random tree graphs and how to compute the centres of a tree and draw a rooted and oriented tree. Finally, algorithms for computing spanning trees and forests are presented.}

\abstract{ The chapter specifically addresses working with tree graphs and graph forests. We illustrate how to generate and recognise random tree graphs and how to compute the centres of a tree and draw a rooted and oriented tree. Finally, algorithms for computing spanning trees and forests are presented. }

\section{Generating random tree graphs}
\label{sec:22.1}

Using the \texttt{RandomTree} class\index{RandomTree@\texttt{RandomTree} class} from the \texttt{graphs} module, we may, for instance, generate a random tree graph with 9 vertices.
\begin{lstlisting}[caption={Generating a random tree graph},label=list:22.1]
>>> from graphs import RandomTree
>>> rt = RandomTree(order=9,seed=100)
>>> rt
  *------- Graph instance description ----*
   Instance class   : RandomTree
   Instance name    : randomTree
   Graph Order      : 9
   Graph Size       : 8
   Valuation domain : [-1.00; 1.00]
   Attributes   : ['name', 'order',
                   'vertices', 'valuationDomain',
                   'edges', 'prueferCode',
                   'size', 'gamma']
   *---- RandomTree specific data ----*
   Pruefer code  : ['v3','v8','v8','v3','v7','v6','v7']
>>> rt.exportGraphViz('tutRandomTree')
  *---- exporting a dot file for GraphViz tools --*
   Exporting to tutRandomTree.dot
   neato -Tpng tutRandomTree.dot -o tutRandomTree.png
\end{lstlisting}
\begin{figure}[ht]
\sidecaption[t]
\includegraphics[width=6cm]{Figures/22-1-tutRandomTree.pdf}
\caption[Random tree graph instance of order 9]{Random tree graph instance of order 9. One may distinguish vertices like \texttt{v1}, \texttt{v2}, \texttt{v4}, \texttt{v5} or \texttt{v9}  of degree 1, called the \emph{leaves} of the tree, and vertices like \texttt{v3}, \texttt{v6}, \texttt{v7} or \texttt{v8} of degree 2 or more, called the \emph{nodes} of the tree} 
\label{fig:22.1}       % Give a unique label
\end{figure}

In Fig.~\vref{fig:22.1} one may notice that a tree graph of order $n > 2$ always contains $n-1$ edges (see Lines 7 and 8 in List.~\vref{list:22.1}) and its structure is entirely characterised by a corresponding \Pruefer \emph{code}, i.e. a list of vertices keys of length $n-2$. See, for instance in Line 15 the code \texttt{['v3'}, \texttt{'v8',} \texttt{'v8',} \texttt{'v3',} \texttt{'v7',} \texttt{'v6',} \texttt{'v7']} corresponding to our sample tree graph \texttt{rt}.

Each position of the code indicates the parent of the remaining leaf with the smallest vertex label. Vertex \texttt{v3} is thus the parent of \texttt{v1} and we drop leaf \texttt{v1}, \texttt{v8} is now the parent of leaf \texttt{v2} and we drop \texttt{v2}, vertex \texttt{v8} is again the parent of leaf \texttt{v4} and we drop \texttt{v4}, vertex \texttt{v3} is the parent of leaf \texttt{v5} and we drop \texttt{v5}, \texttt{v7} is now the parent of leaf \texttt{v3} and we may drop \texttt{v3}, \texttt{v6} becomes the parent of leaf \texttt{v8} and we drop \texttt{v8}, \texttt{v7} becomes the parent of leaf \texttt{v6} and we may drop \texttt{v6}. The two eventually remaining vertices, \texttt{v7} and \texttt{v9}, give the last link in the reconstructed tree \citep{JPB-1991}.  

It is, as well possible to, first, generate a random \Pruefer code of length $n-2$ from a set of $n$ vertices  (see List.~\vref{list:22.2} Lines 1-9 ) and then, construct the corresponding tree of order $n$ by reversing the procedure illustrated above. The resulting tree graph is shown in Fig.~\vref{fig:22.2}.
\begin{lstlisting}[caption={Generating a tree graph with a random \Pruefer code.},label=list:22.2]
>>> verticesList = ['v1','v2','v3','v4','v5','v6','v7']
>>> n = len(verticesList)
>>> from random import seed, choice
>>> seed(101)
>>> code = []
>>> for k in range(n-2):
...     code.append( choice(verticesList) )
>>> print(code)
  ['v5', 'v7', 'v2', 'v5', 'v3']
>>> rt = RandomTree(prueferCode=code)
>>> rt
  *------- Graph instance description ------*
   Instance class   : RandomTree
   Instance name    : randomTree
   Graph Order      : 7
   Graph Size       : 6
   Valuation domain : [-1.00; 1.00]
   Attributes : ['name', 'order', 'vertices',
                 'valuationDomain', 'edges',
                 'prueferCode', 'size', 'gamma']
  *---- RandomTree specific data ----*
   Pruefer code  : ['v5', 'v7', 'v2', 'v5', 'v3']
>>> rt.exportGraphViz('tutPruefTree')
  *---- exporting a dot file for GraphViz tools ---------*
   Exporting to tutPruefTree.dot
   neato -Tpng tutPruefTree.dot -o tutPruefTree.png
\end{lstlisting}
\begin{figure}[ht]
\sidecaption[t]
\includegraphics[width=7cm]{Figures/22-2-tutPruefTree.pdf}
\caption[Tree graph generated with a random \Pruefer code]{Tree graph instance generated with a random \Pruefer code \texttt{['v5',} \texttt{'v7',} \texttt{'v2',} \texttt{'v5',} \texttt{'v3']}} 
\label{fig:22.2}       % Give a unique label
\end{figure}

Following from the bijection between a labelled tree and its \Pruefer code, we actually know that there exist $n^{n-2}$ different tree graphs with the same $n$ vertices.

Given a genuine graph, how can we recognise that it is in fact a tree instance ?

\section{Recognising tree graphs}
\label{sec:22.2}

Given a graph \texttt{g} of order $n$ and size $s$, the following 5 assertions A1, A2, A3, A4 and A5 are all equivalent (see \citep{JPB-1991}):
\begin{itemize}
\item [A1] \texttt{g} is a tree;
\item [A2] \texttt{g} is without (chordless) cycles and $n \,=\, s + 1$;
\item [A3] \texttt{g} is connected and $n \,=\, s + 1$;
\item [A4] Any two vertices of \texttt{g} are always connected by a unique path;
\item [A5] \texttt{g} is connected and dropping any single edge will always disconnect \texttt{g}.
\end{itemize}

Assertion A3, for instance, gives a simple test for recognising a tree graph. In case of a \emph{lazy evaluation} of the test in Listing~\vref{list:22.3} Line 3 below, it is opportune, from a computational complexity perspective, to first, check the order and size of the graph, before checking its potential connectedness. We provide the \texttt{isTree()} method\index{isTree@\texttt{isTree()}} for computing both these tests (see Line 8 in List.~\vref{list:22.3}).
\begin{lstlisting}[caption={Recognizing a tree graph.},label=list:22.3]
>>> from graphs import RandomGraph
>>> g = RandomGraph(order=8,edgeProbability=0.3,seed=62)
>>> if g.order == (g.size +1) and g.isConnected():
...     print('The graph is a tree ?', True)
... else:
...     print('The graph is a tree ?',False)   
  The graph is a tree ? True
>>> g.isTree()
  True  
\end{lstlisting}
The random graph of order $8$ and edge probability $30\%$, generated with seed $62$, is actually a tree graph instance, as confirmed by its graphviz drawing shown in Fig.~\vref{fig:22.3}.
\begin{lstlisting}
>>> g.exportGraphViz('test62')
  *---- exporting a dot file for GraphViz tools ---*
   Exporting to test62.dot
   fdp -Tpng test62.dot -o test62.png
\end{lstlisting}
\begin{figure}[ht]
\sidecaption[t]
\includegraphics[width=7cm]{Figures/22-3-test62.pdf}
\caption[Recognising a tree graph]{Recognising a tree graph. We may notice that vertex \texttt{v2} is actually situated in the \emph{centre} of the tree with a neighbourhood depth of 2} 
\label{fig:22.3}       % Give a unique label
\end{figure}

Yet, we still have to recover its corresponding \Pruefer code. Therefore, we may use the \texttt{tree2Pruefer()} method \index{tree2Pruefer@\texttt{tree2Pruefer()}} from the \texttt{TreeGraph} class. But, first, the instance class of graph \texttt{g} is changed to the \texttt{TreeGraph} type (see Line 2 in List.~\vref{list:22.4}). \index{TreeGraph@\texttt{TreeGraph()}}
\begin{lstlisting}[caption={Computing the \Pruefer code of a tree graph instance.},label=list:22.4]
>>> from graphs import TreeGraph
>>> g.__class__ = TreeGraph  
>>> g.tree2Pruefer()
  ['v6', 'v1', 'v2', 'v1', 'v2', 'v5']
\end{lstlisting}

In Fig.~\vref{fig:22.3}, we noticed that vertex \texttt{v2} is actually situated in the \emph{centre} of the tree with a neighbourhood depth of 2. Centres of a graph are the vertices with minimal neighbourhood depth. For finding such centre(s), the \texttt{Graph} class provides the \texttt{computeGraphCentres()} method.\index{computeGraphCentres@\texttt{computeGraphCentres()}} (see Lines 1-2 in List.~\vref{list:22.4}). Knowing now the centre of graph \texttt{g}, we may draw a correspondingly rooted and oriented tree with the \texttt{exportOrientedTreeGraphViz()} method from the \texttt{TreeGraph} class (see Fig.~\vref{fig:22.4}.\index{exportOrientedTreeGraphViz@\texttt{exportOrientedTreeGraphViz()}}  
\begin{lstlisting}[caption={Computing the centres of a tree and drawing a rooted and oriented tree.},label=list:22.5]
>>> g.computeGraphCentres()
  {'v2': 2}
>>> g.exportOrientedTreeGraphViz(\
...       fileName='rootedTree', root='v2')
  *---- exporting a dot file for GraphViz tools
   Exporting to rootedTree.dot
   dot -Grankdir=TB -Tpng rootedTree.dot -o rootedTree.png
\end{lstlisting}
\begin{figure}[ht]
\sidecaption[t]
\includegraphics[width=6cm]{Figures/22-4-rootedTree.pdf}
\caption{Drawing an oriented tree rooted at its centre} 
\label{fig:22.4}       % Give a unique label
\end{figure}

Let us now turn our attention toward a major application of tree graphs, namely \emph{spanning trees} and \emph{forests} related to graph traversals.

\section{Spanning trees and forests}
\label{sec:22.2}

With the \texttt{RandomSpanningTree} class\index{RandomSpanningTree@\texttt{RandomSpanningTree} class} we may generate, from a given connected graph \texttt{g} instance, \emph{uniform} random instances of a spanning tree by using \Wilson's algorithm (see List.~\vref{list:22.6} and Fig.~\vref{fig:22.5}).
\begin{lstlisting}[caption={Generating uniform random spanning trees.},label=list:22.6]
>>> from graphs import RandomGraph,\
...                    RandomSpanningTree
>>> g = RandomGraph(order=9,\
...                 edgeProbability=0.4,seed=100)
>>> spt = RandomSpanningTree(g)
>>> spt
  *------- Graph instance description ------*
   Instance class   : RandomSpanningTree
   Instance name    : randomGraph_randomSpanningTree
   Graph Order      : 9
   Graph Size       : 8
   Valuation domain : [-1.00; 1.00]
   Attributes       : ['name','vertices','order',
                       'valuationDomain',
                       'edges','size','gamma',
                       'dfs','date', 'dfsx',
                       'prueferCode']
  *---- RandomTree specific data ----*
   Pruefer code  : ['v7','v9','v5','v1','v8','v4','v9']
>>> spt.exportGraphViz(fileName='randomSpanningTree',\
...                    WithSpanningTree=True)
  *---- exporting a dot file for GraphViz tools -----*
   Exporting to randomSpanningTree.dot
   [['v1','v5','v6','v5','v1','v8','v9','v3','v9','v4',
      'v7','v2','v7','v4','v9','v8','v1']]
   neato -Tpng randomSpanningTree.dot\
         -o randomSpanningTree.png
\end{lstlisting}
\begin{figure}[ht]
\sidecaption[t]
\includegraphics[width=7cm]{Figures/22-5-randomSpanningTree.pdf}
\caption{Random spanning tree} 
\label{fig:22.5}       % Give a unique label
\end{figure}

\Wilson's algorithm only works for connected graphs \citep{WIL-1996}. More general, and in case of a not connected graph, the \texttt{RandomSpanningForest} class\index{RandomSpanningForest@\texttt{RandomSpanningForest} class} generates a not necessarily uniform random instance of a \emph{spanning forest} --one or more random tree graphs-- generated from a random \emph{depth first search} of the graph components' traversals (see List.~\vref{list:22.7} and Fig.~\vref{fig:22.6}).
\begin{lstlisting}[caption={Computing spanning forests over disconnected graphs. },label=list:22.7]
>>> g = RandomGraph(order=15,\
...                 edgeProbability=0.1,seed=140)
>>> g.computeComponents()
  [{'v12','v01','v13'}, {'v02','v06'},
   {'v08','v03','v07'}, {'v15','v11','v10','v04','v05'},
   {'v09','v14'}]
>>> spf = RandomSpanningForest(g,seed=100)
>>> spf.exportGraphViz(fileName='spanningForest',\
...                    WithSpanningTree=True)
  *---- exporting a dot file for GraphViz tools -----*
    Exporting to spanningForest.dot
    [['v03','v07','v08','v07','v03'],
     ['v13','v12','v13','v01','v13'],
     ['v02','v06','v02'],
     ['v15','v11','v04','v11','v15',
            'v10','v05','v10','v15'],
     ['v09','v14','v09']]
  neato -Tpng spanningForest.dot -o spanningForest.png
\end{lstlisting}
\begin{figure}[ht]
\sidecaption[t]
\includegraphics[width=7cm]{Figures/22-6-spanningForest.pdf}
\caption{Random spanning forest} 
\label{fig:22.6}       % Give a unique label
\end{figure}

\section{Maximum determined spanning forests}
\label{sec:22.3}

In case of valued graphs --supporting weighted edges-- we may finally construct a \emph{most determined} spanning tree (or forest if not connected) using \Kruskal's greedy minimum-spanning-tree algorithm on the dual valuation of the graph \citep{KRU-1956}\footnote{\Kruskal's algorithm is a minimum-spanning-tree algorithm which finds an edge of the least possible weight that connects any two trees in the forest.}.

In Listing~\vref{list:22.8} we generate, for instance, a randomly valued graph with five vertices and seven edges bipolar-valued in [-1.0; 1.0].\index{RandomValuationGraph@\texttt{RandomValuationGraph} class}
\begin{lstlisting}[caption={Generating randomly bipolar-valued graphs.},label=list:22.8]
>>> from graphs import RandomValuationGraph
>>> g = RandomValuationGraph(order=5,seed=2)
>>> g
  *------- Graph instance description ------*
   Instance class   : RandomValuationGraph
   Instance name    : randomGraph
   Graph Order      : 5
   Graph Size       : 7
   Valuation domain : [-1.00; 1.00]
   Attributes       : ['name', 'order',
                 'vertices', 'valuationDomain',
                 'edges', 'size', 'gamma']
\end{lstlisting}
To inspect the edges' actual weights, we first transform the graph into a corresponding digraph (see Line 1 in List.~\vref{list:22.9}) and use the \texttt{showRelationTable()} method (see Line 2) for printing its symmetric adjacency matrix. 
\begin{lstlisting}[caption={Symmetric relation table},label=list:22.9]
>>> dg = g.graph2Digraph()
>>> dg.showRelationTable()
  *---- Relation Table -----
    S   |  'v1'	 'v2'  'v3'  'v4'  'v5'	  
   -----|------------------------------
   'v1' |  0.00	 0.91  0.90 -0.89 -0.83	 
   'v2' |  0.91	 0.00  0.67  0.47  0.34	 
   'v3' |  0.90	 0.67  0.00 -0.38  0.21	 
   'v4' | -0.89	 0.47 -0.38  0.00  0.21	 
   'v5' | -0.83	 0.34  0.21  0.21  0.00	 
   Valuation domain: [-1.00;1.00]
\end{lstlisting}

To compute the most determined spanning tree or forest, we can use the \texttt{Best\-DeterminedSpanningForest} constructor.\index{BestDeterminedSpanningForest@\texttt{BestDeterminedSpanningForest} class}
\begin{lstlisting}[caption={Computing best determined spanning forests.},label=list:22.10]
>>> from graphs import\
                BestDeterminedSpanningForest
>>> mt = BestDeterminedSpanningForest(g)
>>> print(mt)
  *------- Graph instance description ------*
   Instance class   : BestDeterminedSpanningForest
   Instance name    : bdSpanningForest
   Graph Order      : 5
   Graph Size       : 4
   Valuation domain : [-1.00; 1.00]
   Attributes       : ['name','vertices','order',
                       'valuationDomain',
                       'edges','size','gamma',
                       'dfs','date',
                       'averageTreeDetermination']
  *---- best determined spanning tree  ----*
   Depth first search path(s) :
   [['v1','v2','v4','v2','v5','v2','v1','v3','v1']]
   Average determination(s) : [Decimal('0.655')]
\end{lstlisting}

The random graph \texttt{g} is connected and, hence, admits a single spanning tree of maximum mean determination = $(0.47 + 0.91 + 0.90 + 0.34)/4 = 0.655$ (see Lines 9, 6 and 10 in List.~\vref{list:22.8} and Fig.~\vref{fig:22.7}).
\begin{lstlisting}
>>> mt.exportGraphViz(\
...      fileName='bestDeterminedspanningTree',\
...      WithSpanningTree=True)
  *---- exporting a dot file for GraphViz tools ---*
   Exporting to spanningTree.dot
   [['v4','v2','v1','v3','v1','v2','v5','v2','v4']]
   neato -Tpng bestDeterminedSpanningTree.dot\
         -o bestDeterminedSpanningTree.png
\end{lstlisting}
\begin{figure}[ht]
\sidecaption[t]
\includegraphics[width=7cm]{Figures/22-7-bestDeterminedSpanningTree.pdf}
\caption{Best determined spanning tree} 
\label{fig:22.7}       % Give a unique label
\end{figure}

One may easily verify that all other potential spanning trees, including instead the edges \{\texttt{v3}, \texttt{v5}\} and/or \{\texttt{v4}, \texttt{v5}\}, will show a lower average determination.

\vspace{1cm}

Chapter~\ref{sec:23}, the last on undirected graphs, is devoted to different models of perfect graphs, namely split, interval, comparability and permutation graphs. 
 
%%%%%%% The chapter bibliography
%\normallatexbib
%\clearpage
%\phantomsection
%\addcontentsline{toc}{section}{Chapter Bibliography}
%\chapter{On tree graphs and graph forests}
\label{sec:22}

\abstract*{ The chapter specifically addresses working with tree graphs and graph forests. We illustrate how to generate and recognise random tree graphs and how to compute the centres of a tree and draw a rooted and oriented tree. Finally, algorithms for computing spanning trees and forests are presented.}

\abstract{ The chapter specifically addresses working with tree graphs and graph forests. We illustrate how to generate and recognise random tree graphs and how to compute the centres of a tree and draw a rooted and oriented tree. Finally, algorithms for computing spanning trees and forests are presented. }

\section{Generating random tree graphs}
\label{sec:22.1}

Using the \texttt{RandomTree} class\index{RandomTree@\texttt{RandomTree} class} from the \texttt{graphs} module, we may, for instance, generate a random tree graph with 9 vertices.
\begin{lstlisting}[caption={Generating a random tree graph},label=list:22.1]
>>> from graphs import RandomTree
>>> rt = RandomTree(order=9,seed=100)
>>> rt
  *------- Graph instance description ----*
   Instance class   : RandomTree
   Instance name    : randomTree
   Graph Order      : 9
   Graph Size       : 8
   Valuation domain : [-1.00; 1.00]
   Attributes   : ['name', 'order',
                   'vertices', 'valuationDomain',
                   'edges', 'prueferCode',
                   'size', 'gamma']
   *---- RandomTree specific data ----*
   Pruefer code  : ['v3','v8','v8','v3','v7','v6','v7']
>>> rt.exportGraphViz('tutRandomTree')
  *---- exporting a dot file for GraphViz tools --*
   Exporting to tutRandomTree.dot
   neato -Tpng tutRandomTree.dot -o tutRandomTree.png
\end{lstlisting}
\begin{figure}[ht]
\sidecaption[t]
\includegraphics[width=6cm]{Figures/22-1-tutRandomTree.pdf}
\caption[Random tree graph instance of order 9]{Random tree graph instance of order 9. One may distinguish vertices like \texttt{v1}, \texttt{v2}, \texttt{v4}, \texttt{v5} or \texttt{v9}  of degree 1, called the \emph{leaves} of the tree, and vertices like \texttt{v3}, \texttt{v6}, \texttt{v7} or \texttt{v8} of degree 2 or more, called the \emph{nodes} of the tree} 
\label{fig:22.1}       % Give a unique label
\end{figure}

In Fig.~\vref{fig:22.1} one may notice that a tree graph of order $n > 2$ always contains $n-1$ edges (see Lines 7 and 8 in List.~\vref{list:22.1}) and its structure is entirely characterised by a corresponding \Pruefer \emph{code}, i.e. a list of vertices keys of length $n-2$. See, for instance in Line 15 the code \texttt{['v3'}, \texttt{'v8',} \texttt{'v8',} \texttt{'v3',} \texttt{'v7',} \texttt{'v6',} \texttt{'v7']} corresponding to our sample tree graph \texttt{rt}.

Each position of the code indicates the parent of the remaining leaf with the smallest vertex label. Vertex \texttt{v3} is thus the parent of \texttt{v1} and we drop leaf \texttt{v1}, \texttt{v8} is now the parent of leaf \texttt{v2} and we drop \texttt{v2}, vertex \texttt{v8} is again the parent of leaf \texttt{v4} and we drop \texttt{v4}, vertex \texttt{v3} is the parent of leaf \texttt{v5} and we drop \texttt{v5}, \texttt{v7} is now the parent of leaf \texttt{v3} and we may drop \texttt{v3}, \texttt{v6} becomes the parent of leaf \texttt{v8} and we drop \texttt{v8}, \texttt{v7} becomes the parent of leaf \texttt{v6} and we may drop \texttt{v6}. The two eventually remaining vertices, \texttt{v7} and \texttt{v9}, give the last link in the reconstructed tree \citep{JPB-1991}.  

It is, as well possible to, first, generate a random \Pruefer code of length $n-2$ from a set of $n$ vertices  (see List.~\vref{list:22.2} Lines 1-9 ) and then, construct the corresponding tree of order $n$ by reversing the procedure illustrated above. The resulting tree graph is shown in Fig.~\vref{fig:22.2}.
\begin{lstlisting}[caption={Generating a tree graph with a random \Pruefer code.},label=list:22.2]
>>> verticesList = ['v1','v2','v3','v4','v5','v6','v7']
>>> n = len(verticesList)
>>> from random import seed, choice
>>> seed(101)
>>> code = []
>>> for k in range(n-2):
...     code.append( choice(verticesList) )
>>> print(code)
  ['v5', 'v7', 'v2', 'v5', 'v3']
>>> rt = RandomTree(prueferCode=code)
>>> rt
  *------- Graph instance description ------*
   Instance class   : RandomTree
   Instance name    : randomTree
   Graph Order      : 7
   Graph Size       : 6
   Valuation domain : [-1.00; 1.00]
   Attributes : ['name', 'order', 'vertices',
                 'valuationDomain', 'edges',
                 'prueferCode', 'size', 'gamma']
  *---- RandomTree specific data ----*
   Pruefer code  : ['v5', 'v7', 'v2', 'v5', 'v3']
>>> rt.exportGraphViz('tutPruefTree')
  *---- exporting a dot file for GraphViz tools ---------*
   Exporting to tutPruefTree.dot
   neato -Tpng tutPruefTree.dot -o tutPruefTree.png
\end{lstlisting}
\begin{figure}[ht]
\sidecaption[t]
\includegraphics[width=7cm]{Figures/22-2-tutPruefTree.pdf}
\caption[Tree graph generated with a random \Pruefer code]{Tree graph instance generated with a random \Pruefer code \texttt{['v5',} \texttt{'v7',} \texttt{'v2',} \texttt{'v5',} \texttt{'v3']}} 
\label{fig:22.2}       % Give a unique label
\end{figure}

Following from the bijection between a labelled tree and its \Pruefer code, we actually know that there exist $n^{n-2}$ different tree graphs with the same $n$ vertices.

Given a genuine graph, how can we recognise that it is in fact a tree instance ?

\section{Recognising tree graphs}
\label{sec:22.2}

Given a graph \texttt{g} of order $n$ and size $s$, the following 5 assertions A1, A2, A3, A4 and A5 are all equivalent (see \citep{JPB-1991}):
\begin{itemize}
\item [A1] \texttt{g} is a tree;
\item [A2] \texttt{g} is without (chordless) cycles and $n \,=\, s + 1$;
\item [A3] \texttt{g} is connected and $n \,=\, s + 1$;
\item [A4] Any two vertices of \texttt{g} are always connected by a unique path;
\item [A5] \texttt{g} is connected and dropping any single edge will always disconnect \texttt{g}.
\end{itemize}

Assertion A3, for instance, gives a simple test for recognising a tree graph. In case of a \emph{lazy evaluation} of the test in Listing~\vref{list:22.3} Line 3 below, it is opportune, from a computational complexity perspective, to first, check the order and size of the graph, before checking its potential connectedness. We provide the \texttt{isTree()} method\index{isTree@\texttt{isTree()}} for computing both these tests (see Line 8 in List.~\vref{list:22.3}).
\begin{lstlisting}[caption={Recognizing a tree graph.},label=list:22.3]
>>> from graphs import RandomGraph
>>> g = RandomGraph(order=8,edgeProbability=0.3,seed=62)
>>> if g.order == (g.size +1) and g.isConnected():
...     print('The graph is a tree ?', True)
... else:
...     print('The graph is a tree ?',False)   
  The graph is a tree ? True
>>> g.isTree()
  True  
\end{lstlisting}
The random graph of order $8$ and edge probability $30\%$, generated with seed $62$, is actually a tree graph instance, as confirmed by its graphviz drawing shown in Fig.~\vref{fig:22.3}.
\begin{lstlisting}
>>> g.exportGraphViz('test62')
  *---- exporting a dot file for GraphViz tools ---*
   Exporting to test62.dot
   fdp -Tpng test62.dot -o test62.png
\end{lstlisting}
\begin{figure}[ht]
\sidecaption[t]
\includegraphics[width=7cm]{Figures/22-3-test62.pdf}
\caption[Recognising a tree graph]{Recognising a tree graph. We may notice that vertex \texttt{v2} is actually situated in the \emph{centre} of the tree with a neighbourhood depth of 2} 
\label{fig:22.3}       % Give a unique label
\end{figure}

Yet, we still have to recover its corresponding \Pruefer code. Therefore, we may use the \texttt{tree2Pruefer()} method \index{tree2Pruefer@\texttt{tree2Pruefer()}} from the \texttt{TreeGraph} class. But, first, the instance class of graph \texttt{g} is changed to the \texttt{TreeGraph} type (see Line 2 in List.~\vref{list:22.4}). \index{TreeGraph@\texttt{TreeGraph()}}
\begin{lstlisting}[caption={Computing the \Pruefer code of a tree graph instance.},label=list:22.4]
>>> from graphs import TreeGraph
>>> g.__class__ = TreeGraph  
>>> g.tree2Pruefer()
  ['v6', 'v1', 'v2', 'v1', 'v2', 'v5']
\end{lstlisting}

In Fig.~\vref{fig:22.3}, we noticed that vertex \texttt{v2} is actually situated in the \emph{centre} of the tree with a neighbourhood depth of 2. Centres of a graph are the vertices with minimal neighbourhood depth. For finding such centre(s), the \texttt{Graph} class provides the \texttt{computeGraphCentres()} method.\index{computeGraphCentres@\texttt{computeGraphCentres()}} (see Lines 1-2 in List.~\vref{list:22.4}). Knowing now the centre of graph \texttt{g}, we may draw a correspondingly rooted and oriented tree with the \texttt{exportOrientedTreeGraphViz()} method from the \texttt{TreeGraph} class (see Fig.~\vref{fig:22.4}.\index{exportOrientedTreeGraphViz@\texttt{exportOrientedTreeGraphViz()}}  
\begin{lstlisting}[caption={Computing the centres of a tree and drawing a rooted and oriented tree.},label=list:22.5]
>>> g.computeGraphCentres()
  {'v2': 2}
>>> g.exportOrientedTreeGraphViz(\
...       fileName='rootedTree', root='v2')
  *---- exporting a dot file for GraphViz tools
   Exporting to rootedTree.dot
   dot -Grankdir=TB -Tpng rootedTree.dot -o rootedTree.png
\end{lstlisting}
\begin{figure}[ht]
\sidecaption[t]
\includegraphics[width=6cm]{Figures/22-4-rootedTree.pdf}
\caption{Drawing an oriented tree rooted at its centre} 
\label{fig:22.4}       % Give a unique label
\end{figure}

Let us now turn our attention toward a major application of tree graphs, namely \emph{spanning trees} and \emph{forests} related to graph traversals.

\section{Spanning trees and forests}
\label{sec:22.2}

With the \texttt{RandomSpanningTree} class\index{RandomSpanningTree@\texttt{RandomSpanningTree} class} we may generate, from a given connected graph \texttt{g} instance, \emph{uniform} random instances of a spanning tree by using \Wilson's algorithm (see List.~\vref{list:22.6} and Fig.~\vref{fig:22.5}).
\begin{lstlisting}[caption={Generating uniform random spanning trees.},label=list:22.6]
>>> from graphs import RandomGraph,\
...                    RandomSpanningTree
>>> g = RandomGraph(order=9,\
...                 edgeProbability=0.4,seed=100)
>>> spt = RandomSpanningTree(g)
>>> spt
  *------- Graph instance description ------*
   Instance class   : RandomSpanningTree
   Instance name    : randomGraph_randomSpanningTree
   Graph Order      : 9
   Graph Size       : 8
   Valuation domain : [-1.00; 1.00]
   Attributes       : ['name','vertices','order',
                       'valuationDomain',
                       'edges','size','gamma',
                       'dfs','date', 'dfsx',
                       'prueferCode']
  *---- RandomTree specific data ----*
   Pruefer code  : ['v7','v9','v5','v1','v8','v4','v9']
>>> spt.exportGraphViz(fileName='randomSpanningTree',\
...                    WithSpanningTree=True)
  *---- exporting a dot file for GraphViz tools -----*
   Exporting to randomSpanningTree.dot
   [['v1','v5','v6','v5','v1','v8','v9','v3','v9','v4',
      'v7','v2','v7','v4','v9','v8','v1']]
   neato -Tpng randomSpanningTree.dot\
         -o randomSpanningTree.png
\end{lstlisting}
\begin{figure}[ht]
\sidecaption[t]
\includegraphics[width=7cm]{Figures/22-5-randomSpanningTree.pdf}
\caption{Random spanning tree} 
\label{fig:22.5}       % Give a unique label
\end{figure}

\Wilson's algorithm only works for connected graphs \citep{WIL-1996}. More general, and in case of a not connected graph, the \texttt{RandomSpanningForest} class\index{RandomSpanningForest@\texttt{RandomSpanningForest} class} generates a not necessarily uniform random instance of a \emph{spanning forest} --one or more random tree graphs-- generated from a random \emph{depth first search} of the graph components' traversals (see List.~\vref{list:22.7} and Fig.~\vref{fig:22.6}).
\begin{lstlisting}[caption={Computing spanning forests over disconnected graphs. },label=list:22.7]
>>> g = RandomGraph(order=15,\
...                 edgeProbability=0.1,seed=140)
>>> g.computeComponents()
  [{'v12','v01','v13'}, {'v02','v06'},
   {'v08','v03','v07'}, {'v15','v11','v10','v04','v05'},
   {'v09','v14'}]
>>> spf = RandomSpanningForest(g,seed=100)
>>> spf.exportGraphViz(fileName='spanningForest',\
...                    WithSpanningTree=True)
  *---- exporting a dot file for GraphViz tools -----*
    Exporting to spanningForest.dot
    [['v03','v07','v08','v07','v03'],
     ['v13','v12','v13','v01','v13'],
     ['v02','v06','v02'],
     ['v15','v11','v04','v11','v15',
            'v10','v05','v10','v15'],
     ['v09','v14','v09']]
  neato -Tpng spanningForest.dot -o spanningForest.png
\end{lstlisting}
\begin{figure}[ht]
\sidecaption[t]
\includegraphics[width=7cm]{Figures/22-6-spanningForest.pdf}
\caption{Random spanning forest} 
\label{fig:22.6}       % Give a unique label
\end{figure}

\section{Maximum determined spanning forests}
\label{sec:22.3}

In case of valued graphs --supporting weighted edges-- we may finally construct a \emph{most determined} spanning tree (or forest if not connected) using \Kruskal's greedy minimum-spanning-tree algorithm on the dual valuation of the graph \citep{KRU-1956}\footnote{\Kruskal's algorithm is a minimum-spanning-tree algorithm which finds an edge of the least possible weight that connects any two trees in the forest.}.

In Listing~\vref{list:22.8} we generate, for instance, a randomly valued graph with five vertices and seven edges bipolar-valued in [-1.0; 1.0].\index{RandomValuationGraph@\texttt{RandomValuationGraph} class}
\begin{lstlisting}[caption={Generating randomly bipolar-valued graphs.},label=list:22.8]
>>> from graphs import RandomValuationGraph
>>> g = RandomValuationGraph(order=5,seed=2)
>>> g
  *------- Graph instance description ------*
   Instance class   : RandomValuationGraph
   Instance name    : randomGraph
   Graph Order      : 5
   Graph Size       : 7
   Valuation domain : [-1.00; 1.00]
   Attributes       : ['name', 'order',
                 'vertices', 'valuationDomain',
                 'edges', 'size', 'gamma']
\end{lstlisting}
To inspect the edges' actual weights, we first transform the graph into a corresponding digraph (see Line 1 in List.~\vref{list:22.9}) and use the \texttt{showRelationTable()} method (see Line 2) for printing its symmetric adjacency matrix. 
\begin{lstlisting}[caption={Symmetric relation table},label=list:22.9]
>>> dg = g.graph2Digraph()
>>> dg.showRelationTable()
  *---- Relation Table -----
    S   |  'v1'	 'v2'  'v3'  'v4'  'v5'	  
   -----|------------------------------
   'v1' |  0.00	 0.91  0.90 -0.89 -0.83	 
   'v2' |  0.91	 0.00  0.67  0.47  0.34	 
   'v3' |  0.90	 0.67  0.00 -0.38  0.21	 
   'v4' | -0.89	 0.47 -0.38  0.00  0.21	 
   'v5' | -0.83	 0.34  0.21  0.21  0.00	 
   Valuation domain: [-1.00;1.00]
\end{lstlisting}

To compute the most determined spanning tree or forest, we can use the \texttt{Best\-DeterminedSpanningForest} constructor.\index{BestDeterminedSpanningForest@\texttt{BestDeterminedSpanningForest} class}
\begin{lstlisting}[caption={Computing best determined spanning forests.},label=list:22.10]
>>> from graphs import\
                BestDeterminedSpanningForest
>>> mt = BestDeterminedSpanningForest(g)
>>> print(mt)
  *------- Graph instance description ------*
   Instance class   : BestDeterminedSpanningForest
   Instance name    : bdSpanningForest
   Graph Order      : 5
   Graph Size       : 4
   Valuation domain : [-1.00; 1.00]
   Attributes       : ['name','vertices','order',
                       'valuationDomain',
                       'edges','size','gamma',
                       'dfs','date',
                       'averageTreeDetermination']
  *---- best determined spanning tree  ----*
   Depth first search path(s) :
   [['v1','v2','v4','v2','v5','v2','v1','v3','v1']]
   Average determination(s) : [Decimal('0.655')]
\end{lstlisting}

The random graph \texttt{g} is connected and, hence, admits a single spanning tree of maximum mean determination = $(0.47 + 0.91 + 0.90 + 0.34)/4 = 0.655$ (see Lines 9, 6 and 10 in List.~\vref{list:22.8} and Fig.~\vref{fig:22.7}).
\begin{lstlisting}
>>> mt.exportGraphViz(\
...      fileName='bestDeterminedspanningTree',\
...      WithSpanningTree=True)
  *---- exporting a dot file for GraphViz tools ---*
   Exporting to spanningTree.dot
   [['v4','v2','v1','v3','v1','v2','v5','v2','v4']]
   neato -Tpng bestDeterminedSpanningTree.dot\
         -o bestDeterminedSpanningTree.png
\end{lstlisting}
\begin{figure}[ht]
\sidecaption[t]
\includegraphics[width=7cm]{Figures/22-7-bestDeterminedSpanningTree.pdf}
\caption{Best determined spanning tree} 
\label{fig:22.7}       % Give a unique label
\end{figure}

One may easily verify that all other potential spanning trees, including instead the edges \{\texttt{v3}, \texttt{v5}\} and/or \{\texttt{v4}, \texttt{v5}\}, will show a lower average determination.

\vspace{1cm}

Chapter~\ref{sec:23}, the last on undirected graphs, is devoted to different models of perfect graphs, namely split, interval, comparability and permutation graphs. 
 
%%%%%%% The chapter bibliography
%\normallatexbib
%\clearpage
%\phantomsection
%\addcontentsline{toc}{section}{Chapter Bibliography}
%\chapter{On tree graphs and graph forests}
\label{sec:22}

\abstract*{ The chapter specifically addresses working with tree graphs and graph forests. We illustrate how to generate and recognise random tree graphs and how to compute the centres of a tree and draw a rooted and oriented tree. Finally, algorithms for computing spanning trees and forests are presented.}

\abstract{ The chapter specifically addresses working with tree graphs and graph forests. We illustrate how to generate and recognise random tree graphs and how to compute the centres of a tree and draw a rooted and oriented tree. Finally, algorithms for computing spanning trees and forests are presented. }

\section{Generating random tree graphs}
\label{sec:22.1}

Using the \texttt{RandomTree} class\index{RandomTree@\texttt{RandomTree} class} from the \texttt{graphs} module, we may, for instance, generate a random tree graph with 9 vertices.
\begin{lstlisting}[caption={Generating a random tree graph},label=list:22.1]
>>> from graphs import RandomTree
>>> rt = RandomTree(order=9,seed=100)
>>> rt
  *------- Graph instance description ----*
   Instance class   : RandomTree
   Instance name    : randomTree
   Graph Order      : 9
   Graph Size       : 8
   Valuation domain : [-1.00; 1.00]
   Attributes   : ['name', 'order',
                   'vertices', 'valuationDomain',
                   'edges', 'prueferCode',
                   'size', 'gamma']
   *---- RandomTree specific data ----*
   Pruefer code  : ['v3','v8','v8','v3','v7','v6','v7']
>>> rt.exportGraphViz('tutRandomTree')
  *---- exporting a dot file for GraphViz tools --*
   Exporting to tutRandomTree.dot
   neato -Tpng tutRandomTree.dot -o tutRandomTree.png
\end{lstlisting}
\begin{figure}[ht]
\sidecaption[t]
\includegraphics[width=6cm]{Figures/22-1-tutRandomTree.pdf}
\caption[Random tree graph instance of order 9]{Random tree graph instance of order 9. One may distinguish vertices like \texttt{v1}, \texttt{v2}, \texttt{v4}, \texttt{v5} or \texttt{v9}  of degree 1, called the \emph{leaves} of the tree, and vertices like \texttt{v3}, \texttt{v6}, \texttt{v7} or \texttt{v8} of degree 2 or more, called the \emph{nodes} of the tree} 
\label{fig:22.1}       % Give a unique label
\end{figure}

In Fig.~\vref{fig:22.1} one may notice that a tree graph of order $n > 2$ always contains $n-1$ edges (see Lines 7 and 8 in List.~\vref{list:22.1}) and its structure is entirely characterised by a corresponding \Pruefer \emph{code}, i.e. a list of vertices keys of length $n-2$. See, for instance in Line 15 the code \texttt{['v3'}, \texttt{'v8',} \texttt{'v8',} \texttt{'v3',} \texttt{'v7',} \texttt{'v6',} \texttt{'v7']} corresponding to our sample tree graph \texttt{rt}.

Each position of the code indicates the parent of the remaining leaf with the smallest vertex label. Vertex \texttt{v3} is thus the parent of \texttt{v1} and we drop leaf \texttt{v1}, \texttt{v8} is now the parent of leaf \texttt{v2} and we drop \texttt{v2}, vertex \texttt{v8} is again the parent of leaf \texttt{v4} and we drop \texttt{v4}, vertex \texttt{v3} is the parent of leaf \texttt{v5} and we drop \texttt{v5}, \texttt{v7} is now the parent of leaf \texttt{v3} and we may drop \texttt{v3}, \texttt{v6} becomes the parent of leaf \texttt{v8} and we drop \texttt{v8}, \texttt{v7} becomes the parent of leaf \texttt{v6} and we may drop \texttt{v6}. The two eventually remaining vertices, \texttt{v7} and \texttt{v9}, give the last link in the reconstructed tree \citep{JPB-1991}.  

It is, as well possible to, first, generate a random \Pruefer code of length $n-2$ from a set of $n$ vertices  (see List.~\vref{list:22.2} Lines 1-9 ) and then, construct the corresponding tree of order $n$ by reversing the procedure illustrated above. The resulting tree graph is shown in Fig.~\vref{fig:22.2}.
\begin{lstlisting}[caption={Generating a tree graph with a random \Pruefer code.},label=list:22.2]
>>> verticesList = ['v1','v2','v3','v4','v5','v6','v7']
>>> n = len(verticesList)
>>> from random import seed, choice
>>> seed(101)
>>> code = []
>>> for k in range(n-2):
...     code.append( choice(verticesList) )
>>> print(code)
  ['v5', 'v7', 'v2', 'v5', 'v3']
>>> rt = RandomTree(prueferCode=code)
>>> rt
  *------- Graph instance description ------*
   Instance class   : RandomTree
   Instance name    : randomTree
   Graph Order      : 7
   Graph Size       : 6
   Valuation domain : [-1.00; 1.00]
   Attributes : ['name', 'order', 'vertices',
                 'valuationDomain', 'edges',
                 'prueferCode', 'size', 'gamma']
  *---- RandomTree specific data ----*
   Pruefer code  : ['v5', 'v7', 'v2', 'v5', 'v3']
>>> rt.exportGraphViz('tutPruefTree')
  *---- exporting a dot file for GraphViz tools ---------*
   Exporting to tutPruefTree.dot
   neato -Tpng tutPruefTree.dot -o tutPruefTree.png
\end{lstlisting}
\begin{figure}[ht]
\sidecaption[t]
\includegraphics[width=7cm]{Figures/22-2-tutPruefTree.pdf}
\caption[Tree graph generated with a random \Pruefer code]{Tree graph instance generated with a random \Pruefer code \texttt{['v5',} \texttt{'v7',} \texttt{'v2',} \texttt{'v5',} \texttt{'v3']}} 
\label{fig:22.2}       % Give a unique label
\end{figure}

Following from the bijection between a labelled tree and its \Pruefer code, we actually know that there exist $n^{n-2}$ different tree graphs with the same $n$ vertices.

Given a genuine graph, how can we recognise that it is in fact a tree instance ?

\section{Recognising tree graphs}
\label{sec:22.2}

Given a graph \texttt{g} of order $n$ and size $s$, the following 5 assertions A1, A2, A3, A4 and A5 are all equivalent (see \citep{JPB-1991}):
\begin{itemize}
\item [A1] \texttt{g} is a tree;
\item [A2] \texttt{g} is without (chordless) cycles and $n \,=\, s + 1$;
\item [A3] \texttt{g} is connected and $n \,=\, s + 1$;
\item [A4] Any two vertices of \texttt{g} are always connected by a unique path;
\item [A5] \texttt{g} is connected and dropping any single edge will always disconnect \texttt{g}.
\end{itemize}

Assertion A3, for instance, gives a simple test for recognising a tree graph. In case of a \emph{lazy evaluation} of the test in Listing~\vref{list:22.3} Line 3 below, it is opportune, from a computational complexity perspective, to first, check the order and size of the graph, before checking its potential connectedness. We provide the \texttt{isTree()} method\index{isTree@\texttt{isTree()}} for computing both these tests (see Line 8 in List.~\vref{list:22.3}).
\begin{lstlisting}[caption={Recognizing a tree graph.},label=list:22.3]
>>> from graphs import RandomGraph
>>> g = RandomGraph(order=8,edgeProbability=0.3,seed=62)
>>> if g.order == (g.size +1) and g.isConnected():
...     print('The graph is a tree ?', True)
... else:
...     print('The graph is a tree ?',False)   
  The graph is a tree ? True
>>> g.isTree()
  True  
\end{lstlisting}
The random graph of order $8$ and edge probability $30\%$, generated with seed $62$, is actually a tree graph instance, as confirmed by its graphviz drawing shown in Fig.~\vref{fig:22.3}.
\begin{lstlisting}
>>> g.exportGraphViz('test62')
  *---- exporting a dot file for GraphViz tools ---*
   Exporting to test62.dot
   fdp -Tpng test62.dot -o test62.png
\end{lstlisting}
\begin{figure}[ht]
\sidecaption[t]
\includegraphics[width=7cm]{Figures/22-3-test62.pdf}
\caption[Recognising a tree graph]{Recognising a tree graph. We may notice that vertex \texttt{v2} is actually situated in the \emph{centre} of the tree with a neighbourhood depth of 2} 
\label{fig:22.3}       % Give a unique label
\end{figure}

Yet, we still have to recover its corresponding \Pruefer code. Therefore, we may use the \texttt{tree2Pruefer()} method \index{tree2Pruefer@\texttt{tree2Pruefer()}} from the \texttt{TreeGraph} class. But, first, the instance class of graph \texttt{g} is changed to the \texttt{TreeGraph} type (see Line 2 in List.~\vref{list:22.4}). \index{TreeGraph@\texttt{TreeGraph()}}
\begin{lstlisting}[caption={Computing the \Pruefer code of a tree graph instance.},label=list:22.4]
>>> from graphs import TreeGraph
>>> g.__class__ = TreeGraph  
>>> g.tree2Pruefer()
  ['v6', 'v1', 'v2', 'v1', 'v2', 'v5']
\end{lstlisting}

In Fig.~\vref{fig:22.3}, we noticed that vertex \texttt{v2} is actually situated in the \emph{centre} of the tree with a neighbourhood depth of 2. Centres of a graph are the vertices with minimal neighbourhood depth. For finding such centre(s), the \texttt{Graph} class provides the \texttt{computeGraphCentres()} method.\index{computeGraphCentres@\texttt{computeGraphCentres()}} (see Lines 1-2 in List.~\vref{list:22.4}). Knowing now the centre of graph \texttt{g}, we may draw a correspondingly rooted and oriented tree with the \texttt{exportOrientedTreeGraphViz()} method from the \texttt{TreeGraph} class (see Fig.~\vref{fig:22.4}.\index{exportOrientedTreeGraphViz@\texttt{exportOrientedTreeGraphViz()}}  
\begin{lstlisting}[caption={Computing the centres of a tree and drawing a rooted and oriented tree.},label=list:22.5]
>>> g.computeGraphCentres()
  {'v2': 2}
>>> g.exportOrientedTreeGraphViz(\
...       fileName='rootedTree', root='v2')
  *---- exporting a dot file for GraphViz tools
   Exporting to rootedTree.dot
   dot -Grankdir=TB -Tpng rootedTree.dot -o rootedTree.png
\end{lstlisting}
\begin{figure}[ht]
\sidecaption[t]
\includegraphics[width=6cm]{Figures/22-4-rootedTree.pdf}
\caption{Drawing an oriented tree rooted at its centre} 
\label{fig:22.4}       % Give a unique label
\end{figure}

Let us now turn our attention toward a major application of tree graphs, namely \emph{spanning trees} and \emph{forests} related to graph traversals.

\section{Spanning trees and forests}
\label{sec:22.2}

With the \texttt{RandomSpanningTree} class\index{RandomSpanningTree@\texttt{RandomSpanningTree} class} we may generate, from a given connected graph \texttt{g} instance, \emph{uniform} random instances of a spanning tree by using \Wilson's algorithm (see List.~\vref{list:22.6} and Fig.~\vref{fig:22.5}).
\begin{lstlisting}[caption={Generating uniform random spanning trees.},label=list:22.6]
>>> from graphs import RandomGraph,\
...                    RandomSpanningTree
>>> g = RandomGraph(order=9,\
...                 edgeProbability=0.4,seed=100)
>>> spt = RandomSpanningTree(g)
>>> spt
  *------- Graph instance description ------*
   Instance class   : RandomSpanningTree
   Instance name    : randomGraph_randomSpanningTree
   Graph Order      : 9
   Graph Size       : 8
   Valuation domain : [-1.00; 1.00]
   Attributes       : ['name','vertices','order',
                       'valuationDomain',
                       'edges','size','gamma',
                       'dfs','date', 'dfsx',
                       'prueferCode']
  *---- RandomTree specific data ----*
   Pruefer code  : ['v7','v9','v5','v1','v8','v4','v9']
>>> spt.exportGraphViz(fileName='randomSpanningTree',\
...                    WithSpanningTree=True)
  *---- exporting a dot file for GraphViz tools -----*
   Exporting to randomSpanningTree.dot
   [['v1','v5','v6','v5','v1','v8','v9','v3','v9','v4',
      'v7','v2','v7','v4','v9','v8','v1']]
   neato -Tpng randomSpanningTree.dot\
         -o randomSpanningTree.png
\end{lstlisting}
\begin{figure}[ht]
\sidecaption[t]
\includegraphics[width=7cm]{Figures/22-5-randomSpanningTree.pdf}
\caption{Random spanning tree} 
\label{fig:22.5}       % Give a unique label
\end{figure}

\Wilson's algorithm only works for connected graphs \citep{WIL-1996}. More general, and in case of a not connected graph, the \texttt{RandomSpanningForest} class\index{RandomSpanningForest@\texttt{RandomSpanningForest} class} generates a not necessarily uniform random instance of a \emph{spanning forest} --one or more random tree graphs-- generated from a random \emph{depth first search} of the graph components' traversals (see List.~\vref{list:22.7} and Fig.~\vref{fig:22.6}).
\begin{lstlisting}[caption={Computing spanning forests over disconnected graphs. },label=list:22.7]
>>> g = RandomGraph(order=15,\
...                 edgeProbability=0.1,seed=140)
>>> g.computeComponents()
  [{'v12','v01','v13'}, {'v02','v06'},
   {'v08','v03','v07'}, {'v15','v11','v10','v04','v05'},
   {'v09','v14'}]
>>> spf = RandomSpanningForest(g,seed=100)
>>> spf.exportGraphViz(fileName='spanningForest',\
...                    WithSpanningTree=True)
  *---- exporting a dot file for GraphViz tools -----*
    Exporting to spanningForest.dot
    [['v03','v07','v08','v07','v03'],
     ['v13','v12','v13','v01','v13'],
     ['v02','v06','v02'],
     ['v15','v11','v04','v11','v15',
            'v10','v05','v10','v15'],
     ['v09','v14','v09']]
  neato -Tpng spanningForest.dot -o spanningForest.png
\end{lstlisting}
\begin{figure}[ht]
\sidecaption[t]
\includegraphics[width=7cm]{Figures/22-6-spanningForest.pdf}
\caption{Random spanning forest} 
\label{fig:22.6}       % Give a unique label
\end{figure}

\section{Maximum determined spanning forests}
\label{sec:22.3}

In case of valued graphs --supporting weighted edges-- we may finally construct a \emph{most determined} spanning tree (or forest if not connected) using \Kruskal's greedy minimum-spanning-tree algorithm on the dual valuation of the graph \citep{KRU-1956}\footnote{\Kruskal's algorithm is a minimum-spanning-tree algorithm which finds an edge of the least possible weight that connects any two trees in the forest.}.

In Listing~\vref{list:22.8} we generate, for instance, a randomly valued graph with five vertices and seven edges bipolar-valued in [-1.0; 1.0].\index{RandomValuationGraph@\texttt{RandomValuationGraph} class}
\begin{lstlisting}[caption={Generating randomly bipolar-valued graphs.},label=list:22.8]
>>> from graphs import RandomValuationGraph
>>> g = RandomValuationGraph(order=5,seed=2)
>>> g
  *------- Graph instance description ------*
   Instance class   : RandomValuationGraph
   Instance name    : randomGraph
   Graph Order      : 5
   Graph Size       : 7
   Valuation domain : [-1.00; 1.00]
   Attributes       : ['name', 'order',
                 'vertices', 'valuationDomain',
                 'edges', 'size', 'gamma']
\end{lstlisting}
To inspect the edges' actual weights, we first transform the graph into a corresponding digraph (see Line 1 in List.~\vref{list:22.9}) and use the \texttt{showRelationTable()} method (see Line 2) for printing its symmetric adjacency matrix. 
\begin{lstlisting}[caption={Symmetric relation table},label=list:22.9]
>>> dg = g.graph2Digraph()
>>> dg.showRelationTable()
  *---- Relation Table -----
    S   |  'v1'	 'v2'  'v3'  'v4'  'v5'	  
   -----|------------------------------
   'v1' |  0.00	 0.91  0.90 -0.89 -0.83	 
   'v2' |  0.91	 0.00  0.67  0.47  0.34	 
   'v3' |  0.90	 0.67  0.00 -0.38  0.21	 
   'v4' | -0.89	 0.47 -0.38  0.00  0.21	 
   'v5' | -0.83	 0.34  0.21  0.21  0.00	 
   Valuation domain: [-1.00;1.00]
\end{lstlisting}

To compute the most determined spanning tree or forest, we can use the \texttt{Best\-DeterminedSpanningForest} constructor.\index{BestDeterminedSpanningForest@\texttt{BestDeterminedSpanningForest} class}
\begin{lstlisting}[caption={Computing best determined spanning forests.},label=list:22.10]
>>> from graphs import\
                BestDeterminedSpanningForest
>>> mt = BestDeterminedSpanningForest(g)
>>> print(mt)
  *------- Graph instance description ------*
   Instance class   : BestDeterminedSpanningForest
   Instance name    : bdSpanningForest
   Graph Order      : 5
   Graph Size       : 4
   Valuation domain : [-1.00; 1.00]
   Attributes       : ['name','vertices','order',
                       'valuationDomain',
                       'edges','size','gamma',
                       'dfs','date',
                       'averageTreeDetermination']
  *---- best determined spanning tree  ----*
   Depth first search path(s) :
   [['v1','v2','v4','v2','v5','v2','v1','v3','v1']]
   Average determination(s) : [Decimal('0.655')]
\end{lstlisting}

The random graph \texttt{g} is connected and, hence, admits a single spanning tree of maximum mean determination = $(0.47 + 0.91 + 0.90 + 0.34)/4 = 0.655$ (see Lines 9, 6 and 10 in List.~\vref{list:22.8} and Fig.~\vref{fig:22.7}).
\begin{lstlisting}
>>> mt.exportGraphViz(\
...      fileName='bestDeterminedspanningTree',\
...      WithSpanningTree=True)
  *---- exporting a dot file for GraphViz tools ---*
   Exporting to spanningTree.dot
   [['v4','v2','v1','v3','v1','v2','v5','v2','v4']]
   neato -Tpng bestDeterminedSpanningTree.dot\
         -o bestDeterminedSpanningTree.png
\end{lstlisting}
\begin{figure}[ht]
\sidecaption[t]
\includegraphics[width=7cm]{Figures/22-7-bestDeterminedSpanningTree.pdf}
\caption{Best determined spanning tree} 
\label{fig:22.7}       % Give a unique label
\end{figure}

One may easily verify that all other potential spanning trees, including instead the edges \{\texttt{v3}, \texttt{v5}\} and/or \{\texttt{v4}, \texttt{v5}\}, will show a lower average determination.

\vspace{1cm}

Chapter~\ref{sec:23}, the last on undirected graphs, is devoted to different models of perfect graphs, namely split, interval, comparability and permutation graphs. 
 
%%%%%%% The chapter bibliography
%\normallatexbib
%\clearpage
%\phantomsection
%\addcontentsline{toc}{section}{Chapter Bibliography}
%\chapter{On tree graphs and graph forests}
\label{sec:22}

\abstract*{ The chapter specifically addresses working with tree graphs and graph forests. We illustrate how to generate and recognise random tree graphs and how to compute the centres of a tree and draw a rooted and oriented tree. Finally, algorithms for computing spanning trees and forests are presented.}

\abstract{ The chapter specifically addresses working with tree graphs and graph forests. We illustrate how to generate and recognise random tree graphs and how to compute the centres of a tree and draw a rooted and oriented tree. Finally, algorithms for computing spanning trees and forests are presented. }

\section{Generating random tree graphs}
\label{sec:22.1}

Using the \texttt{RandomTree} class\index{RandomTree@\texttt{RandomTree} class} from the \texttt{graphs} module, we may, for instance, generate a random tree graph with 9 vertices.
\begin{lstlisting}[caption={Generating a random tree graph},label=list:22.1]
>>> from graphs import RandomTree
>>> rt = RandomTree(order=9,seed=100)
>>> rt
  *------- Graph instance description ----*
   Instance class   : RandomTree
   Instance name    : randomTree
   Graph Order      : 9
   Graph Size       : 8
   Valuation domain : [-1.00; 1.00]
   Attributes   : ['name', 'order',
                   'vertices', 'valuationDomain',
                   'edges', 'prueferCode',
                   'size', 'gamma']
   *---- RandomTree specific data ----*
   Pruefer code  : ['v3','v8','v8','v3','v7','v6','v7']
>>> rt.exportGraphViz('tutRandomTree')
  *---- exporting a dot file for GraphViz tools --*
   Exporting to tutRandomTree.dot
   neato -Tpng tutRandomTree.dot -o tutRandomTree.png
\end{lstlisting}
\begin{figure}[ht]
\sidecaption[t]
\includegraphics[width=6cm]{Figures/22-1-tutRandomTree.pdf}
\caption[Random tree graph instance of order 9]{Random tree graph instance of order 9. One may distinguish vertices like \texttt{v1}, \texttt{v2}, \texttt{v4}, \texttt{v5} or \texttt{v9}  of degree 1, called the \emph{leaves} of the tree, and vertices like \texttt{v3}, \texttt{v6}, \texttt{v7} or \texttt{v8} of degree 2 or more, called the \emph{nodes} of the tree} 
\label{fig:22.1}       % Give a unique label
\end{figure}

In Fig.~\vref{fig:22.1} one may notice that a tree graph of order $n > 2$ always contains $n-1$ edges (see Lines 7 and 8 in List.~\vref{list:22.1}) and its structure is entirely characterised by a corresponding \Pruefer \emph{code}, i.e. a list of vertices keys of length $n-2$. See, for instance in Line 15 the code \texttt{['v3'}, \texttt{'v8',} \texttt{'v8',} \texttt{'v3',} \texttt{'v7',} \texttt{'v6',} \texttt{'v7']} corresponding to our sample tree graph \texttt{rt}.

Each position of the code indicates the parent of the remaining leaf with the smallest vertex label. Vertex \texttt{v3} is thus the parent of \texttt{v1} and we drop leaf \texttt{v1}, \texttt{v8} is now the parent of leaf \texttt{v2} and we drop \texttt{v2}, vertex \texttt{v8} is again the parent of leaf \texttt{v4} and we drop \texttt{v4}, vertex \texttt{v3} is the parent of leaf \texttt{v5} and we drop \texttt{v5}, \texttt{v7} is now the parent of leaf \texttt{v3} and we may drop \texttt{v3}, \texttt{v6} becomes the parent of leaf \texttt{v8} and we drop \texttt{v8}, \texttt{v7} becomes the parent of leaf \texttt{v6} and we may drop \texttt{v6}. The two eventually remaining vertices, \texttt{v7} and \texttt{v9}, give the last link in the reconstructed tree \citep{JPB-1991}.  

It is, as well possible to, first, generate a random \Pruefer code of length $n-2$ from a set of $n$ vertices  (see List.~\vref{list:22.2} Lines 1-9 ) and then, construct the corresponding tree of order $n$ by reversing the procedure illustrated above. The resulting tree graph is shown in Fig.~\vref{fig:22.2}.
\begin{lstlisting}[caption={Generating a tree graph with a random \Pruefer code.},label=list:22.2]
>>> verticesList = ['v1','v2','v3','v4','v5','v6','v7']
>>> n = len(verticesList)
>>> from random import seed, choice
>>> seed(101)
>>> code = []
>>> for k in range(n-2):
...     code.append( choice(verticesList) )
>>> print(code)
  ['v5', 'v7', 'v2', 'v5', 'v3']
>>> rt = RandomTree(prueferCode=code)
>>> rt
  *------- Graph instance description ------*
   Instance class   : RandomTree
   Instance name    : randomTree
   Graph Order      : 7
   Graph Size       : 6
   Valuation domain : [-1.00; 1.00]
   Attributes : ['name', 'order', 'vertices',
                 'valuationDomain', 'edges',
                 'prueferCode', 'size', 'gamma']
  *---- RandomTree specific data ----*
   Pruefer code  : ['v5', 'v7', 'v2', 'v5', 'v3']
>>> rt.exportGraphViz('tutPruefTree')
  *---- exporting a dot file for GraphViz tools ---------*
   Exporting to tutPruefTree.dot
   neato -Tpng tutPruefTree.dot -o tutPruefTree.png
\end{lstlisting}
\begin{figure}[ht]
\sidecaption[t]
\includegraphics[width=7cm]{Figures/22-2-tutPruefTree.pdf}
\caption[Tree graph generated with a random \Pruefer code]{Tree graph instance generated with a random \Pruefer code \texttt{['v5',} \texttt{'v7',} \texttt{'v2',} \texttt{'v5',} \texttt{'v3']}} 
\label{fig:22.2}       % Give a unique label
\end{figure}

Following from the bijection between a labelled tree and its \Pruefer code, we actually know that there exist $n^{n-2}$ different tree graphs with the same $n$ vertices.

Given a genuine graph, how can we recognise that it is in fact a tree instance ?

\section{Recognising tree graphs}
\label{sec:22.2}

Given a graph \texttt{g} of order $n$ and size $s$, the following 5 assertions A1, A2, A3, A4 and A5 are all equivalent (see \citep{JPB-1991}):
\begin{itemize}
\item [A1] \texttt{g} is a tree;
\item [A2] \texttt{g} is without (chordless) cycles and $n \,=\, s + 1$;
\item [A3] \texttt{g} is connected and $n \,=\, s + 1$;
\item [A4] Any two vertices of \texttt{g} are always connected by a unique path;
\item [A5] \texttt{g} is connected and dropping any single edge will always disconnect \texttt{g}.
\end{itemize}

Assertion A3, for instance, gives a simple test for recognising a tree graph. In case of a \emph{lazy evaluation} of the test in Listing~\vref{list:22.3} Line 3 below, it is opportune, from a computational complexity perspective, to first, check the order and size of the graph, before checking its potential connectedness. We provide the \texttt{isTree()} method\index{isTree@\texttt{isTree()}} for computing both these tests (see Line 8 in List.~\vref{list:22.3}).
\begin{lstlisting}[caption={Recognizing a tree graph.},label=list:22.3]
>>> from graphs import RandomGraph
>>> g = RandomGraph(order=8,edgeProbability=0.3,seed=62)
>>> if g.order == (g.size +1) and g.isConnected():
...     print('The graph is a tree ?', True)
... else:
...     print('The graph is a tree ?',False)   
  The graph is a tree ? True
>>> g.isTree()
  True  
\end{lstlisting}
The random graph of order $8$ and edge probability $30\%$, generated with seed $62$, is actually a tree graph instance, as confirmed by its graphviz drawing shown in Fig.~\vref{fig:22.3}.
\begin{lstlisting}
>>> g.exportGraphViz('test62')
  *---- exporting a dot file for GraphViz tools ---*
   Exporting to test62.dot
   fdp -Tpng test62.dot -o test62.png
\end{lstlisting}
\begin{figure}[ht]
\sidecaption[t]
\includegraphics[width=7cm]{Figures/22-3-test62.pdf}
\caption[Recognising a tree graph]{Recognising a tree graph. We may notice that vertex \texttt{v2} is actually situated in the \emph{centre} of the tree with a neighbourhood depth of 2} 
\label{fig:22.3}       % Give a unique label
\end{figure}

Yet, we still have to recover its corresponding \Pruefer code. Therefore, we may use the \texttt{tree2Pruefer()} method \index{tree2Pruefer@\texttt{tree2Pruefer()}} from the \texttt{TreeGraph} class. But, first, the instance class of graph \texttt{g} is changed to the \texttt{TreeGraph} type (see Line 2 in List.~\vref{list:22.4}). \index{TreeGraph@\texttt{TreeGraph()}}
\begin{lstlisting}[caption={Computing the \Pruefer code of a tree graph instance.},label=list:22.4]
>>> from graphs import TreeGraph
>>> g.__class__ = TreeGraph  
>>> g.tree2Pruefer()
  ['v6', 'v1', 'v2', 'v1', 'v2', 'v5']
\end{lstlisting}

In Fig.~\vref{fig:22.3}, we noticed that vertex \texttt{v2} is actually situated in the \emph{centre} of the tree with a neighbourhood depth of 2. Centres of a graph are the vertices with minimal neighbourhood depth. For finding such centre(s), the \texttt{Graph} class provides the \texttt{computeGraphCentres()} method.\index{computeGraphCentres@\texttt{computeGraphCentres()}} (see Lines 1-2 in List.~\vref{list:22.4}). Knowing now the centre of graph \texttt{g}, we may draw a correspondingly rooted and oriented tree with the \texttt{exportOrientedTreeGraphViz()} method from the \texttt{TreeGraph} class (see Fig.~\vref{fig:22.4}.\index{exportOrientedTreeGraphViz@\texttt{exportOrientedTreeGraphViz()}}  
\begin{lstlisting}[caption={Computing the centres of a tree and drawing a rooted and oriented tree.},label=list:22.5]
>>> g.computeGraphCentres()
  {'v2': 2}
>>> g.exportOrientedTreeGraphViz(\
...       fileName='rootedTree', root='v2')
  *---- exporting a dot file for GraphViz tools
   Exporting to rootedTree.dot
   dot -Grankdir=TB -Tpng rootedTree.dot -o rootedTree.png
\end{lstlisting}
\begin{figure}[ht]
\sidecaption[t]
\includegraphics[width=6cm]{Figures/22-4-rootedTree.pdf}
\caption{Drawing an oriented tree rooted at its centre} 
\label{fig:22.4}       % Give a unique label
\end{figure}

Let us now turn our attention toward a major application of tree graphs, namely \emph{spanning trees} and \emph{forests} related to graph traversals.

\section{Spanning trees and forests}
\label{sec:22.2}

With the \texttt{RandomSpanningTree} class\index{RandomSpanningTree@\texttt{RandomSpanningTree} class} we may generate, from a given connected graph \texttt{g} instance, \emph{uniform} random instances of a spanning tree by using \Wilson's algorithm (see List.~\vref{list:22.6} and Fig.~\vref{fig:22.5}).
\begin{lstlisting}[caption={Generating uniform random spanning trees.},label=list:22.6]
>>> from graphs import RandomGraph,\
...                    RandomSpanningTree
>>> g = RandomGraph(order=9,\
...                 edgeProbability=0.4,seed=100)
>>> spt = RandomSpanningTree(g)
>>> spt
  *------- Graph instance description ------*
   Instance class   : RandomSpanningTree
   Instance name    : randomGraph_randomSpanningTree
   Graph Order      : 9
   Graph Size       : 8
   Valuation domain : [-1.00; 1.00]
   Attributes       : ['name','vertices','order',
                       'valuationDomain',
                       'edges','size','gamma',
                       'dfs','date', 'dfsx',
                       'prueferCode']
  *---- RandomTree specific data ----*
   Pruefer code  : ['v7','v9','v5','v1','v8','v4','v9']
>>> spt.exportGraphViz(fileName='randomSpanningTree',\
...                    WithSpanningTree=True)
  *---- exporting a dot file for GraphViz tools -----*
   Exporting to randomSpanningTree.dot
   [['v1','v5','v6','v5','v1','v8','v9','v3','v9','v4',
      'v7','v2','v7','v4','v9','v8','v1']]
   neato -Tpng randomSpanningTree.dot\
         -o randomSpanningTree.png
\end{lstlisting}
\begin{figure}[ht]
\sidecaption[t]
\includegraphics[width=7cm]{Figures/22-5-randomSpanningTree.pdf}
\caption{Random spanning tree} 
\label{fig:22.5}       % Give a unique label
\end{figure}

\Wilson's algorithm only works for connected graphs \citep{WIL-1996}. More general, and in case of a not connected graph, the \texttt{RandomSpanningForest} class\index{RandomSpanningForest@\texttt{RandomSpanningForest} class} generates a not necessarily uniform random instance of a \emph{spanning forest} --one or more random tree graphs-- generated from a random \emph{depth first search} of the graph components' traversals (see List.~\vref{list:22.7} and Fig.~\vref{fig:22.6}).
\begin{lstlisting}[caption={Computing spanning forests over disconnected graphs. },label=list:22.7]
>>> g = RandomGraph(order=15,\
...                 edgeProbability=0.1,seed=140)
>>> g.computeComponents()
  [{'v12','v01','v13'}, {'v02','v06'},
   {'v08','v03','v07'}, {'v15','v11','v10','v04','v05'},
   {'v09','v14'}]
>>> spf = RandomSpanningForest(g,seed=100)
>>> spf.exportGraphViz(fileName='spanningForest',\
...                    WithSpanningTree=True)
  *---- exporting a dot file for GraphViz tools -----*
    Exporting to spanningForest.dot
    [['v03','v07','v08','v07','v03'],
     ['v13','v12','v13','v01','v13'],
     ['v02','v06','v02'],
     ['v15','v11','v04','v11','v15',
            'v10','v05','v10','v15'],
     ['v09','v14','v09']]
  neato -Tpng spanningForest.dot -o spanningForest.png
\end{lstlisting}
\begin{figure}[ht]
\sidecaption[t]
\includegraphics[width=7cm]{Figures/22-6-spanningForest.pdf}
\caption{Random spanning forest} 
\label{fig:22.6}       % Give a unique label
\end{figure}

\section{Maximum determined spanning forests}
\label{sec:22.3}

In case of valued graphs --supporting weighted edges-- we may finally construct a \emph{most determined} spanning tree (or forest if not connected) using \Kruskal's greedy minimum-spanning-tree algorithm on the dual valuation of the graph \citep{KRU-1956}\footnote{\Kruskal's algorithm is a minimum-spanning-tree algorithm which finds an edge of the least possible weight that connects any two trees in the forest.}.

In Listing~\vref{list:22.8} we generate, for instance, a randomly valued graph with five vertices and seven edges bipolar-valued in [-1.0; 1.0].\index{RandomValuationGraph@\texttt{RandomValuationGraph} class}
\begin{lstlisting}[caption={Generating randomly bipolar-valued graphs.},label=list:22.8]
>>> from graphs import RandomValuationGraph
>>> g = RandomValuationGraph(order=5,seed=2)
>>> g
  *------- Graph instance description ------*
   Instance class   : RandomValuationGraph
   Instance name    : randomGraph
   Graph Order      : 5
   Graph Size       : 7
   Valuation domain : [-1.00; 1.00]
   Attributes       : ['name', 'order',
                 'vertices', 'valuationDomain',
                 'edges', 'size', 'gamma']
\end{lstlisting}
To inspect the edges' actual weights, we first transform the graph into a corresponding digraph (see Line 1 in List.~\vref{list:22.9}) and use the \texttt{showRelationTable()} method (see Line 2) for printing its symmetric adjacency matrix. 
\begin{lstlisting}[caption={Symmetric relation table},label=list:22.9]
>>> dg = g.graph2Digraph()
>>> dg.showRelationTable()
  *---- Relation Table -----
    S   |  'v1'	 'v2'  'v3'  'v4'  'v5'	  
   -----|------------------------------
   'v1' |  0.00	 0.91  0.90 -0.89 -0.83	 
   'v2' |  0.91	 0.00  0.67  0.47  0.34	 
   'v3' |  0.90	 0.67  0.00 -0.38  0.21	 
   'v4' | -0.89	 0.47 -0.38  0.00  0.21	 
   'v5' | -0.83	 0.34  0.21  0.21  0.00	 
   Valuation domain: [-1.00;1.00]
\end{lstlisting}

To compute the most determined spanning tree or forest, we can use the \texttt{Best\-DeterminedSpanningForest} constructor.\index{BestDeterminedSpanningForest@\texttt{BestDeterminedSpanningForest} class}
\begin{lstlisting}[caption={Computing best determined spanning forests.},label=list:22.10]
>>> from graphs import\
                BestDeterminedSpanningForest
>>> mt = BestDeterminedSpanningForest(g)
>>> print(mt)
  *------- Graph instance description ------*
   Instance class   : BestDeterminedSpanningForest
   Instance name    : bdSpanningForest
   Graph Order      : 5
   Graph Size       : 4
   Valuation domain : [-1.00; 1.00]
   Attributes       : ['name','vertices','order',
                       'valuationDomain',
                       'edges','size','gamma',
                       'dfs','date',
                       'averageTreeDetermination']
  *---- best determined spanning tree  ----*
   Depth first search path(s) :
   [['v1','v2','v4','v2','v5','v2','v1','v3','v1']]
   Average determination(s) : [Decimal('0.655')]
\end{lstlisting}

The random graph \texttt{g} is connected and, hence, admits a single spanning tree of maximum mean determination = $(0.47 + 0.91 + 0.90 + 0.34)/4 = 0.655$ (see Lines 9, 6 and 10 in List.~\vref{list:22.8} and Fig.~\vref{fig:22.7}).
\begin{lstlisting}
>>> mt.exportGraphViz(\
...      fileName='bestDeterminedspanningTree',\
...      WithSpanningTree=True)
  *---- exporting a dot file for GraphViz tools ---*
   Exporting to spanningTree.dot
   [['v4','v2','v1','v3','v1','v2','v5','v2','v4']]
   neato -Tpng bestDeterminedSpanningTree.dot\
         -o bestDeterminedSpanningTree.png
\end{lstlisting}
\begin{figure}[ht]
\sidecaption[t]
\includegraphics[width=7cm]{Figures/22-7-bestDeterminedSpanningTree.pdf}
\caption{Best determined spanning tree} 
\label{fig:22.7}       % Give a unique label
\end{figure}

One may easily verify that all other potential spanning trees, including instead the edges \{\texttt{v3}, \texttt{v5}\} and/or \{\texttt{v4}, \texttt{v5}\}, will show a lower average determination.

\vspace{1cm}

Chapter~\ref{sec:23}, the last on undirected graphs, is devoted to different models of perfect graphs, namely split, interval, comparability and permutation graphs. 
 
%%%%%%% The chapter bibliography
%\normallatexbib
%\clearpage
%\phantomsection
%\addcontentsline{toc}{section}{Chapter Bibliography}
%\input{02-mainMatters/22-chapterTreeGraphs.bbl}
\bibliographystyle{spbasic}
\bibliography{03-backMatters/reference}

\bibliographystyle{spbasic}
\bibliography{03-backMatters/reference}

\bibliographystyle{spbasic}
\bibliography{03-backMatters/reference}

\bibliographystyle{spbasic}
\bibliography{03-backMatters/reference}
