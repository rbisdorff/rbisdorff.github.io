\Extrachap{Introduction}
\label{sec:0}

\phantomsection
\addcontentsline{toc}{section}{The editing strategy}
\section*{The editing strategy}
\label{sec:0.1}

The reader will find in the five parts of this monograph several series of tutorials and advanced topics that present and illustrate computational methods and tools mainly useful in the field of Multiple Criteria Decision Aid and Decision Analysis. These methods and tools were designed and implemented in Python3 by the author over the last decade in order to support the preparation and illustration of a Master Course on Algorithmic Decision Theory taught at the University of Luxembourg from 2010 to 2020.

Each chapter illustrates a specific preference modelling aspect, like building a best choice recommendation, ranking a set of potential decision alternatives, or computing the winner of an election. Explicit Python programming examples, purposely kept elementary, are shown in numerous terminal session style listings. A complete list of the numbered listings, shown over all the chapters, is printed in the Appendix. 

These programming examples were all checked with the \texttt{doctest} module of the standard Python3 library and should work effectively as such either, in a Python3 shell console, or for sure in an \texttt{ipython} console. Some chapters will rely on a given data file that is made available in the \texttt{examples} directory of the \Digraph resources. 

For easing their reading, most chapters do not provide detailed mathematical developments and proofs. Readers interested in such details are invited to consult the references listed separately at the end of each chapter. Most of these references provide a full text access to the corresponding document on the \href{https://orbilu.uni.lu/}{https://orbilu.uni.lu/} repository of the University of Luxembourg.

Readers interested in the technical aspects of the organisation and implementation of the collection of \Digraph Python modules are invited to consult the reference manual: \href{https://digraph3.readthedocs.io/en/latest/techDoc.html}{https://digraph3.readthedocs.io/en/latest/techDoc.html}.

\phantomsection
\addcontentsline{toc}{section}{Organisation of the book}
\section*{Organisation of the book}
\label{sec:0.2}

The book is organised into five parts.
\vspace{0.5cm}

\textbf{Part I} presents three chapters introducing the \Digraph programming resources and the main formal objects discussed in this book, namely \emph{bipolar-valued digraphs} and, in particular, \emph{outranking digraphs}.

In Chapter~\ref{sec:1}, the reader will gain contact with the \Digraph Python resources. First are given the installation instructions and the list of the main \Digraph Python modules with their purpose. A Python terminal session using the root \texttt{digraphs} module eventually illustrates how to generate, save and inspect a random crisp digraph.

Chapter~\ref{sec:2} introduces the bipolar-valued digraph model --the root type of all our digraph models--. A randomly bipolar-valued digraph instance is generated. Drawing the digraph, separating its asymmetric and symmetric parts, or its border and inner parts, is illustrated. The initial digraph instance may be reconstructed by epistemic disjunctive fusion from these respective parts. Dual, converse and codual versions, as well as symmetric and transitive closures are generated. Complete, empty and indeterminate digraphs are eventually presented.

Chapter~\ref{sec:3} presents the bipolar-valued outranking digraph --the main formal object used and discussed in this monograph--. After illustrating its hybrid type --it is conjointly a multiple criteria performance tableau and a bipolar-valued digraph modelling the outranking situations between the given performance records-- pairwise comparisons and the recoding of the digraph characteristic valuation are illustrated. The codual version of the outranking renders the corresponding strict outranking digraph, i.e. its asymmetric part. 
\vspace{0.5cm}

\textbf{Part II} illustrates in eight methodological chapters multiple criteria performance evaluation models and decision algorithms. These chapters are mostly problem oriented.

Chapter~\vref{sec:4} presents the \Rubis best choice recommender system. The approach is illustrated with a best office site selection problem. We show how to explore a given performance tableau and compute the corresponding outranking digraph. After presenting the pragmatic principles that govern our best choice recommendation algorithm we solve the best office site choice problem.

Chapter~\vref{sec:5} illustrates a way of creating a new PerformanceTableau instance by editing a given template with 5 decision alternatives, 3 decision objectives and 6 performance criteria. We discuss in detail how to edit the decision alternatives, the decision objectives, the family of performance criteria, and finally, the evaluations of the decision alternatives on the performance criteria.

Chapter~\vref{sec:6} describes the \Digraph resources for generating random multiple criteria performance tableaux. These random performance tableaux in- stances, mainly meant for illustration and training purposes, were serving the preparation and illustration of the Algorithmic Decision Theory Course lectures. The random generators propose several useful models like a Cost-Benefit tableau, a three Objectives –-economic, societal and environmental-– tableau, and eventually an academic performance tableau.

Chapter~\vref{sec:7} is more specifically devoted to handling linear voting profiles and computing the winner of such election results like the simple majority or the instant-run-off winner. By following \Condorcet 's recipe, we consider pairwise comparisons of election candidates and balance the number of times the first beats the second against the number of times the second beats the first in order to obtain a majority margins digraph, in fact a bipolar-valued digraph. When the voters express contradictory linear voting profiles one may naturally observe cyclic social preferences without seeing any paradox in this situation. Finally, the chapter presents a more politically realistical random generator for linear voting profiles who takes into account pre-election polls.

Chapter~\vref{sec:8} introduces several algorithms for solving the ranking problem with a bipolar-valued outranking digraph. The \Copeland, \NetFlows, \Kemeny, \Slater, \Kohler and the \RankedPairs ranking rules are illustrated with a random outranking digraph. The fitness of their respective ranking result is measured with a bipolar-valued version of \Kendall 's ordinal correlation index.

Chapter~\vref{sec:9} applies order statistics for sorting a set $X$ of n potential decision alternatives, evaluated on $m$ incommensurable performance criteria, into $q$ quantile equivalence classes. The sorting algorithm is based on pairwise outranking characteristics involving the quantile class limits observed on each criterion. Thus we may implement a weak ordering algorithm of complexity $O(nmq)$.

Chapter~\vref{sec:10} addresses the problem of rating multiple criteria performances of a set of potential decision alternatives with respect to performance quantiles learned from historical data gathered from similar decision alternatives observed in the past. We show how to learn performance quantiles from historical performance tableaux. New performance records may now be rated with respect to these learned quantile norms.

Chapter~\vref{sec:11} tackles the ranking of big multiple criteria performance tableaux with thausands or millions of records. To effectively compute rankings from performance tableaux of these sizes, the chapter proposes a collection of cythonized --C-compiled and optimised-- modules that may be run on Linux Debian HPC equipement as available, for instance, at the University of Luxembourg.
\vspace{0.5cm}

\textbf{Part III} delivers three realistic algorithmic decision making case studies.

Chapter~\vref{sec:12} presents a case study concerning the building of a best choice recommendation for Alice, a German student who wants some advice concerning the choice of her future University studies. We present Alice’s performance tableau –-potential foreign language study programs, her decision objectives, performance criteria and performance grades–- and build a best choice recommendation for her. A thorough robustness analysis confirms a very best choice.

In Chapter~\vref{sec:13} we are resolving with our \Digraph resources a ranking decision problem based on published data from the Times Higher Education (THE) World University Rankings 2016 by Computer Science (CS) subject. We first have a look into the THE multiple criteria ranking data with short Python scripts. In a second section, we relax the commensurability hypothesis of the ranking criteria and show how to similarly rank with multiple incommensurable performance criteria of solely ordinal significance. A third section is finally devoted to introduce quality measures for qualifying ranking results.

Chapter~\vref{sec:14} presents and discusses how to rate with the help of our \Digraph resources the apparent student enrolment quality of 41 German higher education institutions. The multiple criteria performance tableau, we use, is based on internet published by \Spiegel magazine in 2004 concerning nearly 50,000 students, enrolled in one of fifteen popular academic subjects, like German Studies, Life Sciences, Psychology, Law or Computer Science.

In Chapter~\vref{sec:15}, we propose a series of decision problems of various difficulties which may serve as exercises and exam questions for an Algorithmic Decision Theory or Multiple Criteria Decision Analysis course. They cover selection, ranking and rating decision problems.
\vspace{0.5cm}

\textbf{Part IV} presents in five chapters more advanced topics showing some pearls of bipolar-valued epistemic logic.

Starting from a motivating decision problem about how to list, from the best to the worst, a set movies that are star-rated by journalists and movie critics, Chapter~\vref{sec:16} shows that \Kendall’s ordinal correlation index tau may be extended to a relational bipolar-valued equivalence measure of bipolar-valued digraphs. This finding gives way, on the one hand, to measure the fitness and fairness of multiple criteria ranking rules. On the other hand, it provides a tool for illustrating preference divergences between decision objectives and criteria.

We illustrate in Chapter~\vref{sec:17}, first, the concept of graph kernel, i.e. maximal independent set of vertices. In non-symmetric digraphs the kernel concept becomes richer and separates into initial and terminal kernels. In, furthermore, lateralised outranking digraphs, initial and terminal kernels become separate and may deliver suitable first resp. last choice recommendations. After commenting the tractability of kernel computations, we close the chapter with the solving of bipolar-valued kernel equation systems.

In Chapter~\vref{sec:18} we propose to link a qualifying significance majority for outranking situations with a required $\alpha\%$-confidence level. We model therefore the significance weights as random variables following more or less widespread distributions around an average weights value that corresponds to the given deterministic significance weight. As the bipolar-valued random credibility of an outranking situation hence results from the simple sum of positive or negative independent random variables, we may apply the Central Limit Theorem (CLT) for computing the bipolar-valued likelihood that the expected significance majority margin will indeed be positive, respectively negative.

The required cardinal significance weights of the performance criteria represent the ’Achilles’ heel of our outranking approach. Rarely will indeed a decision maker be cognitively competent for suggesting precise decimal-valued criteria significance weights. More often, the decision problem will involve more or less equally important decision objectives with more or less equi-significant criteria per objective. In Chapter~\vref{sec:19} we study the robustness of the outranking digraph when the criteria significance weights faithfully indicate solely an order of importance. This approach leads furthermore to the concept of unopposed or Pareto efficient multiobjective choices.

In a social choice context, where decision objectives would match different political parties, Pareto efficient choices represent in fact multipartisan social choices. Chapter~\vref{sec:20} shows that they may judiciously deliver the primary selection in a two stage election system. Our outranking model is based on bipolar approvals-disapprovals of ``\emph{as well as performing as}'' statements. A similar approach is put into practice with bipolar approval-disapproval voting systems. When converting such approval-disapproval voting ballots into corresponding performance records, one obtains a $(-1,0,1)$-valued evaluative voting system. We eventually show in this chapter that in bipolar-approval voting systems, the winner tends to be among the more or less multipartisan candidates.
\vspace{0.5cm}

\textbf{Part V}, with three chapters, concerns eventually simple undirected graphs and illustrates computational resources for working with such graphs.

Chapter~\vref{sec:21} introduces bipolar-valued undirected graphs and illustrates several special graph models and algorithms like Q-coloring, MIS and clique enumeration, line graphs and maximal matchings, grid graphs, and n-cycle graphs with their non-isomorphic maximal independent sets of vertices.

Chapter~\vref{sec:22} specifically addresses working with tree graphs and graph forests. We illustrate how to generate and recognise random tree graphs and how to compute the centres of a tree and draw a rooted and oriented tree. Finally, algorithms for computing spanning trees and forests are presented.

Chapter~\vref{sec:23} eventually presents some famous classes of perfect graphs, namely comparability, interval, permutation and split graphs. We first present an example of a graph which is at the same time a triangulated, a comparability, a split and a permutation graph. The importance to be an interval is illustrated with \Berge ’s mystery story. We discuss furthermore the generation of permutation graphs and close with how to recognise that a given graph is in fact a permutation graph.

\phantomsection
\addcontentsline{toc}{section}{Highlights}
\section*{Highlights}
\label{sec:0.3}

Contrary to what is generally thought, it is the preparation of the multiple criteria performance tableau that takes most of the time, not running any decision algorithm. Designing adequate performance grading criteria functions for each decision objective and collecting  adequate grades is essential for the success of the decision making. This is a very critical and essential step. Chapter~\ref{sec:4} and Chapter~\ref{sec:5} illustrate both how to edit a coherent performance tableau. In order to discover examples of potential performance tableaux, we provide in Chapter~\vref{sec:6} random generators for several kinds of performance tableau. 

Once the performance tableau is ready, starts the thrilling step of discovering the resulting outranking digraph. Are there many chordless outranking circuits? What is the degree of symmetry of the outranking relations? What is the degree of transitivity?  If the number of potential decision alternatives is small --less than 20-- one may try, in the case of a best choice selection problem to compute initial prekernels in order to find potential best choice decision alternatives? Chapters \vref{sec:4}, \vref{sec:12} and \vref{sec:17} are illustrating and discussing this computational problem.

Comparing ranking rules working on bipolar-valued outranking digraphs  with multiple performance tableaux of different kinds: Cost-Benefit, 3-Objectives, academic a.-o., has made us confident about the fact that a convincing criterion for judging the quality of a ranking result may not to be found by mathematical properties, like \Kemeny optimality or \Condorcet consistency. More useful seams to be the fair balancing of decision objectives and performance criteria. To this respect, the \NetFlows ranking rule appears to give the fairest and most convincing multiple criteria rankings. Chapter~\vref{sec:8} on ranking rules, the ranking case study of Chapter~\vref{sec:13} and Chapter~\vref{sec:16} illustrate this point.

The bipolar-valued epistemic logic in which our decision algorithms are computing and expressing their decision solutions provides essential assistance for coping with missing data and imprecise performance gradings. An efficient robustness analysis becomes furthermore available for handling, on the one side, uncertain criteria significance weight leading in Chapter~\ref{sec:18} to $\alpha\%$-confident outranking digraphs. On the other side, we are able to compute in Chapter~\vref{sec:19} robust outranking digraphs when solely ordinal criteria significance weights are given.

The same kind of robustness analysis allows us in Chapter~\vref{sec:20} to propose strategies for tempering plurality tyranny effects in social choice problems by favouring multipartisan candidates, like two-stage elections with multipartisan primary selection of candidates or bipolar-valued approval voting.



% %%%%%%% The chapter bibliography
% % %\normallatexbib
% \clearpage
% % \phantomsection
% % \addcontentsline{toc}{section}{Chapter Bibliography}
% \bibliographystyle{spbasic}
% % \typeout{}
% \bibliography{03-backMatters/reference}
% % %\input{02-mainMatters/00-chapterIntroDigraph3.bbl}
