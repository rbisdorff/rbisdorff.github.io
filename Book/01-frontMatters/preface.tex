%%%%%%%%%%%%%%%%%%%%%%preface.tex%%%%%%%%%%%%%%%%%%%%%%%%%%%%%%%%%%%%%%%%%
% sample preface
%
% Use this file as a template for your own input.
%
%%%%%%%%%%%%%%%%%%%%%%%% Springer %%%%%%%%%%%%%%%%%%%%%%%%%%

\preface

%% Please write your preface here
% Use the template \emph{preface.tex} together with the Springer document class SVMono (monograph-type books) or SVMult (edited books) to style your preface in the Springer layout.

% A preface\index{preface} is a book's preliminary statement, usually written by the \textit{author or editor} of a work, which states its origin, scope, purpose, plan, and intended audience, and which sometimes includes afterthoughts and acknowledgments of assistance. 

% When written by a person other than the author, it is called a foreword. The preface or foreword is distinct from the introduction, which deals with the subject of the work.

% Customarily \textit{acknowledgments} are included as last part of the preface.

\paragraph{Origin}

In this book we present series of tutorials and advanced topics originally written as documentation parts for the \Digraph collection of Python ressources over the last ten years. These \Digraph resources -- like the \texttt{outrankingDigraphs} module -- were essentially used for the preparation and illustration of the \emph{Algoritmic Decision Theory} Course taught at the University of Luxembourg from 2010-2020. Some parts also served for preparing and illustrating the Lectures of a \emph{Computational Statistics} Course like the \texttt{randomNumbers} module. Curious readers will also discover some resources like the \texttt{arithmetics} module, used for preparing and illustrating a first Semester Course on Discrete Mathematics.

\paragraph{Scope}

The \Digraph documentation, available \href{https://digraph3.readthedocs.io/en/latest/}{on the Read The Docs site}, describes the Python resources for implementing decision aid algorithms in the context of a bipolar-valued outranking approach. These programming resources are useful in the field of Algorithmic Decision Theory and more specifically for the outranking approach of Multiple Criteria Decision Aid (MCDA).

In this scientific filed, we address essentially three kinds of usage. First, we present methods an illustrate computing tools for solving either a multiple criteria best choice selection or a dual worst choice rejection problem, mainly interesting for management or political studies. We also address the problem of how to list a set of items with multiple incommensurable performance criteria either from the best to the worst (ranking problem) or from the worst to the best (ordering problem), mainly interesting for schedulers or designers of recommender systems. The third kind of usage should mainly be of interest for performance auditors, as we present methods and tools for relative or absolute quantiles rating of mulitiple criteria performance records. 

The \Digraph ressources do not provide a professional Python software library. The collection of Python modules, we describe in this book, was not built following any professional software development methodology. The design of classes and methods was kept as simple and elementary as was opportune for the author. Sophisticated and criptic overloading of classes, methods and variables is more or less avoided all over. A simple copy, paste and ad hoc customization development strategy was generally preferred. As a consequence, the \Digraph modules keep a large part of independence and are hence easier to maintain and adapt.  Furthermore, the development of the \Digraph modules being spread over a decade, our programming style did evolve with our growing experience and the changes and enhancement coming up with the ongoing new releases of the standard Python3 libraries. The required backward compatibility necessarily introduced so with time some notation and programming technique changes.

Finally, the book is not presenting mathematical developments and proofs. Readers interested in the mathematical backgroung of our decision methods and tools are invited to consult the references provided at Chapter level. Full texts of most of the given references may be downloaded from the \href{https://orbilu.uni.lu/}{https://orbilu.uni.lu/} repository of the University of Luxembourg. 

\paragraph{Plan}

\noindent The book contains five parts. The first Part presents three a chapters introducing the \Digraph resources and the main formal objects discussed in this book, namely bipolar-valued digraphs and, in particular, outranking digraphs.

The second, methodological Part illustrates in eight chapters decision methods and tools and performance evaluation models. These methodological chapters are mostly problem oriented and show how to build a best choice recommendation or compute the winner of an election. Others show how to linearly rank or order incommensurable multiple criteria performance records. We also show how to rate such performance records with order-statistical quantiles.

The third Part, with four Chapters, present three case studies: --one for building a best choice recommendation for selecting a post-secondary study program, --one for ranking world famous Computer Science Dpts, and --a third one for rating the recruitment quality of German Universities. The case studies are followed by a set of exercises suitable for a Course on Algorithmic Decision Theory.

The fourth Part presents in five Chapters more advanced topics showing some pearls of bipolar-valued epistemic logic. We first generalize Kendall's ordinal correlation index to bipolar-valued digraphs in order to measure the fitness of our ranking results. A second and important advanced topic concerns the effective computation of digraph kernels with solving of \Berge kernel equation systems. We furthermore introduce confidence levels for outranking situations when facing uncertain criteria significance weights. We also develop a robustness analysis of the outranking digraph when facing only ordinal criteria significance weights. The last advanced topic concerns the tempering of plurality tyranny effects in social choice problems with the help a.o. of bipolar approval voting. 

A last Part, with three Chapters, concerns undirected graphs and present operational aspects of working with graphs, like q-coloring, enumerating maximal independent sets (MISs) and cliques, computing maximal matchings, simulating Metropolis random walks and enumerating the non isomorphic MISs of the n-cycle graph. A second Chapter is more specifically devoted to tree graphs and graph forests. A very last Chapter, inspired by Martin Golumbic ‘s book on Algorithmic Graph Theory and Perfect Graphs, introduces split, interval and permutation graphs.

\paragraph{Intended audience}

The material in this book is first valuable for master students and doctoral candidates in Computer Science, Engineering Sciences or Computational Management Sciences taking a course on Algorithmic Decision Theory, Multiple criteria Decision Aid or Decision Analysis. A certain experience in computer programming, in particular Python, is certainly easing the lecture, but not a prerequisite.

But it may also be of interest for professional designers of web recommender systems, as we present Python methods and tools for ranking multiple criteria performance records from best to worst, especially when facing big performance tableaux.

Similarly, the relative and absolute rating methods and tools we present in this book might be of interest for professional private or public performance auditors. 

\paragraph{Aknowledgments}

This book contains many ideas, methods and tools that are not only the author’s. They have been shared and enhanced with friends and colleagues: 

\vspace{0.5cm}
\begin{minipage}{7cm}
\emph{Denis Bouyssou, Luis Dias,}\\ 
\emph{Claude Lamboray, Patrick Meyer,}\\
\emph{Vincent Mousseau, Alex Olteanu,}\\
\emph{Marc Pirlot, late Bernard Roy,}\\
\emph{Alexis Tsouki\`as, Thomas Veneziano,}\\
and especially,\\
\emph{late Marc Roubens}.
\end{minipage}\quad
\begin{minipage}{3cm}
\includegraphics[width=3cm]{Figures/Marc-Roubens.jpg} \\
{\tiny \emph{Marc Roubens}}
\end{minipage}

\vspace{0.3cm}
Their help is gratefully acknowledged.

\vspace{\baselineskip}
\begin{flushright}\noindent
Luxembourg, 2021\hfill {\it Raymond Bisdorff}\\
%month year\hfill {\it Firstname  Surname}\\
\end{flushright}


